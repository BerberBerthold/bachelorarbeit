\subsection{Concepts From Algebraic Geometry}

In the following, $K$ is a field of characteristic $0$ and $G$ a linear algebraic group, that is a group whose set is an affine variety, and whose multiplcation and inversion are morphisms of affine varieties.

\begin{proposition}\label{rabbi}
  Let $V = K^n$ for some $n \in \mathbb{N}$.
  For a polynomial $p \in K[V] = K[\{X_i\}_{i\in[n]}]$, the set $ X_p := \{\, v \in V \mid p(v) \neq 0 \,\}$ is an affine variety with the coordinate ring $K[X_p] = K[\{X_i\}_{i \in [n]}, p^{-1}]$.
\end{proposition}

\begin{proof}
  The set $X_p$ is not an algebraic set itself.
  The trick (the ``Rabinovich-trick'') is to consider $X_p$ as a subset of $K \times V$.
  We do this as follows:
  Consider $\tilde{X}_p := Z \left( X_0 \cdot p -1 \right) \subseteq K \times V$.
  We see that $X_p$ corresponds to $\tilde{X}_p$ by noticing that $\tilde{X}_p = \{\, (1/p(v), v) \in K \times V \mid v \in X_p \,\}$, which therefore means we have the one-to-one correspondence $(1/p(v),v) \leftrightarrow v$.
  By definition $\tilde{X}_p$ is an affine variety, and it is also easy to see that $K[X_p] \cong K[\tilde{X}_p] = K[K\times V] / (X_0 \cdot p -1)$ via $X_0 \leftrightarrow p^{-1}$, including evaluations with the above described correspondence of points, which is easy to check.
\end{proof}

\begin{example}[The General Linear Group $\operatorname{GL}_n$]
  One of the most important examples is the general linear group $\operatorname{GL}_n$, which will be an essential theme in my work.
  By the above proposition this group is an affine variety via $p = \operatorname{det}$ with the coordinate ring $K[\{X_{i,j}\}_{i,j \in [n]}, \operatorname{det}^{-1}]$.
  This makes $\operatorname{GL}_n$ into a \textit{linear algebraic group}, that is a group which is an affine variety whose group operations of the multiplication and inversion are morphisms of affine varieties.
\end{example}

\begin{definition}[Algebraic Cohomomorphism For Product Spaces]\label{coh}
% We denote by $m$ the group multiplication of the group $G$.
% We want to view the pullback of $m$ as a map $m^\ast : K[G] \longrightarrow K[G] \otimes K[G]$, which makes sense, because $m$ and $\otimes$ are associative.
% The strict pullback, which I will call $ \hat{m} $, should be a map of the type $ K[G] \longrightarrow K[ G \times G] $, where $ f \mapsto f \circ m $.
% If we want to give the variables names, we can equivalently say it is a map $ \left. K[Z] \right|_G \longrightarrow \left. K[X,Y] \right|_{G \times G} $, where $ Z = \lbrace Z_1 , \ldots , Z_k \rbrace $, $X$ and $Y$ analogously (here, $ m $ canonically takes its left input via $ X $ and its right input via $ Y $).
  Let $m \colon U_1 \times U_2 \longrightarrow W$ be a morphism of affine varieties.
  The strict algebraic cohomomorphism, which we shall call $\hat{m}$, is a map of the type $K[W] \longrightarrow K[U_1 \times U_2]$.
  We have $ K[U_1 \times U_2] = K[\{X_k\}_{k\in[r]},\{Y_l\}_{l\in[s]}]$, where $\{X_k\}_{k\in[r]}$ and $\{Y_l\}_{l\in[s]}$ are generators of $K[U_1]$ and $K[U_2]$ respectively.
  % If we want to give the variables names, we can equivalently say it is a map $ \left. K[Z] \right|_W \longrightarrow \left. K[X,Y] \right|_{U_1 \times U_2} $, where $ Z = \lbrace Z_1 , \ldots , Z_k \rbrace $ $X$ and $Y$ analogously (here, $ m $ canonically takes its left input via $ X $ and its right input via $ Y $).
  We define the following map:
  % \begin{equation}
  %   \begin{aligned}
  %     t \colon K[\{X_k\}_{k\in[r]},\{Y_l\}_{l\in[s]}]
  %     & \longrightarrow K[\{X_k\}_{k\in[r]}] \otimes K[\{Y_l\}_{l\in[s]}] \\
  %     \sum_i \lambda_i \prod_j X_{j}^{d_{i,j}} \prod_k Y_{k}^{e_{i,k}} &\longmapsto \sum_i \lambda_i \prod_j X_{j}^{d_{i,j}} \otimes \prod_k Y_{k}^{e_{i,k}}
  %   \end{aligned}
  % \end{equation}
    \begin{equation}
    \begin{aligned}
      t \colon K[U_1 \times U_2]
      & \longrightarrow K[U_1] \otimes K[U_2]\\
      \sum_i \lambda_i \prod_j X_{j}^{d_{i,j}} \prod_k Y_{k}^{e_{i,k}} &\longmapsto \sum_i \lambda_i \prod_j X_{j}^{d_{i,j}} \otimes \prod_k Y_{k}^{e_{i,k}}
    \end{aligned}
  \end{equation}
  This is independent of the choice of generators and independent of the representatives and therefore well-defined.
  It is even an isomorphism.
  Now, finally, we define $m^\ast := t \circ \hat{m} : K[W] \longrightarrow K[U_1] \otimes K[U_2]$.
  Most literature still also calls $m^\ast$ the algebraic cohomomorphism of $m$, the $K$-algebra of all evaluation maps induced by $K[U_1] \otimes K[U_2]$ is equal to $K[U_1 \times U_2]$.
\end{definition}

\begin{remark}
One might ask why we want to look at these objects $ m^\ast \left( f \right) $ instead of $ \hat{m} \left( f \right) $.
Really, these objects are hardly different (the spaces are isomorphic), but it helps to formalize performing operations only on the ``left part'' or the ``right part'', as we will soon see.
This is an approach that \cite{DK15} follows, but other literature such as \cite{Stu08} (and probably also Cayley) rather consider $ \hat{m} \left( f \right) $ written as $ \hat{m} \left( f \right) = f (XY) $.
To give a very simple example:
If $f \in \left. K[Z] \right|_G$, we will write $\operatorname{id} \otimes \frac{\partial}{\partial Z_i} (m^\ast f)$ as in \cite{DK15}, whereas \cite{Stu08} would write $\frac{\partial}{\partial Y_i} f(XY)$.
\end{remark}

\subsection{Concepts From Invariant Theory}

\begin{definition}[Regular Action, Finite Rational Representation]
  Let $G$ be a linear algebraic group, $X$ an affine variety.
  We call an action $G \times X \longrightarrow X$ a \textbf{regular action}, if and only if $\mu$ is a morphism of affine varieties.
  We say \textbf{$ G $ acts regularly on $ X $}, and we also call $X$ a \textbf{$G$-variety}.

  For a finite-dimensional vector space $V$, let $\mu \colon G \times V \longrightarrow V$ be a representation in the classical sense, that is for all $g \in G$ we have $D_\mu (g) := (v \mapsto \mu(g,v)) \in \operatorname{GL}(V)$.
  We call $\mu$ a \textit{finite rational representation} if and only if it is regular.
  % We can view $V$ as an affine variety with respect to a basis.
  % We call $\mu$ a \textbf{finite rational representation} iff it is regular with respect to any basis.
  
  % For a finite dimensional vector space $V$ we call $D \colon G \longrightarrow \operatorname{GL}(V)$ a \textbf{regular representation} iff $\mu_D := ((g,v) \mapsto D(g)(v))$ is a regular action.
  % If an action $\mu \colon G \times V \longrightarrow V$ is given such that $D_\mu (g) := (v \mapsto g.v) \in \operatorname{GL}(V)$, we also call $\mu$ a regular representation (such $\mu$ and $D$ are in bijection).
\end{definition}

% \begin{remark}
%   If $V$ is a finite-dimensional vector-space and $\mu \colon G \times V \longrightarrow V$ is a representation of $G$, then $\mu$ is a finite rational representation if and only if it is regular with respect to a single basis.
%   Really, one can define 
% \end{remark}

\begin{definition}[Rational Representation]\label{rr}
  Let $V$ be a vector space (not necessarily finite dimensional), and $ \mu : G \times V \longrightarrow V $ an action.
  We call $ \mu $ a \textbf{rational representation} if and only if there exists a linear map $ \mu^\prime \colon V \longrightarrow K[G] \otimes V $ such that $ \mu \left( \sigma , v \right) = \left( \left( \epsilon_\sigma \otimes \operatorname{id} \right) \circ \mu^\prime \right) \left(v\right) $.
  We also call $V$ a rational representation of $G$.
\end{definition}

\begin{definition}\label{funrep}
  Let $\mu \colon G \times X \longrightarrow X$ be a regular action.
  We define an action $\bar{\mu} \colon G \times K[X] \longrightarrow K[X]$ via $\bar{\mu}(\sigma,f)(x) := f(\mu(\sigma^{-1},x))$, or $\sigma.f (x)\ref{da} := f( \sigma^{-1}.x )$, where  $\sigma \in G$, $f \in K[X]$ and $x \in X$.
  This action is obviously regular, but it is also easily shown that it is in fact a rational representation:
  If $\tilde{\mu} \colon G \times X \longrightarrow X$ is the morphism of affine varieties (it is in fact a left action) defined by $ \tilde{\mu} (\sigma,x) := \mu (\sigma^{-1},x)$, ($\sigma \in G$, $x \in X$), then we can define $ \bar{\mu}^\prime := \tilde{\mu}^\ast $ with the desired properties.
\end{definition}

\begin{definition}
  Let $V$ be a finite dimensional vector-space $\mu \colon G \times V \longrightarrow V$ a finite rational representation.
  We then define an action $\hat{\mu} \colon G \times V^\ast \longrightarrow V^\ast$, $ (\sigma,\varphi) \mapsto \sigma.\varphi := (v \mapsto \varphi(\mu(\sigma^{-1},v)) = \varphi(\sigma^{-1}.v))$, which is a finite rational represenation of $G$.
\end{definition}

\begin{proposition}\label{rara}
  let $X$ be an affine variety and $\mu \colon G \times X \longrightarrow X$ a regular action.
  For $f \in K[X]$, if we have $\bar{\mu}^\prime (f) = \Sigma_{i = 1}^r p_i \otimes g_i$ for some $\{g_i\}_{i\in [r]}$, then for $\sigma \in G$ we get $\bar{\mu}^\prime (\sigma.f) = \Sigma_{i = 1}^r \sigma^{-1} \dot{\phantom{.}} p_i \otimes g_i$.
\end{proposition}

\begin{proof}
  Let $f \in K[X]$ and let $ \bar{\mu}^\prime (f) = \Sigma_{i=1}^r p_i \otimes g_i$ for some $\{g_i\}_{i \in [r]}$.
  Let $\sigma \in G$.
  Then for all $\tau \in G$ and for all $x \in X$ we have $ \bar{\mu}^\prime (\sigma.f) (\tau,x) = ((\epsilon_\tau \otimes \operatorname{id}) \circ \bar{\mu}^\prime) (\sigma.f) (x) = (\tau.(\sigma.f)) (x) = \Sigma_{i=1}^r p_i(\tau\sigma) g_i(x) = \Sigma_{i=1}^r \sigma^{-1}\dot{\phantom{.}}(\tau) g_i(x) = (\Sigma_{i=1}^r \sigma^{-1} \dot{\phantom{.}}p_i \otimes g_i) (\tau,x)$
\end{proof}

\begin{definition}[locally finite]
  For a vector-space $V$, we call an action $\mu \colon G \times V \longrightarrow V$ \textbf{locally finite}, if and only if for every $v \in V$ there exists a $G$-stable finite-dimensional vector space $U \subseteq V$ such that $v \in U$.
\end{definition}

\begin{definition}
  Let $V$ be a vector-space and $\mu \colon G \times V \longrightarrow V$ an action.
  For $v \in V$ we define $V_v := \operatorname{span} G.v$.
\end{definition}

\begin{remark}
  $V_v$ is always a $G$-stable subspace of $V$.
  For any $G$-stable subspace $W \subseteq V$ we have $V_v \subseteq W$.
  Therefore, an action $\mu \colon G \times V \longrightarrow V$ is locally finite if and only if $V_v$ is finite-dimensional.
\end{remark}

\begin{proposition}\label{locfin}
  Let $V$ be a vector-space.
  \begin{enumerate}[(a)]
  \item If $\mu \colon G \times V \longrightarrow V$ is a rational
    representation, then the action is locally finite, and every
    finite-dimensional $G$-stable subspace $W$,
    $\left. \mu \right|_{G\times W}$ is a finite rational
    representation.
  \item If $V$ is a finite-dimensional vector-space and $\mu \colon G \times V \longrightarrow V$ is a finite rational representation, then $\mu$ is also a rational representation.
  \end{enumerate}
\end{proposition}

\begin{proof}
  See \cite[A.1.8]{DK15} and \cite[2.2.5(b)$\implies$(c), 2.2.6]{DK15}

  \underline{(a)}\\
  Assume that $\mu$ is a rational representation.
  Let $v \in V$.
  We can write $\mu^\prime (v) = \Sigma_{i=1}^l f_i \otimes v_i$. %with $f_i$ linearly independant over $K$.
  We then easily see that $V_v \subseteq \operatorname{span}\{v_i\}_{i=1}^l$, showing that the action is locally finite.
  Now let $W$ be a finite-dimensional $G$-stable subspace of $V$ with the basis $\{w_i\}_{i=1}^r$.
  % For $w \in W$ there exist $\{p_k\}_k \in [s]$ such that $ \mu^\prime (w) = \Sigma_{i=1}^r p_i \otimes w_i + \Sigma_{m = r+1}^s p_m v_m$ for $v_m in 
  There exist $\{p_{i,j}\}_{i,j \in [r]} \subseteq K[G]$ with $ \mu^\prime (w_j) = \Sigma_{i=1}^r p_{i,j} \otimes w_i$.
  Now let $w = \Sigma_{j=1}^r \lambda_j w_j \in W$.
  Then for all $\sigma \in G$ we have
  \begin{equation}
    \begin{aligned}
      &D_{\left. \mu \right|_{G \times W}} (w)
      &=&\mu (\sigma,w)\\
      &&=& \left(\left(\epsilon_\sigma \otimes \operatorname{id} \right) \circ \mu^\prime \right) \left(w \right) \\
      &&=& \sum_{j=1}^r \lambda_j  \sum_{i=1}^r p_{i,j}\left(\sigma\right) \cdot w_i \\
    \end{aligned}
  \end{equation}
  from which we immediately notice that $D_{\left. \mu \right|_{G \times W}} (\sigma) \in \operatorname{GL}(W)$.
  Therefore $\left. \mu \right|_{G\times W}$ is a finite rational representation.\\
  \underline{(b)}\\
  Let $V$ be a finite-dimensional vector-space and $\mu \colon G \times V \longrightarrow V$ a finite rational representation.
  This means that for all $\sigma \in G$ we have $D_\mu (\sigma) \in \operatorname{GL}(V)$.
  Let us now choose a basis $\{v_i\}_{i \in [r]}$ of $V$.
  For all $\sigma \in G$ there then exist unique $\{ \left( D_\mu \right)_{i,j}\}_{i,j \in [r]} \subseteq K$ such that for all $i \in [r]$ we have $\mu (\sigma,v_i) = \Sigma_{k=1}^r \left(D_\mu\right)_{i,k} v_k$.
  Since the action is regular, we must have $p_{i,j} := \left( \mu \mapsto \left(D_\mu\right)_{i,k}\right) \in K[G]$.
  We now define $\mu^\prime \colon V \longrightarrow K[G] \otimes V$ as the linear extension of $v_i \mapsto \Sigma_{k=1}^r p_{i,k} \otimes v_k$ where for $i \in [r]$.
  It should be clear that $\mu^\prime$ satisfies $ \mu \left( \sigma , v \right) = \left( \left( \epsilon_\sigma \otimes \operatorname{id} \right) \circ \mu^\prime \right) \left(v\right) $ for all $\sigma \in G$.
  This shows that $\mu$ is a rational representation.
  % By assumption there exists a finite-dimensional $G$-stable vector space $W \subseteq V$ such that $v \in W$.
  % Choose a basis $\{w_1, \ldots, w_r\}$ of $W$, where $w_1 = v$ (this is done to make some terms easier, but choosing any basis also works).
  % Since $\left. \mu \right|_{G\times W}$ is a finite rational representation, we have $D_\mu (\sigma) \in \operatorname{GL}(W)$ for all $\sigma \in G$, and since our action is also regular, there exist $\{p_i\}_{i \in [r]} \subseteq K[G]$ such that for all $\sigma \in G$ we have $\mu(\sigma, v) = D_\mu(\sigma)(v) = \Sigma_{i=1}^r p_i(\sigma) \cdot w_i$.
  % Now define $\mu^\prime (v) := \Sigma_{i=1}^r p_i \otimes w_i$.
  % We shall now check that this is well-defined:
  \end{proof}
% \begin{definition}[Rational Representation]
%   Let $G$ be a linear algebraic group.
%   A representation $V$ of $G$ is called a \textbf{rational representation}, iff its corresponding action $ G \times V \longrightarrow V $ is a regular action.
% \end{definition}

% \textbf{Claim:} If $V$ is finite dimensional, the notions of the definitions coincide.\\
% \textit{Proof:} First, let $V$ be a rational representation of $G$ with basis $\{ v_1 , \ldots , v_N \}$ be a basis of $V$.
% By our assumption, we have a rational representation, therefore there exist $p_{i,j} \in K \left\lbrack G \right\rbrack$ such that $\mu\left( \sigma, v_j \right) = \Sigma_{i=1}^{N} p_{i,j}\left(\sigma\right) \cdot v_i$.
% Define $\mu^\ast \left( v_j \right) := \Sigma_{i=1}^{N} p_{i,j} \otimes v_i$.
% Now we easily see:
% \begin{equation}
%   \begin{aligned}
%     \mu\left(\sigma,v\right)
%     &= \mu \left(\sigma, \Sigma_{j=1}^N \lambda_j v_j \right) \\
%     &= \sum_{j=1}^N \lambda_j  \sum_{i=1}^N p_{i,j}\left(\sigma\right) \cdot v_i \\
%     &= \sum_{j=1}^N \lambda_j \left(\left(\epsilon_\sigma \otimes \operatorname{id} \right) \circ \mu^\ast \right) \left(v_j \right)
%     &= \left(\left(\epsilon_\sigma \otimes \operatorname{id} \right) \circ \mu^\ast \right) \left(v \right)
%   \end{aligned}
% \end{equation}
% \textit{which was to show.}

\begin{remark}
  This shows that for a finite-dimensional vector space $V$, an action is a rational representation if and only if it is a finite rational representation.
  In other words, we have shown that finite rational representations are exactly the rational representations that are finite dimensional, which justifies the choice of the names of our definitions.
\end{remark}

\begin{remark}
  A finite rational representation $\mu \colon G \times V \longmapsto V$ is of the following form:\\
  Consider $D_{\mu} \colon G \longmapsto \operatorname{GL}(V)$.
  If then $ a_{i,j} : G \longrightarrow K $ is the function of the $\left( i,j \right) $-entry of $D_{\mu}$, then $ a_{i,j} \in K\lbrack G\rbrack $.\\
  In fact, it is equivalent to define a representation $\mu \colon G \times V \longrightarrow V$ ($V$ finite dimensional) as rational, iff $D_{\mu} \colon G \longrightarrow \operatorname{GL}(V)$ is a map of affine varieties.
\end{remark}

\begin{definition}
  If $\mu \colon G \times V \longrightarrow V$ is a finite rational representation, we define an action on $\hat{\mu} \colon G \times V \longrightarrow V$ by $(\sigma,\varphi) \mapsto \sigma.\varphi := v \mapsto \varphi(\sigma^{-1}.v)$.
  $\hat{mu}$ is also a finite rational representation of $G$.
\end{definition}

\begin{definition}[Invariants]
  Let $ G $ act on $ X $ regularly.
  \begin{equation}
    X^G := \left\{\, x \in X \mid \forall g \in G : g . x = x \,\right\}
  \end{equation}
  This defines a linear subspace.
  The given action induces an action $ \bar{\mu} \colon G \times K\lbrack X\rbrack \longrightarrow K\lbrack X\rbrack $ as per definition \ref{funrep}.
  % \begin{equation}
  %   \left( g , f \right) \longmapsto g \cdot f :=
  %   \left( x \mapsto f \left( \sigma^{-1} . x \right) \right)
  % \end{equation}
  The \textbf{invariant ring} of the representation is defined as
  \begin{equation}
    K\lbrack X\rbrack^G := \left\{ \, f \in K\lbrack X \rbrack \mid \forall g \in G : g . f = f \, \right\}
  \end{equation}
  As the name implies, $ K\lbrack X\rbrack^G $ defines a subalgebra of $ K\lbrack X\rbrack^G $.
\end{definition}

The general theme of my work revolves around the question of whether the invariant ring $ K\lbrack X\rbrack^G $ is finitely generated.

\textit{Hilbert's finiteness theorem} states that if the group $G$ is linearly reductive, $ K\lbrack V\rbrack^G $ is finitely generated.
The strict definition of ``linearly reductive'' is quite tricky, but we will give an alternate equivalent definition shortly.

%%% Local Variables:
%%% mode: latex
%%% TeX-master: "roughdraft"
%%% End: