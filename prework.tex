In the following, $K$ is a field of characteristic $0$ and $G$ a linear algebraic group, that is a group whose set is an affine variety, and whose multiplcation and inversion are morphisms of affine varieties.

\begin{definition}
We denote by $m$ the group multiplication of the group $G$.
We want to view the pullback of $m$ as a map $m^\ast : K[G] \longrightarrow K[G] \otimes K[G]$, which makes sense, because $m$ and $\otimes$ are associative.
The strict pullback, which I will call $ \hat{m} $, should be a map of the type $ K[G] \longrightarrow K[ G \times G] $, where $ f \mapsto f \circ m $.
If we want to give the variables names, we can equivalently say it is a map $ \left. K[Z] \right|_G \longrightarrow \left. K[X,Y] \right|_{G \times G} $, where $ Z = \lbrace Z_1 , \ldots , Z_k \rbrace $, $X$ and $Y$ analogously (here, $ m $ canonically takes its left input via $ X $ and its right input via $ Y $).
Define
\begin{equation}
  \begin{aligned}
   t \colon \left. K \left\lbrack X , Y \right\rbrack \right|_{G \times G} \longrightarrow \left. K \left\lbrack Z \right\rbrack \right|_G \otimes \left. K\left\lbrack Z \right\rbrack \right|_G \\
    \sum_i \lambda_i \prod_j X_{j}^{d_{i,j}} \prod_j Y_{j}^{e_{i,j}} \mapsto \sum_i \lambda_i \prod_j Z_{j}^{d_{i,j}} \otimes \prod_j Z_{j}^{e_{i,j}} 
  \end{aligned}
\end{equation}
This is independant of the choice of the representatives and therefore well-defined.
It is even an isomorphism.
Now, finally, define $m^\ast := t \circ \hat{m} : K[G] \longrightarrow K[G] \otimes K[G]$.
\end{definition}

\begin{remark}
One might ask why we want to look at these objects $ m^\ast \left( f \right) $ instead of $ \hat{m} \left( f \right) $.
Really, these objects are hardly different (the spaces are isomorphic), but it helps to formalize performing operations only on the ``left part'' or the ``right part'', as we will soon see.
This is an approach that \cite{DK15} follows, but other literature such as \cite{Stu08} (and probably also Cayley) rather consider $ \hat{m} \left( f \right) $ written as $ \hat{m} \left( f \right) = f (XY) $.
To give a very simple example:
If $f \in \left. K[Z] \right|_G$, we will write $\operatorname{id} \otimes \frac{\partial}{\partial Z_i} (m^\ast f)$ as in \cite{DK15}, whereas \cite{Stu08} would write $\frac{\partial}{\partial Y_i} f(XY)$.
\end{remark}


\begin{definition}[Rational Representation]\label{rr}
  Let $V$ be a vector space (not necessarily finite dimensional), and $ \mu : G \times V \longrightarrow V $ an action.
  We call $ \mu $ a \textbf{rational representation} iff there exists a linear map $ \mu^\prime \colon V \longrightarrow K[G] \otimes V $ such that $ \mu \left( \sigma , v \right) = \left( \left( \epsilon_\sigma \otimes \operatorname{id} \right) \circ \mu^\prime \right) \left(v\right) $.
\end{definition}

\begin{remark}
  If $\tilde{\mu}$ is the action defined by $ \tilde{\mu} (\sigma,x) = \mu (\sigma^{-1},x)$, then $ \mu^\prime = \tilde{\mu}^\ast $, where $\mu^\ast$ denotes the algebraic cohomomorphism.
\end{remark}

\begin{definition}[Regular Action, Regular Representation]
  Let $G$ be a linear algebraic group, $X$ an affine variety.
  We call an action $G \times X \longrightarrow X$ a \textbf{regular action}, iff it is a morphism of affine varieties.
  We say \textbf{$ G $ acts regularly on $ X $}.

  For a finite-dimensional vector space $V$, if $\mu \colon G \times V \longrightarrow V$ is a representation in the classical sense, that is for all $g \in G$ we have $D_\mu (g) := (v \mapsto g.v) \in \operatorname{GL}(V)$, we call $\mu$ a \textbf{regular representation} iff it is regular.
  
  % For a finite dimensional vector space $V$ we call $D \colon G \longrightarrow \operatorname{GL}(V)$ a \textbf{regular representation} iff $\mu_D := ((g,v) \mapsto D(g)(v))$ is a regular action.
  % If an action $\mu \colon G \times V \longrightarrow V$ is given such that $D_\mu (g) := (v \mapsto g.v) \in \operatorname{GL}(V)$, we also call $\mu$ a regular representation (such $\mu$ and $D$ are in bijection).
\end{definition}

\begin{proposition}
  An action $\mu \colon G \times V \longrightarrow V$ is a rational representation if and only if the action is locally finite (id est for every $v \in V$, $\operatorname{span}(G.v)$ is finite-dimensional), and every finite-dimensional $G$-stable subspace $W$, $\left. \mu \right|_{G\times W}$ is a regular representation.
\end{proposition}

\begin{proof}
  See \cite[A.1.8]{DK15} and \cite[2.2.5(b)$\implies$(c),2.2.6]{DK15}

  Assume that $\mu$ is a rational representation.
  Let $v \in V$.
  We can write $\mu^\ast (v) = \Sigma_{i=1}^l f_i \otimes v_i$. %with $f_i$ linearly independant over $K$.
  We then easily see that $K(G.v) \subseteq \operatorname{span}\{v_i\}_{i=1}^l$, showing that the action is locally finite.
  Now let $W$ be a finite-dimensional subrepresentation with the basis $\{w_i\}_{i=1}^r$.
  By assummption, we have $p_{i,j} \in K[G]$ with $ \mu^\ast (w_j) = \Sigma_{i=1}^r p_{i,j} \otimes w_i$.
  Now let $w = \Sigma_{j=1}^r \lambda_j w_j \in W$.
  Then for all $\sigma \in G$ we have
  \begin{equation}
    \begin{aligned}
      &\mu (\sigma,w)
      &=& \left(\left(\epsilon_\sigma \otimes \operatorname{id} \right) \circ \mu^\ast \right) \left(w \right) \\
      &&=& \sum_{j=1}^N \lambda_j  \sum_{i=1}^N p_{i,j}\left(\sigma\right) \cdot w_i \\
    \end{aligned}
  \end{equation}
  from which we immediately notice that $D_{\left. \mu \right|_{G \times W}} (\sigma) \in \operatorname{GL}(W)$.
  Therefore $\left. \mu \right|_{G\times W}$ is a regular representation.

  Now let $\mu$ be an action such that for every finite-dimensional $G$-stable subspace $W$, $\left. \mu \right|_{G\times W}$ is a regular representation.
  Let $v \in V$.
  There exists a 
\end{proof}

\begin{definition}[Rational Representation]
  Let $G$ be a linear algebraic group.
  A representation $V$ of $G$ is called a \textbf{rational representation}, iff its corresponding action $ G \times V \longrightarrow V $ is a regular action.
\end{definition}

\textbf{Claim:} If $V$ is finite dimensional, the notions of the definitions coincide.\\
\textit{Proof:} First, let $V$ be a rational representation of $G$ with basis $\{ v_1 , \ldots , v_N \}$ be a basis of $V$.
By our assumption, we have a rational representation, therefore there exist $p_{i,j} \in K \left\lbrack G \right\rbrack$ such that $\mu\left( \sigma, v_j \right) = \Sigma_{i=1}^{N} p_{i,j}\left(\sigma\right) \cdot v_i$.
Define $\mu^\ast \left( v_j \right) := \Sigma_{i=1}^{N} p_{i,j} \otimes v_i$.
Now we easily see:
\begin{equation}
  \begin{aligned}
    \mu\left(\sigma,v\right)
    &= \mu \left(\sigma, \Sigma_{j=1}^N \lambda_j v_j \right) \\
    &= \sum_{j=1}^N \lambda_j  \sum_{i=1}^N p_{i,j}\left(\sigma\right) \cdot v_i \\
    &= \sum_{j=1}^N \lambda_j \left(\left(\epsilon_\sigma \otimes \operatorname{id} \right) \circ \mu^\ast \right) \left(v_j \right)
    &= \left(\left(\epsilon_\sigma \otimes \operatorname{id} \right) \circ \mu^\ast \right) \left(v \right)
  \end{aligned}
\end{equation}
\textit{which was to show.}

\begin{remark}
  A rational representation $ G \longrightarrow \operatorname{GL}\left(V\right) $ is of the following form:\\
  If $ a_{i,j} : G \longrightarrow K $ is the function of the $\left( i,j \right) $-entry, then $ a_{i,j} \in K\lbrack G\rbrack $.\\
  In fact, it is equivalent to define a representation as rational, iff its map $ G \longrightarrow \operatorname{GL} \left( V \right) $ is a map of affine varieties.
\end{remark}

\begin{definition}[Invariants]
  Let $ G $ act on $ X $ regularly.
  \begin{equation}
    X^G := \left\{\, x \in X \mid \forall g \in G : g . x = x \,\right\}
  \end{equation}
  It defines a linear subspace.
  This given action induces an action $ G \times K\lbrack X\rbrack \longrightarrow K\lbrack X\rbrack $, where $K\lbrack X\rbrack$ is the coordinate ring, as follows:
  \begin{equation}
    \left( g , f \right) \longmapsto g \cdot f :=
    \left( x \mapsto f \left( \sigma^{-1} . x \right) \right)
  \end{equation}
  The \textbf{invariant ring} of the representation is defined as
  \begin{equation}
    K\lbrack X\rbrack^G := \left\{ \, f \in K\lbrack X \rbrack \mid \forall g \in G : g \cdot f = f \, \right\}
  \end{equation}
  As the name implies, $ K\lbrack X\rbrack^G $ defines a subalgebra of $ K\lbrack X\rbrack^G $.
\end{definition}

The general theme of my work revolves around the question of whether the invariant ring $ K\lbrack X\rbrack^G $ is finitely generated.

\textit{Hilbert's finiteness theorem} states that if the group $G$ is linearly reductive, $ K\lbrack V\rbrack^G $ is finitely generated.
The strict definition of ``linearly reductive'' is quite tricky, but we will give an alternate equivalent definition shortly.

%%% Local Variables:
%%% mode: latex
%%% TeX-master: "roughdraft"
%%% End: