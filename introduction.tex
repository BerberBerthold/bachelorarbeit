A very important concept in mathematics is the idea of an \textit{invariant}:
An object which does not change under a certain action.
In 1872, Felix Klein came up with a then new method of describing geometries with group theory, called the Klein Erlangen program (see \cite{Kle93}).
Here, the central idea of a geometry is characterized by its associated symmetry group, the group of transformations which leaves certain objects or properties unchanged, for example angles.
The study of these transformations, for instance, is called conformal geometry.

Let us discuss the following important example in geometry:
Consider all transformations which map lines to lines, meaning such transformations under which the property of being a line is invariant.
The fundamental theorem of projective geometry gives us that these maps are exactly the projective transformations (see \cite[Ex V.44,~Ex I.51]{Aud03}).

Conversely, we can consider projective transformations as our given group of transformations.
Invariant theory asks: What invariants exist?
We can loosely notice a kind of duality between geometries viewed as in the Klein Erlangen program and invariant theory.
This discipline of mathematics usually only looks at invariants described with so called regular terms, or more concretely formulated:  In invariant theory, we try to find invariant polynomial-like functions.

Staying in our example of considering projective transformations as our given group, a well known example for an invariant is the cross ratio.
It is a rational function which takes as its input four collinear points.
Is this the only invariant?
How can we find other invariants?
How big is the ring of all invariants?

\textit{Hilbert's finiteness theorem} states that for regular actions under certain groups, such that are \textit{linearly reductive}, the invariant ring is finitely generated.
If we can find these finite generators, we have a grasp of what all invariants look like.
Hilbert's first proof for this theorem, which he published in 1890 (see \cite{Hil90}), was non-constructive.
The central idea of this proof is the existence of a Reynolds operator, which projects the coordinate ring to the invariant ring.

One of the most important and most common groups is the general linear group $\operatorname{GL}_n$.
This group is linearly reductive and there are multiple ways to see this.
Motivated by averaging for finite groups, for compact groups it is possible to replace the sum by an integral with the Haar-measure, from which we can show that $\operatorname{GL}_n$ is linearly reductive (see \cite[p.~285-288]{Kra85}).
One can also show linear reductivity by the Schur-Weyl-duality:  The symmetric group is finite, from which we can therefore see that in any rational $\operatorname{GL}_n$-representation we can again construct module complements (see \cite[p.~243]{Pro07}).

Here, we will show that $\operatorname{GL}_n$ is linearly reductive in an even different way.
For one, we want to show that a Reynolds Operator exists, which already means that $\operatorname{GL}_n$ is linearly reductive.
But we want even more than just its existence.
What does it do for our motivation to get a grasp of what all (or even just some) invariants look like, if we merely prove the existence of a finite generating set for the invariants?
Since this operator projects polynomials to invariant polynomials, if we can find an explicit formula for computing the Reynolds operator applied to a polynomial, we can obtain concrete invariants.

This is possible with \textit{Cayley's $\Omega$-process}, which Arthur Cayley came up with as early as 1846 (see \cite{Cay46}).

Similar to the first proof of Hilbert's finiteness theorem (by Hilbert himself, see \cite{Hil90}), we can show that there is a finite set of polynomials whose images under the Reynolds operator will generate the invariant ring.
Although this is not what we will be discussing in detail, there is in fact an algorithm to compute these certain polynomials.
With the help of Cayley's $\Omega$-process, we then get a complete algorithm that gives us the generators of the invariant ring.
(See \cite[4.1.9]{DK15})

\subsection{Outline}
In this work, we cover many concepts from invariant theory before arriving at Cayley's $\Omega$-process.

First, the framework of invariant theory is set up.
For a linear algebraic group $G$, we give affine varieties the structure of a $G$-variety and we make sense of rational representations, or rational $G$-modules.
We give the coordinate ring $K[X]$ of an affine $G$-variety $X$ the structure of a rational $G$-module, which allows us to describe the invariant ring $K[X]^G$.
This is the main object of interest in this work.

Motivated by Hilbert's finiteness theorem, which states that $K[X]^G$ is finitely generated for a linearly reductive group $G$ and an affine $G$-variety $X$, we discuss different characterizations of the notion of a linearly reductive group.
Great \linebreak attention is brought to the Reynolds Operator, which for a given $G$-variety $X$ projects $K[X]$ to the invariant ring $K[X]^G$.
For linearly reductive groups, this operator always exists, and this helps us to prove Hilbert's finiteness theorem.

We then discuss a very useful condition for a linear algebraic group $G$ being linearly reductive:  If the Reynolds Operator $R_G$ of $G$ exists, we can express any Reynolds operator $R\colon K[X] \twoheadrightarrow K[X]^G$ in terms of $R_G$, which makes $G$ linearly reductive.

For the general linear group $\operatorname{GL}_n$, we will concretely express the Reynolds Operator $R_{\operatorname{GL}_n}$ with Cayley's $\Omega$-process, to which we devote an entire section.
We will then look at an example and calculate the Reynolds Operator applied to concrete polynomials to obtain invariants.

Lastly, without going into great detail, we discuss a complete algorithm for computing the generators of the invariant ring $K[V]^{\operatorname{GL}_n}$ for a given $\operatorname{GL}_n$-representation $V$.
\vspace{0.5cm}

The main source of this work is \textit{Computational invariant theory} by Harm Derksen and Gregor Kemper \cite{DK15}.
Section 3 and section 4 closely follow chapters 2.2.1 and 4.5.3 in this book, respectively, while most definitions and notions are also borrowed from there.

%%% Local Variables:
%%% mode: latex
%%% TeX-master: "roughdraft"
%%% End:
