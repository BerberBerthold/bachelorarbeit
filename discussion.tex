\subsection{A Complete Algorithm for Retrieving Generators of the Invariant Ring}
Our motivation for having a construction of the Reynolds operator was to not only see that $\operatorname{GL}_n$ is linearly reductive, but also to yield some invariants.
It would also be very helpful if we could somehow produce a generating set for the invariant ring.  \\
In example \ref{theex}, we saw that applying the Reynolds operator to any polynomial does not always result in retrieving a nonzero invariant.
It suggests that we somehow need to find the ``correct'' polynomials to apply the Reynolds operator to.
The following proposition (see \cite[prop. 4.1.1]{DK15}) gives us exactly that.
\begin{proposition}
  Let $V$ be a rational $G$-representation where $G$ is linearly reductive, and let $I_{>0}$ denote the ideal generated by all non-constant invariants.
  If $I_{>0} = \left(\{f_i\}_{i \in[r]}\right)$ for some homogeneous polynomials $\{f_i\}_{i\in[r]} \subseteq K[V]$, we have $I_{>0} = \left(\{R(f_i)\}_{i\in[r]}\right)$ and $K[V]^G = K\left[\{R(f_i)\}_{i\in[r]}\right]$.
\end{proposition}
In the proof of Hilbert's finiteness theorem (\ref{hilbert}), we made use of the existence of a finite set of invariants generating $I_{>0}$, which was non-constructively given.
The previous proposition looks helpful since we have a construction for the Reynolds operator for $G=\operatorname{GL}_n$ via Cayley's $\Omega$-process, but the problem still remains that we need to have a finite set of homogeneous polynomials generating $I_{>0}$, whose existence is here also non-constructively given.  \\
It is in fact possible to compute them with Groebner bases, which is extensively described in \cite[Algorithm 4.1.9]{DK15}.
This gives us a complete algorithm that takes as its input all of the information necessary to describe our rational representation, which can all be given in terms of polynomials, and outputs a list of generators of the invariant ring.

\subsection{Cross Ratio}
In examples \ref{domcr} and \ref{crinv} we discussed the cross ratio.
Our setting was affine and in $K^2$, which makes our results different from the projective setting, where there are not very many other polynomials other than the cross ratio.  \\
Using the same conventions and definitions as in the aforementioned examples, we can define the projective cross ratio:
\begin{equation}
  \begin{aligned}
    \operatorname{cr} \colon&&Y&\longrightarrow K \\
    &&([x_1],[x_2],[x_3],[x_4]) &\longmapsto \frac{\operatorname{det}(x_1,x_2)\operatorname{det}(x_3,x_4)}{\operatorname{det}(x_2,x_3)\operatorname{det}(x_4,x_1)}
  \end{aligned}
\end{equation}
where $Y \subseteq P(K^2)^4$ is the set of all pairwise distinct four-tuples of points in $P(K^2)$.
It should be clear that this is well-defined.
The action of $\operatorname{GL}_2$ on $X$ induces an action of $\operatorname{PGL}_n$ on $Y$.  \\
Let $f \colon Y \longrightarrow K$ be an invariant regular function.
If $X_1,X_2,X_3,Y_1,Y_2,Y_3 \in P(K^2)$ with $X_1$, $X_2$ and $X_3$ pairwise distinct and $Y_1$, $Y_2$ and $Y_3$ pairwise distinct, then an important theorem in projective geometry is that there exists a (unique) projective transformation $ \rho \in \operatorname{PGL}(K^2)$ such that $\rho (X_1) = Y_1$, $\rho(X_2) = Y_2$ and $\rho(X_3) = Y_3$ (see \cite[prop 5.6]{Aud03}).
  Let $A,B,C,D \in Y$, which implies that $B,C,D$ are pairwise distinct.
  For $x \in K $ we define $x_P := \left[\binom{x}{1}\right]$ and $\infty_P := \left[\binom{1}{0}\right]$.
  There then exists a $\rho \in \operatorname{PGL}(K^2)$ such that $\rho (B) = 0_P$, $\rho (C) = 1_P$ and $\rho (D) = \infty_P$.
  Since $A$ is distinct from $D$ we know $\rho (A) \neq \infty_P$, and therefore there exists some $a \in K$ such that $\rho(A) = [\binom{a}{1}]$.
  We then compute
  \begin{equation}
    \begin{aligned}
      &\rho (A)
      &&= \left[\binom{a}{1}\right]\\
      &&&= \operatorname{cr}\left(\left[\binom{a}{1}\right],\left[\binom{0}{1}\right],\left[\binom{1}{1}\right],\left[\binom{1}{0}\right]\right)_P\\
      &&&= \operatorname{cr}(\rho(A),\rho(B),\rho(C),\rho(D))_P &&= \operatorname{cr}(A,B,C,D)_P
    \end{aligned}
  \end{equation}
  We then have
  \begin{equation}
    \begin{aligned}
      &f(A,B,C,D)
      &&= \rho^{-1}.f (A,B,C,D)\\
      &&&= f(\rho(A),\rho(B),\rho(C),\rho(D))\\
      &&&= f(\operatorname{cr}(A,B,C,D)_P,0_P,1_P,\infty_P)
    \end{aligned}
  \end{equation}
  This shows that in the projective setting, there don't exist many more invariants than the cross ratio.  \\
  In our affine setting, it suggests that we can transfer the idea, which would mean that $K[\operatorname{cr},p(\binom{\operatorname{cr}}{1},\binom{0}{1},\binom{1}{1},\binom{1}{0})^{-1}] = K[\operatorname{cr},(\operatorname{cr}(\operatorname{cr}-1))^{-1}] = K[\operatorname{cr},\operatorname{cr}(X_1,X_3,X_4,X_2)]$ are all invariants.
  This is not true though, since for instance $\frac{\operatorname{det}(X_1,X_2)}{\operatorname{det}(X_3,X_4)} \in K[X]^{\operatorname{GL}_2}$ is an invariant not included in $K[\operatorname{cr},\operatorname{cr}(X_1,X_3,X_4,X_2)]$.
  
%%% Local Variables:
%%% mode: latex
%%% TeX-master: "roughdraft"
%%% End: