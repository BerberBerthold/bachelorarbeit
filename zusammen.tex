Ein wichtiges Theorem in der Invariantentheorie ist \textit{Hilberts Endlichkeitssatz}:  Ist eine Gruppe $G$ linear reduktiv, so gilt f\"ur jede affine $G$-Variet\"at $X$, dass der Invariantenring $K[X]^G$ endlich erzeugt ist, das hei{\ss}t es gibt $\{f_i\}_{i\in[r]} \subseteq K[X]$ sodass $K[X]^G = K[\{f_i\}_{i\in[r]}]$.
Im Beweis ist die Zentrale Idee die Existenz eines \textit{Reynolds Operators} $R \colon K[X] \twoheadrightarrow K[X]^G$, eine $G$-invariante lineare Projektion von $K[X]$ auf $K[X]^G$.
Dieser erm\"oeglicht uns, Invarianten zu finden.  \\
Eine der wichtigsten Gruppen in der Mathematik ist die allgemeine lineare Gruppe $\operatorname{GL}_n$.
Diese ist in der Tat linear reduktiv, was man auf verschiedene Weisen zeigen kann.
Eine M\"oglichkeit die lineare Reduktivit\"at nachzuweisen, ist zu zeigen, dass ein Reynolds Operator der Gruppe existiert.
Die Existenz dieses Reynolds Operators wird in dieser Arbeit konstruktiv mit Hilfe von \textit{Cayleys $\Omega$-Prozess} nachgewiesen, was nicht nur zur Folge hat, dass $\operatorname{GL}_n$ linear reduktiv ist, soncern auch, dass wir eine explizit Formel des Reynolds Operators haben, womit wir konkret Invarianten ausrechnen k\"onnen.
%%% Local Variables:
%%% mode: latex
%%% TeX-master: "roughdraft"
%%% End:
