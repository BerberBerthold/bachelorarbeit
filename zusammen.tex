Ein wichtiges Theorem in der Invariantentheorie ist \textit{Hilberts Endlichkeitssatz}:  Ist eine Gruppe $G$ linear reduktiv, so gilt f\"ur jede affine $G$-Variet\"at $X$, dass der Invariantenring $K[X]^G$ endlich erzeugt ist, das hei{\ss}t es gibt $\{f_i\}_{i\in[r]} \subseteq K[X]$ sodass $K[X]^G = K[\{f_i\}_{i\in[r]}]$ (siehe \cite{Hil90}).
Im Beweis ist die Zentrale Idee die Existenz eines \textit{Reynolds Operators} $R \colon K[X] \twoheadrightarrow K[X]^G$, eine $G$-invariante lineare Projektion von $K[X]$ auf $K[X]^G$.

Eine der wichtigsten Gruppen in der Mathematik ist die allgemeine lineare Gruppe $\operatorname{GL}_n$.
Diese ist in der Tat linear reduktiv, was man auf verschiedene Weisen zeigen kann.
Eine M\"oglichkeit, die lineare Reduktivit\"at nachzuweisen, ist zu zeigen, dass der Reynolds Operator der Gruppe existiert.
Die Existenz dieses Operators wird in dieser Arbeit konstruktiv mit Hilfe von \textit{Cayleys $\Omega$-Prozess} nachgewiesen, welcher von Arthur Cayley im Jahre 1846 schon ver\"offentlicht wurde (siehe \cite{Cay46}).
Dies hat nicht nur zur Folge, dass $\operatorname{GL}_n$ linear reduktiv ist, sondern auch, dass wir eine explizite Formel des Reynolds Operators haben, womit wir konkret Invarianten ausrechnen k\"onnen.

Es werden in dieser Arbeit mehrere Konzepte vorgestellt, bevor Cayley's $\Omega$-Prozess behandelt wird.

Als erstes werden die Grundz\"uge der Invariantentheorie besprochen.
Wir geben affinen Variet\"aten die Struktur einer $G$-Variet\"at, und behandeln \linebreak desweiteren rationale $G$-Darstellungen und $G$-Module.
F\"ur eine linear reduktive Gruppe $G$ geben wir dem Koordinatenring $K[X]$ die Struktur eines rationalen $G$-Moduls, was uns erlaubt den Invariantenring $K[X]^G$ zu beschreiben.
Dies ist ein zentrales Objekt in dieser Arbeit.

Motiviert von Hilberts Endlichkeitssatz, welcher besagt, dass $K[X]^G$ endlich erzeugt ist f\"ur eine linear reduktive Gruppe $G$ und eine affine $G$-Variet\"at $X$, werden wir verschiedene Charakterisierungen des Begriffes einer linear reduktiven Gruppe besprechen.
Es wird besonders auf den Reynolds Operator aufmerksam gemacht, welcher $K[X]$ auf den Invariantenring $K[X]^G$ projeziert f\"ur eine affine $G$-Variet\"at $X$.
F\"ur linear reduktive Gruppen existiert dieser Operator immer, und dies ist zentral f\"ur den Beweis von Hilbert's Endlichkeitssatz.

Daraufhin besprechen wir ein sehr n\"utzliches Resultat:  Wenn der Reynolds Operator $R_G$ der Gruppe $G$ existiert, k\"onnen wir den Reynolds Operator \linebreak$R \colon K[X] \twoheadrightarrow K[X]^G$ f\"ur jede affine $G$-Variet\"at mithilfe von $R_G$ darstellen, was zeigt, dass $G$ linear reduktiv ist.

Ein ganzer Abschnitt befasst sich mir der expliziten Darstellung von dem Reynolds Operator $R_{\operatorname{GL}_n}$ der allgemeinen linearen Gruppe $\operatorname{GL}_n$, was wir mithilfe von Cayley's $\Omega$-Prozess realisieren.
Wir werden hiermit in einem Beispiel explizit die Anwendung des Reynolds Operators auf konkrete Polynome ausrechnen, womit wir Invarianten erhalten.

Zuletzt wird oberfl\"achlich ein kompletter Algorithmus zur Berechnung der Erzeuger des Invariantenrings $K[V]^{\operatorname{GL}_n}$ einer gegebenen $\operatorname{GL}_n$-Darstellung $V$ thematisiert.
\vspace{0.3cm}

Die Hauptquelle dieser Arbeit ist das Buch \textit{Computational invariant theory} von Harm Derksen und Gregor Kemper \cite{DK15}.
Abschnitt 3 und Abschnitt 4 halten sich eng an die Kapitel 2.2.1 und 4.5.3 in diesem Buch.
Die meisten Definitionen sind ebenfalls hieraus entnommen.


%%% Local Variables:
%%% mode: latex
%%% TeX-master: "roughdraft"
%%% End:
