\documentclass{article}

\usepackage{amssymb}
\usepackage{amsmath}
\usepackage{amsthm}
\usepackage{cite}
\usepackage{mathtools}
\usepackage{color}

\definecolor{gray}{gray}{0.5}

\begin{document}
\begin{enumerate}
\item Preliminary Work
  \begin{itemize}
  \item Definition:  The pullback $m^\ast$ (Isomorphism between $K[G \times G]$ and $K[G] \otimes K[G]$)
  \item Definition:  Rational representation (with infinite-dimensional vector spaces)
  \item Definition:  Regular action
  \item An alternate definition for rational representation for finite vector spaces, and the proof for its equivalence
  \item Another equivalent definition for a rational representation which immediately follows from the last characterization
  \item Definition:  Invariants and the invariant ring
  \end{itemize}
\item Linearly reductive groups, the Reynolds operator and Hilbert's finiteness theorem
  \begin{itemize}
  \item Definition:  Reynolds operator
  \item Uniqueness of the Reynolds operator
  \item Definition:  linear reductiveness; definition via the Reynolds operator
  \item Remark about different ways to characterize linear reductiveness
  \item Definition:  A multiplication in $K[G]^\ast$
  \item Proposition:  The afore mentioned multiplication makes $K[G]^\ast$ into an associative algebra
  \item Definition:  The action of $K[G]^\ast$ on $V$ induced by an action of $G$ on $V$.
  \item A remark about how we can view the $K[G]^\ast$ action as an extension of the given one ($\sigma . v = \epsilon_\sigma \cdot v$ where $\epsilon_\sigma \in K[G]^\ast$ is the evaluation homomorphism).
  \item Remark:  $K[G]^\ast$ contains the Lie algebra as a subalgebra.
  \item Proposition:  A regular action of $G$ on an affine variety $X$ induces a rational representation $K[X]$ of $G$
  \item Proposition:  A Reynolds operator on $K[G]$ (action via left-multiplication) yields a Reynolds operator on $K[X]$ for any regular action of $G$ on $X$
  \item TODO:\\
    \textcolor{gray}{The usual lemmas for Hilbert's finiteness theorem}\\
    \textcolor{gray}{Hilbert's finiteness theorem}
  \end{itemize}
\item Cayley's $\Omega$-Process
  \begin{itemize}
  \item Definition:  Cayley's $\Omega$-Process
  \item A Lemma about a relationship between $m^\ast$ and $\Omega$
  \item Lemma and Definition:  The constants $c_{p,n} := \Omega^p (\operatorname{det}(Z)^p)$
  \item Theorem:  $\operatorname{GL}_n$ is linearly reductive via the Reynolds operator that can be defined by Cayley's $\Omega$-process
  \item Corollary:  $\operatorname{SL}_n$ is linearly reductive
  \end{itemize}
\item Examples (known or easy ones)
  \begin{itemize}
  \item The cross ratio
  \item $SAS^T$
  \end{itemize}
\end{enumerate}
\end{document}

%%% Local Variables:
%%% mode: latex
%%% TeX-master: t
%%% End:
