% Created 2018-03-01 Thu 01:04
\documentclass[11pt]{article}
\usepackage[utf8]{inputenc}
\usepackage[T1]{fontenc}
\usepackage{fixltx2e}
\usepackage{graphicx}
\usepackage{longtable}
\usepackage{float}
\usepackage{wrapfig}
\usepackage{rotating}
\usepackage[normalem]{ulem}
\usepackage{amsmath}
\usepackage{textcomp}
\usepackage{marvosym}
\usepackage{wasysym}
\usepackage{amssymb}
\usepackage{hyperref}
\tolerance=1000
\date{}
\title{scratch}
\hypersetup{
  pdfkeywords={},
  pdfsubject={},
  pdfcreator={Emacs 25.3.1 (Org mode 8.2.10)}}
\begin{document}

\maketitle
\section*{Introduction}
\label{sec-1}
\section*{Preliminary Work}
\label{sec-2}
\subsection*{Algebraic Geometry}
\label{sec-2-1}
\begin{itemize}
\item affine varieties
\item linear algebraic groups, example GL
\item algebraic cohomomorphism, example multiplication and regular actions
\end{itemize}
\subsection*{Invariant Theory}
\label{sec-2-2}
\begin{itemize}
\item regular action, example, cross ratio maybe
\item rational representation, example coordinate rings, conjugation
\item invariants, example cross ratio
\end{itemize}
\section*{Linearly Reductive Groups, The Reynolds Operator And Hilbert's Finiteness Theorem}
\label{sec-3}
\subsection*{The Reynolds Operator And Linearly Reductive Groups}
\label{sec-3-1}
\begin{itemize}
\item equivalences of "linearly reductive"
\end{itemize}
\subsection*{Hilbert's Finiteness Theorem}
\label{sec-3-2}
\begin{itemize}
\item theorem
\item embeddings, example cross ratio
\end{itemize}
\subsection*{The Reynolds Operator Of A Linear Algebraic Group}
\label{sec-3-3}
\begin{itemize}
\item K[G]* as an associative K-algebra
\item existence Reynolds operator of group implies linearly reductive
\end{itemize}
\section*{Cayley's Omega Process}
\label{sec-4}
\begin{itemize}
\item everything
\item examples conjugation and cross ratio
\end{itemize}
\section*{Further Discussion}
\label{sec-5}
\begin{itemize}
\item an algorithm for computing the generators of the invariant ring
\end{itemize}
% Emacs 25.3.1 (Org mode 8.2.10)
\end{document}