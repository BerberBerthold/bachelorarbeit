\documentclass{article}
\title{Cayley's $\Omega$-Process And The Reynolds Operator}
\author{Bert Lorke}

\usepackage{amssymb}
\usepackage{amsmath}
\usepackage{amsthm}
\usepackage{cite}
\usepackage{mathtools}
\usepackage{enumerate}
\usepackage{perpage}

\newtheoremstyle{prrt}
  {\topsep}
  {\topsep}
  {}
  {0pt}
  {\bfseries}
  {\\}
  { }
  {\thmname{#1}\thmnumber{ #2}\thmnote{: #3}}

\theoremstyle{prrt}
\newtheorem{theorem}{Theorem}[section]
\newtheorem{definition}[theorem]{Definition}
\newtheorem{lemma}[theorem]{Lemma}
\newtheorem{proposition}[theorem]{Proposition}
\newtheorem{corollary}{Corollary}[theorem]
\newtheorem{remark}{Remark}[theorem]
\newtheorem{cremark}{Remark}[corollary]
\newtheorem{example}[theorem]{Example}
\newtheorem{dexample}{Example}[theorem]

\setlength{\parindent}{0pt}

\setcounter{tocdepth}{5}

\MakePerPage{footnote}

\begin{document}
\bibliographystyle{alpha}

\maketitle
\tableofcontents

\section{Introduction}

A very important concept in mathematics is the idea of an \textit{invariant}:
An object which does not change under a certain action.
In 1872, Felix Klein came up with a then new method of describing geometries with group theory, called the Klein Erlangen program.
Here, the central idea of a geometry is characterized by its associated symmetry group, the group of transformations which leaves certain objects unchanged, for example: angles.
The study of these transformations is called conformal geometry.

Let us discuss the following important example in projective geometry:
Consider all transformations which map lines to lines, id est, such transformations under which the the property of being a line is invariant.
In real projective geometry, the fundamental theorem of projective geometry gives us that these maps are exactly the projective transformations.

\textbf{Conversely}, we can now just consider projective transformations as our given group of transformations.
\textbf{Invariant theory asks: What invariants exist?}
We can loosely notice a kind of duality between geometries viewed as in the Klein Erlangen program and invariant theory.
This discipline of mathematics usually only looks at invariants described with so called regular terms, or more concretely formulated:  In invariant theory, we try to find invariant polynomial-like functions.

Staying in our example of considering projective transformations as our given group, a well known example for an invariant is the cross ratio.
It is a rational polynomial which takes as its input four collinear points.
Is this the only invariant?
How can we find other invariants?
How big is the set (this will be a ring) of all invariants?

\textit{Hilbert's finiteness theorem} states that for regular actions under certain groups, such that are \textit{linearly reductive}, the invariant ring is finitely generated.
If we can find these finite generators, we have a grasp of what all invariants look like.
Hilbert's first proof for this theorem was non-constructive.
It is claimed\footnote{I read somewhere that it is not certain} that this proof was responsible for Gordan's famous quote ``Das ist Theologie und nicht Mathematik''.
The central idea of this proof is the existence of a Reynolds operator.

One of the most important and most common groups is the general linear group $\operatorname{GL}_n$.
It would be great if this group were linearly reductive.
But it is!
There are multiple ways to see this.
In a seminar I held, with the help of the Haar measure, I discussed a way to see that a module complement exists for every representation, making $\operatorname{GL}_n$ linearly reductive.
One can also show linear reductivity by the Schur-Weyl-dualty:  The symmetric group is finite, and therefore we can see that rational $\operatorname{GL}_n$ representations are semisimple, from which we can again construct module complements.

Here, we will show that $\operatorname{GL}_n$ is linear reductive in an even different way.
For one, we want to show that a Reynolds Operator exists, which already means that $\operatorname{GL}_n$ is linearly reductive.
But we want even more than just the existence!
What does it help for our motivation to get a grasp of what all (or even just some) invariants look like, if we merely prove the existence of a finite generator set for the invariants?
Since this operator projects polynomials to invariant polynomials, if we can find an explicit formula for computing the Reynolds operator applied to a polynomial, we can more easily receive concrete invariants.

\textbf{This is possible with \textit{Cayley's $\Omega$-process}!}
This is the main focus of my work.

I say ``more easily'' receive invariants, because if we take a polynomial at random and apply the Reynolds Operator, we might very likely just get a constant polynomial, which is not a very interesting invariant, and we also want to know if there are more invariants.
Similar to the first proof of Hilbert's finiteness theorem (by Hilbert himself), we can show that there are certain finitely many polynomials whose images under the Reynolds operator will generate the invariant ring.
Although this is not what I will be discussing in detail in my work, there is in fact an algorithm to compute these certain polynomials.
With the help of Cayley's $\Omega$-process, we then get a complete algorithm that gives us the generators of the invariant ring.

%%% Local Variables:
%%% mode: latex
%%% TeX-master: "roughdraft"
%%% End:


% In a seminar, I discussed Hilbert's finiteness theorem for the group $\operatorname{GL}_n$.
% The proof I presented was a version of Hilbert's non-constructive proof, which, apparently (it is not known how true this is), was responsible for Gordan's famous quote ``Das ist Theologie und nicht Mathematik''.
% The central idea of the proof is the existence of the Reynolds Operator.
% After the dirty work is done and some useful properties are shown, the proof of the theorem is very straightforward.
% In fact, if we are able to contruct finite homogenious generators of the nullcone, all that is left to do is apply the Reynolds Operator to each generator, and the resulting polynomials will be generators of the invariant ring.
% But even this step is easier said then done.
% In my presentation at the seminar, I constructed a Haar measure, which gives me a unitary $\operatorname{GL}_n$-invariant inner product, which allows us to define the Reynolds Operator. If one wants to explicitly calculate the image under the Reynolds Operator of a concrete polynomial, following these steps is not really feasible.
% For the group $\operatorname{GL}_n$, there is an operator called \textit{Cayley's $\Omega$-Process} which allows us to compute the Reynold's Operator.
% This is the main focus of my work.

% \section{Invariant Theory and the Reynolds Operator}

% In this section, we will explain and discuss some terminology and underlying theorems.
% This is very general, and since we really only focus on $\mathbb{C}[\operatorname{GL}_n]$, we can formulate a lot of this stuff more specifically and without the need to introduce some of these notions.
% Still though, I feel that it is beneficial to talk about the general framework to aid some readers to categorize this specific case.\\
% This loosely follows \cite[p.31]{DK15}.
% \begin{definition}[Regular Action]
  Let $G$ be a linear algebraic group, $X$ an affine variety.
  We call an action $G \times X \longrightarrow X$ a \textbf{regular action}, iff it is a morphism of affine varieties.
  We say \textbf{$ G $ acts regularly on $ X $}.
\end{definition}

\begin{definition}[Rational Representation]
  Let $G$ be a linear algebraic group.
  A representation $V$ of $G$ is called a \textbf{rational representation}, iff its corresponding action $ G \times V \longrightarrow V $ is a regular action.
\end{definition}

\begin{remark}
  A rational representation $ G \longrightarrow \operatorname{GL}\left(V\right) $ is of the following form:\\
  If $ a_{i,j} : G \longrightarrow K $ is the function of the $\left( i,j \right) $-entry, $ a_{i,j} \in K\lbrack G\rbrack $.\\
  In fact, it is equivalent to define a representation as rational, iff its map $ G \longrightarrow \operatorname{GL} \left( V \right) $ is a map of affine varieties.
\end{remark}

\begin{definition}[Invariants]
  Let $ G $ act on $ X $ regularly.
  \begin{equation}
    X^G := \left\{\, x \in X \mid \forall g \in G : g . x = x \,\right\}
  \end{equation}
  It defines a linear subspace.
  This given action induces an action $ G \times K\lbrack X\rbrack \longrightarrow K\lbrack X\rbrack $, where $K\lbrack X\rbrack$ is the coordinate ring, as follows:
  \begin{equation}
    \left( g , f \right) \longmapsto g \cdot f :=
    \left( x \mapsto f \left( \sigma^{-1} . x \right) \right)
  \end{equation}
  The \textbf{invariant ring} of the representation is defined as
  \begin{equation}
    K\lbrack X\rbrack^G := \left\{ \, f \in K\lbrack X \rbrack \mid \forall g \in G : g \cdot f = f \, \right\}
  \end{equation}
  As the name implies, $ K\lbrack X\rbrack^G $ defines a subalgebra of $ K\lbrack X\rbrack^G $.
\end{definition}

The general theme of my work revolves around the question of whether the invariant ring $ K\lbrack X\rbrack^G $ is finitely generated.

\textit{Hilbert's finiteness theorem} states that if the group $G$ is linearly reductive, $ K\lbrack V\rbrack^G $ is finitely generated.
The strict definition of ``linearly reductive'' is quite tricky, but we will give an alternate equivalent definition shortly.

\begin{definition}[Reynolds Operator]
  Let $ V $ be a rational representation of a linear algebraic group $ G $.
  A $ G $-invariant linear projection $ K\lbrack V\rbrack \longrightarrow K\lbrack V\rbrack^G $ is called a \textbf{Reynolds Operator}.
\end{definition}

\begin{remark}
  If a Reynolds Operator exists, it is unique (\textbf{??}).
  See \cite[p.39f]{DK15}: In the proof of the equivalences, in the step ``(b)$\implies$(c)'', only the existence of the Reynolds operator is needed.
  Therefore, the existence of the Reynolds Operator already implies its uniqueness (\textbf{??}).
\end{remark}

\begin{definition}[linearly reductive]
  A group G is called \textbf{linearly reductive} iff there exists a Reynolds operator for the regular action $ G \times G \longrightarrow G $ by left multiplication $ R_G : K\lbrack G \rbrack \longrightarrow K\lbrack G \rbrack^G = K $.
\end{definition}

\begin{remark}
  We could have also defined linear reductive groups as such, for which every regular action has a Reynolds Operator.
  We will prove that this is in fact equivalent.
\end{remark}


%%% Local Variables:
%%% mode: latex
%%% TeX-master: "roughdraft"
%%% End:


\section{Preliminary Work}\label{pw}

\bigskip

\subsection{Notation}

In the following, $K$ is a field of characteristic $0$ and $G$ a linear algebraic group, that is a group which is an affine variety, and whose multiplcation and inversion are morphisms of affine varieties.  \\
For us, zero is an element of the natural numbers.
Furthermore, for $n \in \mathbb{N}$ we write $[n] := \{\, m \in \mathbb{N} \mid 1 \leq m \leq n \,\}$.  \\
For an affine variety $X$, we denote by $K[X]$ the coordinate ring of $X$.
If $\{f_i\}_{i\in[r]} \subseteq K[X]$, we denote by $K[\{f_i\}_{i\in[r]}]$ the subring of $K[X]$ generated by $\{f_i\}_{i\in[r]}$.
For a finite-dimensional vector-space $V$, we denote by $X_i$ the coordinate functions for a given (often a canonical) basis.
For a set of functions in the coordinate ring $F \subseteq K[X]$ we denote by $Z(F)$ the zero set of $F$. % and for a set of points $P \subseteq X$ we denote by $V(P)$ the vanishing ideal of $P$.
For a subset of a ring $M$, $(M)$ denotes the ideal generated by $M$.

\subsection{Concepts From Algebraic Geometry}

\begin{proposition}[Rabinowitsch Trick]\label{rabbi}
  Let $V = K^n$ for some $n \in \mathbb{N}$.
  For a polynomial $p \in K[V] = K[\{X_i\}_{i\in[n]}]$, the set $ X_p := \{\, v \in V \mid p(v) \neq 0 \,\}$ has the structure of an affine variety with the coordinate ring $K[X_p] = K[\{X_i\}_{i \in [n]}, p^{-1}]$.  \\
  (Compare to \cite{Rab30})
\end{proposition}

\begin{proof}
  The set $X_p$ is not an algebraic set itself.
  The trick (the ``Rabinowitsch-trick'') is ``adding an additional variable $X_0$'', that means to consider $X_p$ as a subset of $K \times V$.
  We do this as follows:
  Consider the algebraic set $\tilde{X}_p := Z \left( X_0 \cdot p -1 \right) \subseteq K \times V$.
  We notice that $\tilde{X}_p = \{\, (p(v)^{-1},v) \in K \times V \mid v \in X_p \,\}$.
  This means that $X_p$ corresponds to $\tilde{X}_p$ via the bijection $\Phi \colon X_p \longrightarrow \tilde{X}_p$, $ v \leftrightarrow (1/p(v),v)$.
  The coordinate ring of $\tilde{X}_p$ can be written as $K[\bar{X_0}, \{\bar{X}_i\}_{i \in [n]}]$, where $\bar{X}_i = X_i \operatorname{mod} X_0 \cdot p -1$.
  Let $x \in X_p$.
  We have $\bar{X}_0 (\Phi(v)) = p(x)^{-1}$ and for $ i \in [n] $ we have $\bar{X_i}(\Phi(x)) = v_i$.
  This shows our claim: $X_p$ has the structure of an affine variety with the coordinate ring $K[X] = K[\{X_i\}_{i \in [n]}, p^{-1}]$.
  % We see that $X_p$ corresponds to $\tilde{X}_p$ by noticing that $\tilde{X}_p = \{\, (1/p(v), v) \in K \times V \mid v \in X_p \,\}$, which therefore means we have the one-to-one correspondence $(1/p(v),v) \leftrightarrow v$.
  % By definition $\tilde{X}_p$ is an affine variety, and it is also easy to see that $K[X_p] \cong K[\tilde{X}_p] = K[K\times V] / (X_0 \cdot p -1)$ via $X_0 \leftrightarrow p^{-1}$, including evaluations with the above described correspondence of points, which is easy to check.
\end{proof}

\begin{example}[The General Linear Group $\operatorname{GL}_n$]
  One of the most important examples is the general linear group $\operatorname{GL}_n$, which will be an essential theme in my work.
  By the above proposition this group is an affine variety via $p = \operatorname{det}$ with the coordinate ring $K[\{X_{i,j}\}_{i,j \in [n]}, \operatorname{det}^{-1}]$.
  This makes $\operatorname{GL}_n$ into a \textit{linear algebraic group}, that is a group which is an affine variety whose group operations of the multiplication and inversion are morphisms of affine varieties:
  The multiplication is just a polynomial function in each entry
  For the inversion each entry is a fraction of polynomials with $\operatorname{det}$ as the quotient, which means that each entry is in $K[\operatorname{GL}_n]$.
\end{example}

\begin{definition}[Algebraic Cohomomorphism For Product Spaces]\label{coh}
% We denote by $m$ the group multiplication of the group $G$.
% We want to view the pullback of $m$ as a map $m^\ast : K[G] \longrightarrow K[G] \otimes K[G]$, which makes sense, because $m$ and $\otimes$ are associative.
% The strict pullback, which I will call $ \hat{m} $, should be a map of the type $ K[G] \longrightarrow K[ G \times G] $, where $ f \mapsto f \circ m $.
% If we want to give the variables names, we can equivalently say it is a map $ \left. K[Z] \right|_G \longrightarrow \left. K[X,Y] \right|_{G \times G} $, where $ Z = \lbrace Z_1 , \ldots , Z_k \rbrace $, $X$ and $Y$ analogously (here, $ m $ canonically takes its left input via $ X $ and its right input via $ Y $).

  Let $m \colon U_1 \times U_2 \longrightarrow W$ be a morphism of affine varieties.
  The algebraic cohomomorphism $m^\ast$ of $m$ is just the pullback, that is a map of the type $m^\ast \colon K[W] \longrightarrow K[U_1 \times U_2]$.
  We have $ K[U_1 \times U_2] = K[\{X_k\}_{k\in[r]},\{Y_l\}_{l\in[s]}]$, where $\{X_k\}_{k\in[r]}$ and $\{Y_l\}_{l\in[s]}$ are generators of $K[U_1]$ and $K[U_2]$ respectively.
  
  % If we want to give the variables names, we can equivalently say it is a map $ \left. K[Z] \right|_W \longrightarrow \left. K[X,Y] \right|_{U_1 \times U_2} $, where $ Z = \lbrace Z_1 , \ldots , Z_k \rbrace $ $X$ and $Y$ analogously (here, $ m $ canonically takes its left input via $ X $ and its right input via $ Y $).
  
  The map
  % \begin{equation}
  %   \begin{aligned}
  %     t \colon K[\{X_k\}_{k\in[r]},\{Y_l\}_{l\in[s]}]
  %     & \longrightarrow K[\{X_k\}_{k\in[r]}] \otimes K[\{Y_l\}_{l\in[s]}] \\
  %     \sum_i \lambda_i \prod_j X_{j}^{d_{i,j}} \prod_k Y_{k}^{e_{i,k}} &\longmapsto \sum_i \lambda_i \prod_j X_{j}^{d_{i,j}} \otimes \prod_k Y_{k}^{e_{i,k}}
  %   \end{aligned}
  % \end{equation}
    \begin{equation}
    \begin{aligned}
      K[U_1 \times U_2]
      & \longrightarrow K[U_1] \otimes K[U_2]\\
      \sum_i \lambda_i \prod_j X_{j}^{d_{i,j}} \prod_k Y_{k}^{e_{i,k}} &\longmapsto \sum_i \lambda_i \prod_j X_{j}^{d_{i,j}} \otimes \prod_k Y_{k}^{e_{i,k}}
    \end{aligned}
  \end{equation}
  is independent of the choice of generators and independent of the representatives and therefore well-defined.
  This is an isomorphism, and each evaluation of $K[U_1]\otimes K[U_2]$ corresponds to exactly one element in $K[U_1 \times U_2]$, which means that in terms of algebraic geometry, we can view them as equal, in the sense that $K[U_1]\otimes K[U_2]$ describes the coordinate ring $K[U_1 \times U_2]$.
  We therefore write $m^\ast \colon K[W]\longrightarrow K[U_1]\otimes K[U_2]$
\end{definition}

\begin{remark}
One might ask why we use this notation ``$K[U_1]\otimes K[U_2]$''.
It helps to formalize performing operations only on the ``left part'' or the ``right part'', as we will soon see.
This notation is found in \cite{DK15}, but other literature such as \cite{Stu08} don't take this approach.
To give a very simple example:
If $G$ is a linear algebraic group and $m$ is its multiplication, for $f \in \left. K[Z] \right|_G$ we would write $\operatorname{id} \otimes \frac{\partial}{\partial Z_i} (m^\ast f)$ as in \cite{DK15}, whereas \cite{Stu08} would write $\frac{\partial}{\partial Y_i} (m^\ast f)$, often also written as $\frac{\partial}{\partial Y_i} (f(XY))$.
\end{remark}

\subsection{Concepts From Invariant Theory}

\begin{definition}[Regular Action, Rational Representation]
  Let $G$ be a linear algebraic group and $X$ an affine variety.
  We call an action $G \times X \longrightarrow X$ a \textbf{regular action}, if and only if $\mu$ is a morphism of affine varieties.
  We say \textbf{$ G $ acts regularly on $ X $}, and we also call $X$ a \textbf{$G$-variety}.\\
  For a finite-dimensional vector space $V$, let $\mu \colon G \times V \longrightarrow V$ be a representation in the classical sense, that is for all $g \in G$ we have $D_\mu (g) := (v \mapsto \mu(g,v)) \in \operatorname{GL}(V)$.
  We call $\mu$ a \textit{rational representation} if and only if it is regular.\\
  (See \cite[p.~31]{DK15})
  % We can view $V$ as an affine variety with respect to a basis.
  % We call $\mu$ a \textbf{rational representation} iff it is regular with respect to any basis.
  
  % For a finite dimensional vector space $V$ we call $D \colon G \longrightarrow \operatorname{GL}(V)$ a \textbf{regular representation} iff $\mu_D := ((g,v) \mapsto D(g)(v))$ is a regular action.
  % If an action $\mu \colon G \times V \longrightarrow V$ is given such that $D_\mu (g) := (v \mapsto g.v) \in \operatorname{GL}(V)$, we also call $\mu$ a regular representation (such $\mu$ and $D$ are in bijection).
\end{definition}

\begin{dexample}
  If $G$ is a linear algebraic group, then the multiplication $m \colon G \times G \longrightarrow G$ defines a regular action, meaning that $G$ itself is a $G$-variety.
\end{dexample}

\begin{definition}
  If $\mu \colon G \times V \longrightarrow V$ is a rational representation, we define a rational representation $\hat{\mu} \colon G \times V^\ast \longrightarrow V^\ast$ by $(\sigma,\varphi) \mapsto \sigma.\varphi := (v \mapsto \varphi(\tilde{\mu}(\sigma,v))= \varphi(\sigma^{-1}.v))$.
\end{definition}

% \begin{dexample}
%   A less trivial one
% \end{dexample}

% \begin{remark}
%   If $V$ is a finite-dimensional vector-space and $\mu \colon G \times V \longrightarrow V$ is a representation of $G$, then $\mu$ is a rational representation if and only if it is regular with respect to a single basis.
%   Really, one can define 
% \end{remark}

\begin{definition}[Rational Linear Action]\label{rr}
  Let $V$ be a vector space (not necessarily finite dimensional), and $ \mu : G \times V \longrightarrow V $ an action.
  We call $ \mu $ a \textbf{rational linear action} if and only if there exists a linear map $ \mu^\prime \colon V \longrightarrow K[G] \otimes V $ such that $ \mu \left( \sigma , v \right) = \left( \left( \epsilon_\sigma \otimes \operatorname{id} \right) \circ \mu^\prime \right) \left(v\right) $.\\
  (See \cite[A.1.7]{DK15})
\end{definition}

\begin{remark}
  From the definition, it should immediately be apparent that rational linear actions are linear and regular.
\end{remark}

\begin{definition}\label{funrep}
  Let $\mu \colon G \times X \longrightarrow X$ be a regular action.
  We define an action $\bar{\mu} \colon G \times K[X] \longrightarrow K[X]$ via $\bar{\mu}(\sigma,f)(x) := f(\mu(\sigma^{-1},x))$, and we write $\sigma.f (x) := f( \sigma^{-1}.x )$, where  $\sigma \in G$, $f \in K[X]$ and $x \in X$.\\
  This action is obviously regular, but it is also easily shown that it is in fact a rational linear action:
  If $\tilde{\mu} \colon G \times X \longrightarrow X$ is the morphism of affine varieties (it is in fact a right action) defined by $ (\sigma,x)\mapsto\tilde{\mu} (\sigma,x) := \mu (\sigma^{-1},x)$, then we can define $ \bar{\mu}^\prime := \tilde{\mu}^\ast $ with the desired properties.
\end{definition}

\begin{proposition}\label{rara}
  Let $X$ be an affine $G$-variety.
  If for $f \in K[X]$ we have $\bar{\mu}^\prime (f) = \Sigma_{i =1}^r p_i \otimes g_i $, then for every $\sigma \in G$ we have $\bar{\mu}^\prime (f) = \Sigma_{i=1}^r \sigma.p_i \otimes \sigma.g_i$.
\end{proposition}

\begin{proof}
  Let $\tau \in G$ and $x \in X$.
  Then
  \begin{equation}
    \begin{aligned}
      &\Sigma_{i=1}^r \sigma.p_i \otimes \sigma.g_i (\tau,x)
      &&= \Sigma_{i=1}^r p_i(\sigma^{-1}\tau) \otimes g_i (\sigma^{-1}.x)  \\
      &&&= \sigma^{-1}\tau.f(\sigma^{-1}.x)  \\
      &&&= \tau.f(x) &&= \bar{\mu}^\prime (f) (\tau,x)
    \end{aligned}
  \end{equation}
\end{proof}

\begin{definition}\label{back}
  Let $V$ be a finite dimensional vector-space $\mu \colon G \times V \longrightarrow V$ a rational representation.
  We then define an action $\hat{\mu} \colon G \times V^\ast \longrightarrow V^\ast$, $ (\sigma,\varphi) \mapsto \sigma.\varphi := (v \mapsto \varphi(\mu(\sigma^{-1},v)) = \varphi(\sigma^{-1}.v))$, which is a rational represenation of $G$.
\end{definition}

\begin{definition}\label{rac}
  Let $G$ be a linear algebraic group with the multimplication $m \colon G \times G \longrightarrow G$.
  For $\sigma \in G$ and for $p \in K[G]$ we define $\sigma\dot{\phantom{.}}p := (\tau \mapsto p(\tau\sigma))$.
\end{definition}

\begin{proposition}\label{roro}
  Let $X$ be an affine variety and $\mu \colon G \times X \longrightarrow X$ a regular action.
  For $f \in K[X]$, if we have $\bar{\mu}^\prime (f) = \Sigma_{i = 1}^r p_i \otimes g_i$ for some $\{g_i\}_{i\in [r]}$, then for $\sigma \in G$ we get $\bar{\mu}^\prime (\sigma.f) = \Sigma_{i = 1}^r \sigma \dot{\phantom{.}} p_i \otimes g_i$.
\end{proposition}

\begin{proof}
  For $f \in K[X]$ we have $ \bar{\mu}^\prime (f) = \Sigma_{i=1}^r p_i \otimes g_i$ for some $\{g_i\}_{i \in [r]}$.
  Now let $\sigma \in G$.
  Then for all $\tau \in G$ and for all $x \in X$ we have
  \begin{equation}
    \begin{aligned}
      &\bar{\mu}^\prime (\sigma.f) (\tau,x)
      &&=((\epsilon_\tau \otimes \operatorname{id}) \circ \bar{\mu}^\prime) (\sigma.f) (x)\\
      &&&=(\tau.(\sigma.f)) (x)\\
      &&&=\Sigma_{i=1}^r p_i(\tau\sigma) g_i(x)\\
      &&&=\Sigma_{i=1}^r \sigma\dot{\phantom{.}}p_i(\tau) g_i(x)
      &&=(\Sigma_{i=1}^r \sigma \dot{\phantom{.}}p_i \otimes g_i) (\tau,x)
    \end{aligned}
  \end{equation}
\end{proof}

\begin{definition}[locally finite]
  For a vector space $V$, we call an action $\mu \colon G \times V \longrightarrow V$ \textbf{locally finite}, if and only if for every $v \in V$ there exists a $G$-stable finite-dimensional vector space $U \subseteq V$ such that $v \in U$.
\end{definition}

\begin{definition}
  Let $V$ be a vector-space and $\mu \colon G \times V \longrightarrow V$ an action.
  For $v \in V$ we define $V_v := \operatorname{span} G.v$.
\end{definition}

\begin{remark}
  $V_v$ is always a $G$-stable subspace of $V$.
  For any $G$-stable subspace $W \subseteq V$ we have $V_v \subseteq W$.
  Therefore, an action $\mu \colon G \times V \longrightarrow V$ is locally finite if and only if $V_v$ is finite-dimensional.
\end{remark}

\begin{proposition}\label{locfin}
  Let $V$ be a vector space.
  \begin{enumerate}[(a)]
  \item If $\mu \colon G \times V \longrightarrow V$ is a rational linear action, then the action is locally finite, and every
    finite-dimensional $G$-stable subspace $W$,
    $\left. \mu \right|_{G\times W}$ is a rational
    representation.
  \item If $V$ is a finite-dimensional vector space and $\mu \colon G \times V \longrightarrow V$ is a rational representation, then $\mu$ is also a rational linear action.
  \end{enumerate}
\end{proposition}

\begin{proof}
  See \cite[A.1.8]{DK15} and \cite[2.2.5, 2.2.6]{DK15}

  \underline{(a)}\\
  Assume that $\mu$ is a rational linear action.
  Let $v \in V$.
  We can write $\mu^\prime (v) = \Sigma_{i=1}^l f_i \otimes v_i$. %with $f_i$ linearly independant over $K$.
  We then easily see that $V_v \subseteq \operatorname{span}\{v_i\}_{i=1}^l$, showing that the action is locally finite.
  Since $\mu^\prime$ is linear, $\mu$ is also linear, therefore we immediately get that $\left. \mu \right|_{G\times W}$ is a rational representation.\\
  \underline{(b)}\\
  Let $V$ be a finite-dimensional vector-space and $\mu \colon G \times V \longrightarrow V$ a rational representation.
  This means that for all $\sigma \in G$ we have $D_\mu (\sigma) \in \operatorname{GL}(V)$.
  Let us now choose a basis $\{v_i\}_{i \in [r]}$ of $V$.
  For all $\sigma \in G$ there then exist unique $\{ \left( D_\mu \right)_{i,j}\}_{i,j \in [r]} \subseteq K$ such that for all $i \in [r]$ we have $\mu (\sigma,v_i) = \Sigma_{k=1}^r \left(D_\mu\right)_{i,k} v_k$.
  Since the action is regular, we must have $p_{i,j} := \left( \mu \mapsto \left(D_\mu\right)_{i,k}\right) \in K[G]$.
  We now define $\mu^\prime \colon V \longrightarrow K[G] \otimes V$ as the linear extension of $v_i \mapsto \Sigma_{k=1}^r p_{i,k} \otimes v_k$ where for $i \in [r]$.
  It should be clear that $\mu^\prime$ satisfies $ \mu \left( \sigma , v \right) = \left( \left( \epsilon_\sigma \otimes \operatorname{id} \right) \circ \mu^\prime \right) \left(v\right) $ for all $\sigma \in G$ and $v \in V$.
  This shows that $\mu$ is a rational linear action.
  % By assumption there exists a finite-dimensional $G$-stable vector space $W \subseteq V$ such that $v \in W$.
  % Choose a basis $\{w_1, \ldots, w_r\}$ of $W$, where $w_1 = v$ (this is done to make some terms easier, but choosing any basis also works).
  % Since $\left. \mu \right|_{G\times W}$ is a rational representation, we have $D_\mu (\sigma) \in \operatorname{GL}(W)$ for all $\sigma \in G$, and since our action is also regular, there exist $\{p_i\}_{i \in [r]} \subseteq K[G]$ such that for all $\sigma \in G$ we have $\mu(\sigma, v) = D_\mu(\sigma)(v) = \Sigma_{i=1}^r p_i(\sigma) \cdot w_i$.
  % Now define $\mu^\prime (v) := \Sigma_{i=1}^r p_i \otimes w_i$.
  % We shall now check that this is well-defined:
  \end{proof}
% \begin{definition}[Rational Representation]
%   Let $G$ be a linear algebraic group.
%   A representation $V$ of $G$ is called a \textbf{rational representation}, iff its corresponding action $ G \times V \longrightarrow V $ is a regular action.
% \end{definition}

% \textbf{Claim:} If $V$ is finite dimensional, the notions of the definitions coincide.\\
% \textit{Proof:} First, let $V$ be a rational representation of $G$ with basis $\{ v_1 , \ldots , v_N \}$ be a basis of $V$.
% By our assumption, we have a rational representation, therefore there exist $p_{i,j} \in K \left\lbrack G \right\rbrack$ such that $\mu\left( \sigma, v_j \right) = \Sigma_{i=1}^{N} p_{i,j}\left(\sigma\right) \cdot v_i$.
% Define $\mu^\ast \left( v_j \right) := \Sigma_{i=1}^{N} p_{i,j} \otimes v_i$.
% Now we easily see:
% \begin{equation}
%   \begin{aligned}
%     \mu\left(\sigma,v\right)
%     &= \mu \left(\sigma, \Sigma_{j=1}^N \lambda_j v_j \right) \\
%     &= \sum_{j=1}^N \lambda_j  \sum_{i=1}^N p_{i,j}\left(\sigma\right) \cdot v_i \\
%     &= \sum_{j=1}^N \lambda_j \left(\left(\epsilon_\sigma \otimes \operatorname{id} \right) \circ \mu^\ast \right) \left(v_j \right)
%     &= \left(\left(\epsilon_\sigma \otimes \operatorname{id} \right) \circ \mu^\ast \right) \left(v \right)
%   \end{aligned}
% \end{equation}
% \textit{which was to show.}

\begin{remark}
  This shows that for a finite-dimensional vector space $V$, an action is rational if and only if it defines a rational representation.
  In other words, we have shown that rational representations are exactly defined by rational linear actions on finite-dimensional vector-spaces, which justifies the choice of the names of our definitions.
\end{remark}

\begin{remark}
  A rational representation $\mu \colon G \times V \longmapsto V$ is of the following form:\\
  Consider $D_{\mu} \colon G \longmapsto \operatorname{GL}(V)$.
  If then $ a_{i,j} : G \longrightarrow K $ is the function of the $\left( i,j \right) $-entry of $D_{\mu}$, then $ a_{i,j} \in K\lbrack G\rbrack $.\\
  In fact, it is equivalent to define a representation $\mu \colon G \times V \longrightarrow V$ ($V$ finite dimensional) as rational, iff $D_{\mu} \colon G \longrightarrow \operatorname{GL}(V)$ is a map of affine varieties.
\end{remark}

\begin{definition}[Invariants]
  Let $ G $ act on $ X $ regularly.
  \begin{equation}
    X^G := \left\{\, x \in X \mid \forall g \in G : g . x = x \,\right\}
  \end{equation}
  This defines a linear subspace.
  The given action induces an action $ \bar{\mu} \colon G \times K\lbrack X\rbrack \longrightarrow K\lbrack X\rbrack $ as per definition \ref{funrep}.
  % \begin{equation}
  %   \left( g , f \right) \longmapsto g \cdot f :=
  %   \left( x \mapsto f \left( \sigma^{-1} . x \right) \right)
  % \end{equation}
  The \textbf{invariant ring} of the representation is defined as
  \begin{equation}
    K\lbrack X\rbrack^G := \left\{ \, f \in K\lbrack X \rbrack \mid \forall g \in G : g . f = f \, \right\}
  \end{equation}
  As the name implies, $ K\lbrack X\rbrack^G $ defines a subalgebra of $ K\lbrack X\rbrack^G $.
\end{definition}

The general theme of my work revolves around the question of whether the invariant ring $ K\lbrack X\rbrack^G $ is finitely generated.

\textit{Hilbert's finiteness theorem} states that if the group $G$ is linearly reductive, $ K\lbrack V\rbrack^G $ is finitely generated.
The strict definition of ``linearly reductive'' is quite tricky, but we will shortly give alternate characterizations.

%%% Local Variables:
%%% mode: latex
%%% TeX-master: "roughdraft"
%%% End:

\section{Linearly Reductive Groups, The Reynolds Operator And Hilbert's Finiteness Theorem}

\subsection{The Reynolds Operator And Linearly Reductive Groups}

\begin{definition}[Linearly Reductive Group]
  Let $G$ be a linear algebraic group.
  We call $G$ \textbf{linearly reductive}, if and only if for any finite rational representation $V$ and for any $v \in V^G \setminus \{ 0 \}$ there exists an $f \in \left( V^\ast \right)^G $ such that $f(v) \neq 0$.
\end{definition}

\begin{proposition}\label{dual}
  For a linearly reductive group $G$ and a finite rational representation $V$, $V^G$ and $(V^\ast)^G$ are dual to each other with respect to the non-degenerate bilinear form $ b\colon V^\ast \times V \longrightarrow K, (f,v) \mapsto f(v)$.
\end{proposition}

\begin{proof}
  We shall first show that $\operatorname{dim}(V^\ast)^G = \operatorname{dim}V^G$.
  Let $\{v_1, \ldots , v_r \}$ be a basis of $V^G$, and extend this to a basis $\{v_1, \ldots , v_m \}$ of $V$.
  We define $\{ f_1 , \ldots , f_n \}$ to be the basis of $V^\ast$ which is dual to $\{v_1, \ldots , v_n \}$, that is we have $ f_i (v_j) = \delta_{i,j} $.
  It should be clear that $\{f_1, \ldots , f_r \} \subseteq (V^\ast)^G$ \textbf{(NO! There is something wronge here!!)}.
  Now let $f \notin \operatorname{span} \{ f_1, \ldots, f_r\}$, that is we have $f = \Sigma_{i=1}^n \lambda_i f_i $, and there exists a $j > r$ such that $ \lambda_j \neq 0$.
  For this $j$, we have $v_j \notin V^G$, therefore there exists a $\sigma \in G$ such that $\sigma . v_j \neq v_j$.
  We then get \textbf{(there is a lot wrong here...)}
  We have now shown that $(V^\ast)^G = \operatorname{span}\{f_1,\dots,f_r\}$, therefore $\operatorname{dim}(V^\ast)^G = \operatorname{dim}V^G$.
  Since $G$ is linearly reductive, we then get via our definition that $\left. b \right|_{(V^\ast)^G \times V^G}$ is non-degenerate in the first variable.
  Since $\operatorname{dim}(V^\ast)^G = \operatorname{dim}V^G$, we have that $\left. b \right|_{(V^\ast)^G \times V^G}$ is non-degenerate in both variables.
  This exactly means that the spaces $(V^\ast)^G$ and $V^G$ are dual to each other with respect to $b$.
\end{proof}

% \begin{proposition}\label{aybee}
%   Let $G$ be linearly reductive.
%   Then for every rational representation $V$ there exists a unique subrepresentation $W \subseteq V$ such that $V = V^G \oplus W$.
%   For this subrepresentation $W$ we have $(W^\ast)^G = \{0\}$.
% \end{proposition}

\begin{definition}[Reynolds Operator]
  Let $ X $ be an affine $G$-variety.
  A $ G $-invariant linear projection $R \colon K\lbrack X \rbrack \longrightarrow K\lbrack X \rbrack^G $ is called a \textbf{Reynolds operator}.
\end{definition}

\begin{definition}
  Assume that $V$ is a finite rational representation of $V$ such that there exists a unique subrepresentation $W$ of $V$ such that $V = V^G \oplus W$.
 We define $R_V \colon V \twoheadrightarrow V^G$ as the linear projection of $V$ onto $V^G$ along $W$.
\end{definition}

\begin{remark}
  $R_V$ is a $G$-invariant projection of $V$ onto $V^G$:
  If for $v \in V$ we write $v = u + w$ with $u \in V^G$ and $w \in W$, then for $\sigma \in G$ we have $\sigma.v = \sigma.u + \sigma.w = u + \sigma.w$, and therefore $R_V(\sigma.v) = u = R_V(v)$.
\end{remark}

\begin{lemma}\label{lamm}
  Assume that $G$ is a linear algebraic group with the following property:
  For every finite rational representation $V$ of $G$ there exists a unique subrepresentation $W$ of $V$ such that $V = V^G \oplus W$, and for this $W$ we have $(W^\ast)^G = \{0\}$.
  The following properties hold:
  \begin{enumerate}[(a)]
  \item If $V$ is a subrepresentation of a finite rational representation $V^\prime$ of $G$, we have $\left. R_{V^\prime} \right|_V = R_V$.
  \item If $V$ is a finite rational representation of $G$ and $R^\prime_V \colon V \longrightarrow Y$ is a $G$-invariant linear map with $V \subseteq Y$ and $ \left. R^\prime_V \right|_{V^G} = \operatorname{id}_{V^G}$, we have $R^\prime_V = R_V$, id est $R_V$ is unique with this property (\textbf{Do I need to mention that we should then view $R_V \colon V \longrightarrow V$ instead of $ \twoheadrightarrow V^G$??}).
  \item If $X$ is an affine $G$-variety and $R \colon K[X] \twoheadrightarrow K[X]^G$ is a Reynolds operator, then for every $G$-stable subspace $V$ of $K[X]$ we have $\left. R \right|_V = R_V$.
  \item If $X$ is an affine $G$-variety, $R \colon K[X] \twoheadrightarrow K[X]^G$ a Reynolds operator and $W$ is any $G$-stable subspace of $K[X]$, we have $R(W) = W^G$.
  \item If $X$ is an affine $G$-variety, the Reynolds operator for $K[X]$ is unique
  \end{enumerate}
\end{lemma}

\begin{proof}
  \hfill \break
  \underline{(a)}\\
  Let $V$ be a subrepresentation of a finite rational representation $V^\prime$ of $G$.
  We write $V = V^G \oplus W$ and $V^\prime = (V^\prime)^G \oplus W^\prime$, where $W$ and $W^\prime$ are each the unique subrepresentations of $V$ and $V^\prime$ repspectively with this property as in our assumption.
  Let $w \in W$.
  We write $w = u^\prime + w^\prime$ where $u^\prime \in (V^\prime)^G$ and $w^\prime \in W^\prime$.
  We choose a basis $\{u^\prime_i\}_{i \in [r]}$ of $(V^\prime)^G$ and $\{w^\prime_j\}_{j \in [s]}$ of $W^\prime$ and write $w = \Sigma_{i=1}^r \lambda_i u^\prime_i + \Sigma_{j=1}^s \mu_j w^\prime_j$.
  For $i \in [r]$, let us consider $\hat{u}^\prime_i \in (V^\prime)^\ast$, the dual basis element of $u^\prime_i$ with respect to the basis $\{u^\prime_i\}_{i \in [r]} \cup \{w^\prime_j\}_{j \in [s]}$ of $V^\prime$.
  Because of our assumption we have $(W^\ast)^G = \{0\}$, so we must have $\left. \hat{u}^\prime_i \right|_W = 0$, and therefore $\lambda_i = \hat{u}^\prime_i (w) = \left. \hat{u}^\prime_i \right|_W (w) = 0$.
  We retreive $u^\prime = 0$, implying $ w  = w^\prime \in W^\prime $.
  We have now shown $W \subseteq W^\prime$.
  Let $v \in V$.
  With $V^G \subseteq (V^\prime)^G$ and $R_V (v) - v \in W \subseteq W^\prime$, we retrieve $R_{V^\prime}(v) - R_V (v) = R_{V^\prime}(v - R_V(v)) = 0$.
  This concludes $\left. R_{V^\prime} \right|_V = R_V$.  \\
  \underline{(b)}\\
  Let $V$ be a finite rational representation of $G$, and let $R^\prime_V \colon V \longrightarrow Y$ be a $G$-invariant linear map where $V \subseteq Y$.
  Via our assumption, we can find a unique subrepresentation $W$ of $V$ such that $V = V^G \oplus W$.
  We obviously have $\left. R^\prime_V \right|_{V^G} = \operatorname{id}_{V^G} = \left. R_V \right|_{V^G}$.
  Let $w \in W$.
  We choose a basis $\{w_i\}_{i \in [r]}$ of $U:= \operatorname{span}(W + R^\prime_V (w))$, and we write $R^\prime_V (w) = \Sigma_{i=1}^r \lambda_i w_i$.
  Let $\{w^\prime_i\}_{i \in [r]}$ be the basis of $U^\ast$ dual to the previously mentioned basis of $U$.
  For $i \in [r]$, we have $\left. (w^\prime_i \circ R^\prime_V) \right|_W \in (W^\ast)^G = \{0\}$ via our assumption, and therefore $ \lambda_i = w^\prime_i (R^\prime_V(w)) = \left. (w^\prime_i \circ R^\prime_V) \right|_W (w) = 0$.
  This means that $R(w) = 0$.
  We now have shown $\left. R \right|_{W} = 0$.
  This concludes that $R^\prime_V = R_V$.  \\
  \underline{(c)}\\
  This follows immediately from (b):
  If $X$ is an affine $G$-variety and $R \colon K[X] \twoheadrightarrow K[X]^G$ is a Reynolds operator and $V$ is a $G$-stable subspace of $K[X]$, we have that $\left. R \right|_V \colon V \longrightarrow K[X]$ is a linear map with $V \subseteq K[X]$ and $\left. R_V \right|_{V^G} = \operatorname{id}_{V^G}$.
  Therefore we have $\left. R \right|_V = R_V$.  \\
  \underline{(d)}\\
  Let $X$ be an affine $G$-variety, $R \colon K[X] \twoheadrightarrow K[X]^G$ a Reynolds operator and $W$ is any $G$-stable subspace of $K[X]$.
  now let $w \in W$.
  Since $W$ is $G$-stable we have $V_w \subseteq W$ and with (c) therefore $R(w) = R_{V_w} (w) \in V_w^G \subseteq W^G$.
  We have therefore shown $R(W) \subseteq W^G$.
  Also $\left. R \right|_{W^G} = \operatorname{id}_{W^G}$ since $W^G \subseteq K[X]^G$, concluding $R(W) = W^G$.  \\
  \underline{(e)}\\
  This follows immediately from (c):
  Let $X$ be an affine $G$-variety and $R_1,R_2 \colon K[X] \twoheadrightarrow K[X]^G$ each a Reynolds operator.
  Now let $f \in K[X]$.
  Then $R_1(f) = R_{V_f} (f) = R_2 (f)$.
\end{proof}

\begin{remark}
  $K[V]_d$, that is the subspace of all homogeneous polynomials of degree $d$, is a $G$-stable subspace of $K[V]$.
  Since $K[V] = \bigoplus_{d \geq 0} K[X]_d$, we therefore also have $K[V]^G = \bigoplus_{d \geq 0} K[V]_d^G$, which means that all $R_{K[X]_d}$ characterize $R$.
  This is important for the proof of Hilbert's finiteness theorem.
\end{remark}

\begin{remark}
  Note that in lemma \ref{lamm}(e) we just showed uniqueness without mentioning existence.
  In the following, we see that in fact there always exists a Reynolds operator for groups with the previously described properties.
\end{remark}

% \begin{remark}
%   If a Reynolds Operator exists, it is unique (?).
%   See \cite[p.39f]{DK15}: In the proof of the equivalences, in the step ``(b)$\implies$(c)'', only the existence of the Reynolds operator is needed.
%   Therefore, the existence of the Reynolds Operator already implies its uniqueness (?).
% \end{remark}

\begin{theorem}\label{equiv}
  Let $G$ be a linear algebraic group.
  The following are equivalent:
  \begin{enumerate}[(a)]
  \item $G$ is linearly reductive
  \item For every finite rational representation $V$ of $G$ there exists a unique subrepresentation $W$ with $V = V^G \oplus W$.
    For this subrepresentation $W$ we have $(W^\ast)^G = \{0\}$.
    % This subrepresentation $W$ satisfies $\left( W^\ast \right)^G = \{0\}$.
  \item For every affine $G$-variety $X$ there exists a Reynolds operator $R \colon K[X] \twoheadrightarrow K[X]^G $.
  \end{enumerate}
\end{theorem}

\begin{proof}
  \hfill \break
  \underline{(a)$\implies$(b)}\\
  Let $V$ be a finite rational representation of $G$.
  Consider the subspace $ ((V^\ast)^G)^\bot \subseteq V $.
  It is easily seen that this is a subrepresentation of $V$.
  In proposition \ref{dual}, we showed that $V^G$ and $(V^\ast)^G$ are dual to each other.
  For this reason, we have $V = V^G \oplus ((V^\ast)^G)^\bot $.
  We have shown the existence, now we shall show uniqueness.
  Let $W$ be a subrepresentation of $V$ with $V = V^G \oplus W $.
  Again, it is easily seen that $W^\bot \subseteq V^\ast$ is a subrepresentation.
  $G$ must act trivially on $W^\bot \subseteq V^\ast$:
  Let $f \in W^\bot$, and let $\sigma \in G$.
  We have $\sigma.f \in W^\bot$ and therefore $\sigma.f - f \in W^\bot$.
  Now, let $v \in V$.
  We write $v = u + w$ for (unique) $u \in V^G$ and $w \in W$ and compute:
  \begin{equation}
    \begin{aligned}
      &(\sigma.f -f)(v)&=&(\sigma.f -f)(u) + (\sigma.f -f)(w)\\
      &&=&f(\sigma^{-1}.u) - f(u) + 0\\
      &&=&f(u)-f(u) = 0
    \end{aligned}
  \end{equation}
  Which means that $\sigma.f = f$.
  Hence $G$ does act trivially on $W^\bot$.
  This means that $W^\bot \subseteq (V^\ast)^G$.
  But we also have $\operatorname{dim}W^\bot = \operatorname{dim}V^G = \operatorname{dim}(V^\ast)^G$, which implies $W^\bot = (V^\ast)^G$, and therefore also $W = (W^\bot)^\bot = ((V^\ast)^G)^\bot$, which concludes the claim of uniqueness.
  Finally, we notice that that $W$ and $W^\ast$ are isomorphic representations \textbf{(How clear is this??)}, which also means that $(W^\ast)^G$ and $W^G$ are isomorphic.
  Since we have $W^G = \{0\}$, we therefore must also have $(W^\ast)^G = \{0\}$.\\
  \underline{(b)$\implies$(c)}\\
  Let $X$ be an affine $G$-variety.
  Let $f \in K[X]$.
  We define the map $R \colon K[X] \longrightarrow K[X]^G $, $ f \mapsto R_{V_f}(f)$.
  For $f \in K[X]$ we denote by $W_f$ the unique subrepresentation of $V_f$ such that $V_f = V_f^G \oplus W_f$ as in (b).
  This map is linear:
  Let $f,g \in K[X]$ and $\lambda \in K$.
  We notice that $V_f,V_g,V_{\lambda f + g} \subseteq V_f + V_g$, which together with lemma \ref{lamm}(a) gives us $R(\lambda f +g) = R_{V_{\lambda f +g}}(\lambda f+g) = R_{V_f +V_g}(\lambda f+g) = \lambda R_{V_f + V_g} (f) + R_{V_f + V_g}(g) = \lambda R_{V_f} (f) + R_{V_g}(g) = \lambda R(f) + R(g)$.
  The map $R$ is also a projection onto $K[X]^G$, since for each $f \in K[X]$ we have $V_f^G \subseteq K[X]^G$.
  $R$ is also $G$-invariant, since for all $f \in K[X]$ $R_{V_f}$ is $G$-invariant and for all $\sigma \in G$ we have $V_f = V_{\sigma.f}$.
  This concludes that $R$ is a Reynolds operator, which shows (c).  \\
  \underline{(c)$\implies$(a)}  \\
  Let $V$ be a finite rational representation of $G$ and let $v \in V^G \setminus \{0\}$.
  We choose a basis $\{v_i\}_{i\in [r]}$ of $V$ with $v_1 = v$.
  Let $\hat{v} \in V^\ast$ be the dual basis vector of $v$ with respect the afore mentioned basis.
  Now we define $p_v \colon K[V^\ast] \twoheadrightarrow K$, $f \mapsto f(\tilde{v})$.
  Consider the isomorphism $\Phi \colon V \longrightarrow (V^\ast)^\ast$, $w \mapsto (\phi \mapsto \phi (w))$.
  We have $(V^\ast)^\ast \subseteq K[V^\ast]$.
  Since $V^\ast$ is a finite rational representation and since via our assumption (c) we have a Reynolds operator $R \colon K[V^\ast] \twoheadrightarrow K[V^\ast]^G$, we can define $ \psi_v := p_v \circ R \circ \Phi \colon V \longrightarrow K$.
  Since each map is linear, we have $\psi_v \in V^\ast$.
  After we notice that since $v \in V^G$ we have $\Phi (v) \in K[V^\ast]^G$, we can calculate $\psi_v (v) = p_v (\Psi(v)) = \Phi (v) (\tilde{v}) = \tilde{v} (v) = 1 \neq 0$.
  This concludes that $G$ is linearly reductive, showing (a).
\end{proof}

\begin{theorem}\label{decomp}
  If $K$ is an algebraically closed field, then a linear algebraic group $G$ is linearly reductive if and only if $G$ is semisimple, that is for every finite rational representation $V$ of $G$ and subrepresentation $W$ of $V$ there exists a subrepresentation $Z$ of $V$ such that $V = W \oplus Z$.
\end{theorem}

\begin{proof}
  Asume that $G$ is linearly reductive.
  Now let $V$ be a finite rational representation of $G$.\\
  Let us first assume that we have an irreducible subrepresentation $W$ of $V$.
  We can identify $\operatorname{Hom}_K(W,V)^\ast$ with $\operatorname{Hom}_K(V,W)$ via  $A \leftrightarrow (B \mapsto k^{-1}\operatorname{tr}(A \circ B))$ where $k \in \mathbb{N}$ is the dimension of $W$.
  If we let $G$ act on $ \operatorname{Hom}_K(W,V)$ by $ \sigma.B :=  w \mapsto \sigma . (B(w))$ and on $\operatorname{Hom}_K(V,W)$ by $ \sigma.A := v \mapsto A(\sigma^{-1}.v) $, we then see that our identification $A \leftrightarrow (B \mapsto k^{-1}\operatorname{tr}(A \circ B))$ is an isomorphism of representations between $\operatorname{Hom}_K(W,V)^\ast$ and $\operatorname{Hom}_K(V,W)$. %meaning $\operatorname{Hom}_K(V,W)^G$ corresponds to $(\operatorname{Hom}_K(W,V)^\ast)^G$.
  Now let $B \in \operatorname{Hom}_K(W,V)^G$ be the inclusion map.
  Since $G$ is linearly reductive, there exists an $A \in \operatorname{Hom}_K(V,W)^G$ such that $k^{-1} \operatorname{tr}(A \circ B) \neq 0$.
  Since $K$ is algebraically closed and since $W$ is irreducible, Schur's lemma \textbf{(CITE!)} gives us that $A \circ B$ must be a non-zero multiple of the identity map.
  Therefore, if $Z$ is the kernel of $A$, which is a subrepresentation of $V$ since $A$ is $G$-invariant, we have $V = W \oplus Z$.\\
  Now let us prove the claim for an arbitrary subrepresentatio $W$ of $V$ by induction over $k := \operatorname{dim}$.
  If $k=0$ the statement is trivial.
  Assume that for $k \in \mathbb{N}$ the statement is true for all $m \leq k$.
  Now let $\operatorname{dim}W = k +1$.
  We choose a non-trivial irreducible subrepresentation of $W$, say $W^\prime := \operatorname{span}G.w$ for some $w \in W \setminus \{0\}$.
  By what we showed earlier, there exists a subrepresentation $Z^\prime$ of $V$ such that $V = W^\prime \oplus Z^\prime$.
  We also have that $W \cap Z^\prime$ is a subrepresentation $V$ and $W = W^\prime \oplus W \cap Z^\prime$.
  Since $W^\prime$ is non-trivial, we get $\operatorname{dim} W \cap Z^\prime \leq k$, and therefore by induction hypothesis there exists a subrepresentation $Z$ of $Z^\prime$ such that $Z^\prime = W \cap Z^\prime \oplus Z$.
  We then have $V = W^\prime \oplus Z^\prime = W^\prime \oplus W \cap Z^\prime \oplus Z = W \oplus Z$.
  This shows the forwards implication of our initial claim.\\  
  Now assume that for every finite rational representation $V$ of $G$ and subrepresentation $W$ of $V$ there exists a subrepresentation $Z$ of $V$ such that $V = W \oplus Z$.
  Let $V$ be a finite rational representation of $G$.
  By our assumption there exists a subrepresentation $W$ of $V$ such that $V = V^G \oplus W$.
  If we have $v \in V^G \setminus \{0\}$, we can extend to a basis $B_{V^G}$ of $V^G$ with $v \in B_{V^G}$.
  Now we choose any basis $B_W$ of $W$ and can define $\phi_v \in V^\ast$ to be the dual vector of $v$ with respect to the basis $B_{V^G} \cup B_W$.
  We then have $\phi_v \in (V^\ast)^G $ and $\phi_v (v) = 1 \neq 0$.
  This means that $G$ is linearly reductive.

  We have now proven both implications of our claim.
\end{proof}

\subsection{Hilbert's Finiteness Theorem}

\begin{proposition}
  See \cite[p.41 Corollary 2.2.7]{DK15}\\
  Let $G$ be a linearly reductive group, and let $ R \colon K[X] \twoheadrightarrow K[X]^G $ be the Reynolds operator for an affine $G$-variety $X$.
  If $f \in K[X]^G$ and $g \in K[X]$ we have $R(fg) = fR(g)$, id est the Reynolds operator is a \textit{$K[X]^G$-module homomorphism}.
\end{proposition}

\begin{proof}
  Let $f \in K[X]^G$ and $g \in K[X]$.
  by theorem \ref{equiv}, we can decompose $V_g = V_g^G \oplus W_g$ uniquely, where $W_g$ is a subrepresentation of $V_g$, and we also have $(W_g^\ast)^G = \{0\}$.
  $fV_g$ is also a representation of $G$ with subrepresentations $fV_g^G$ and $fW_g$ of $G$ and we notice that $(fV_g)^G = fV_g^G$.
  We easily check that the map $R_{V_g}^\prime \colon fV_g \longrightarrow fV_g$, $fh \mapsto f R(h)$ is a $G$-invariant linear map with $\left. R_{fV_g}^\prime \right|_{(fV_g)^G} = \operatorname{id}_{(fV_g)^G}$, which by lemma \ref{lamm}(b) means that we have $R_{(fV_g)}^\prime = R_{(fV_g)}$, which means that we have $R(fg) = fR(g)$.
\end{proof}

% \begin{lemma}\label{dl}
%   If $G$ is linearly reductive and $X$ is a $G$-variety, we have:
%   \begin{enumerate}[(a)]
%   \item For all $G$-stable subspaces $W \subseteq K[X]$ we have $R(W) = W^G$
%   \item For all $f \in K[X]^G$ and for all $g \in K[X]$ we have $R(fg) = fR(g)$
%   \end{enumerate}
% \end{lemma}

\begin{theorem}[Hilbert's Finiteness Theorem]\label{hilbert}
  If $G$ is linearly reductive and $V$ is a finite-dimensional rational $G$-representation, the invariant ring $K[V]^G$ is finitely generated.
\end{theorem}

\begin{proof}
  Let $I_{>0}$ denote the ideal generated by all non-constant invariants in $K[V]$.
  Since $K[V]$ is noetherian, there exist finitely many linearly independent $\{f_i\}_{i \in [r]} \subseteq K[V]$ such that $ \left( \{f_i\}_{i \in [r]} \right) = I_{i>0} $.
  These must be non-constant invariants (the zero polynomial will always be omitted).
  Claim: $K[\{f_i\}_{i \in [r]}] = K[V]^G$.
  The inclusion ``$\subseteq$'' is clear.
  To show is $\supseteq$''.
  This is equivalent to showing that for all $d \in \mathbb{N}$ we have $K[V]^G_{<d} \subseteq K[\{f_i\}_{i \in [r]}] $.
  We will show our claim via induciton over the degree $d$.
  For $g \in K[V]^G_{<1} = K$ we are already done since $K \subseteq K[\{f_i\}_{i \in [r]}]$.
  Now assume that for $d \in \mathbb{N}$ we have $K[V]^G_{<d} \subseteq K[\{f_i\}_{i \in [r]}]$.
  Let $g \in K[V]^G_{< d+1}$.
  By construction, $g \in I_{>0}$, therefore there exist $\{g_i\}_{i \in [r]} \subseteq K[V]$ such that $g = \Sigma_{i=1}^r g_i f_i$.
  Since the $f_i$ are non-constant and linearly independent, and since $\operatorname{deg} g < d+1$, we must have $ \operatorname{deg} g_i < d $.
  We now make use of the Reynolds Operator:
  \begin{equation}
      g = R(g)
      =  R \left( \sum_{i=1}^r g_i f_i \right)
      = \sum_{i=1}^r R( g_i) f_i
  \end{equation}
  Since $R$ maps $K[V]_{<d}$ to $K[V]^G_{<d}$, we have $R(g_i) \in K[V]^G_{<d} \subseteq K[\{f_i\}_{i \in [r]}]$ by our induction hypothesis.
  This finally implies $ g \in K[\{f_i\}_{i \in [r]}]$, which concludes our proof:
  We have $K[V]^G = K[\{f_i\}_{i \in [r]}]$ which means that $K[V]^G$ is finitely generated, which was to show.
\end{proof}

\begin{lemma}\label{bloblo}
  See \cite[2.2.8]{DK15}

  Let $K$ be an algebraically closed filed and $V$ and $W$ be finite rational representations of a linearly reductuve group $G$.
  For a surjective $G$-equivariant linear map $A \colon V \twoheadrightarrow W$ we then have $A(V^G) = W^G$
\end{lemma}

\begin{proof}
  Let $A \colon V \twoheadrightarrow W$ be a surjective $G$-equivariant linear map.
  Let $Z := \operatorname{ker}A$, which is a subrepresentation of $V$ since $A$ is $G$-equivariant.
  Since $G$ is linearly reductive and since $K$ is algebraically closed, we can apply theorem \ref{decomp} and get a subrepresentation $W^\prime$ of $V$ such that $V = Z \oplus W^\prime$.
  This yields an isomorphism of representations $\left. A \right|_{W^\prime} \colon W^\prime \xrightarrow{\sim} W$, which implies $A(V^G) = A(Z^G + {W^\prime}^G) = A({W^\prime}^G) = A(W^\prime)^G = W^G$.
\end{proof}

\begin{lemma}\label{emb}
  See \cite[A1.9]{DK15}.
  
  Let $X$ be an affine $G$-variety.
  Then there exists a finite rational representation $V$ of $G$ and a $G$-equivariant embedding $i \colon X \hookrightarrow V$.
\end{lemma}

% \begin{corollary}
%   For an affine $G$-variety $X$ and for $d \geq 0$, $K[X]_d$ is a $G$-stable subspace of $K[X]$.
% \end{corollary}

% \begin{proof}
%   This follows immediately from the fact that $i^\ast \colon K[V] \twoheadrightarrow K[X]$ is a surjective $G$-invariant ring homomorphism.
% \end{proof}

\begin{proof}
  We choose generators $\{f_i\}_{i \in [r]}$ of $K[X]$ and define $W := \sum_{i \in [r]} V_{f_i}$, which is a finite-dimensional $G$-stable subspace of $K[X]$ containing $\{f_i\}_{i \in [r]}$.
  This gives us the $G$-invariant morphism of affine varieties $i \colon X \longrightarrow W^\ast$, $x \mapsto (w \mapsto w(x))$.
  This is injective, since $W$ contains a generating set of $K[X]$, which means that $i$ is an embedding.
\end{proof}

\begin{dexample}[The Domain Of The Cross Ratio]
  We would like to look at four distinct points in the projective line over an algebraically closed field $K$.
  Since the projective line isn't an affine variety, we will look at points in $K^2$ to make the situation affine, which will make some things different from the setting in projective geometry.
  Consider $(K^2)^4$ and with the coordinate functions $\{(X_i)_k\}_{i \in [4], k \in [2]}$. \textbf{bars over coordfcts?}
  (We write $X_i = \binom{(X_i)_1}{(X_i)_2}$ for $i \in [4]$.)
  Define $q := \prod_{i,j \in [r], i<j} \operatorname{det}(X_i,X_j)$.
  As described in \ref{rabbi}, we have an affine variety
  \begin{equation}
    % \begin{aligned}
    %   &X&:=&\{\, (a,b,c,d) \in (K^2)^4 \mid (b_1c_2 - b_2c_1)(d_1a_2 - d_2a_1) \neq 0 \,\}\\
    %   &&=& \{\, (a,b,c,d) \in (K^2)^4 \mid \operatorname{det}(b,c)\operatorname{det}(d,a) \neq 0 \,\}
    % \end{aligned}
    X := \{\, (x_1,x_2,x_3,x_4) \in (K^2)^4 \mid q(x_1,x_2,x_3,x_4) \neq 0 \,\}
  \end{equation}
  with the coordinate ring $K[X] = K[\{(X_i)_k\}_{i \in [4], k \in [2]},q^{-1}]$.
  Nothe that $q(x_1,x_2,x_3,x_4) \neq 0$ is equivalent to saying that for $i\neq j$ we have $x_i \notin \operatorname{span}{x_j}$, or rather in projective terms $[x_i] \neq [x_j]$.
  Now consider the action of $\operatorname{GL}_2$ on $X$ via pointwise application.
  This definition of the action already implies the inclusion $i \colon X \hookrightarrow K \times (K^2)^4$ of the Rabinovich trick as described in proposition \ref{rabbi}:
  If we define an action on $K \times (K^2)^4$ by $(\sigma,(z,x_1,x_2,x_3,x_4)) \mapsto (\operatorname{det}(\sigma)^{-6}z,\sigma x_1,\sigma x_2,\sigma x_3,\sigma x_4)$, it should be clear that $i$ is a $G$-equivariant inclusion map of affine $G$-varieties.
\end{dexample}

\begin{lemma}\label{foremb}
  See \cite[2.2.9]{DK15}.
  
  Assume that $K$ is algebraically closed and that $G$ is linearly reductive.
  Let $X$ be an affine $G$-variety, $V$ a finite rational representation of $G$ and $i \colon X \hookrightarrow V$ a $G$-equivariant embedding.
  The surjective $G$-equivariant ring homomorphism $i^\ast \colon K[V] \twoheadrightarrow K[X]$ then has the property $i^\ast (K[V]^G) = K[X]^G$.
\end{lemma}

\begin{proof}
  We obviously have $i^\ast(K[X]^G) \subseteq K[V]^G$.
  let $f \in K[V]^G$.
  Since $i^\ast$ is surjective, there are $ \{g_i\}_{i \in [r]} \subseteq K[V]$ such that $\{i^\ast(g_i)\}_{i \in [r]}$ form a basis of $V_f$.
  Since $i^\ast$ is $G$-equivariant, $Z := \operatorname{span}\{g_i\}$ is a $G$-stable subspace of $K[V]$ with $i^\ast(Z) = V_f$.
  By lemma \ref{bloblo} we have $i^\ast (Z^G) = V_f^G$, in particular $f \in i^\ast (Z^G) \subseteq i^\ast(K[V]^G)$.
  This concludes $i^\ast(K[V]^G) = K[X]^G$.
\end{proof}

\begin{theorem}[Hilbert's Finiteness Theorem For Affine Varieties]
  If $K$ is an algebraically closed field, $G$ a linearly reductive group and $X$ is an affine $G$-variety, $K[X]^G$ is finitely generated.
\end{theorem}

\begin{proof}
  By lemma \ref{emb}, there exists a rational representation $V$ of $G$ and and an embedding $i \colon X \hookrightarrow V$.
  By theorem \ref{hilbert} there exist $ \{f_i\}_{i \in [r]} \subseteq K[V]$ such that $K[V]^G = K[\{f_i\}_{i \in [r]}]$.
  By lemma \ref{foremb} we have $K[X]^G = i^\ast (K[V]^G) = i^\ast (K[\{f_i\}_{i \in [r]}]) = K[\{i^\ast\}_{i \in [r]}]$, which shows that $K[X]^G$ is finitely generated.
\end{proof}

\subsection{The Reynolds Operator Of A Group}

% \begin{definition}[linearly reductive]
%   A group $G$ is called \textbf{linearly reductive} iff there exists a Reynolds operator $ R_G \colon K\lbrack G \rbrack \longrightarrow K\lbrack G \rbrack^G = K $ for the regular action $ G \times G \longrightarrow G $ by left multiplication.
% \end{definition}

% \begin{remark}
%   We could have also defined linear reductive groups as such, for which every regular action has a Reynolds Operator.
%   We will prove in proposition \ref{ro} that this is in fact equivalent.
% \end{remark} 

In theorem \ref{equiv} we have learned about four characterizations of linearly reductive groups, but for a given linear algebraic group, it is still hard to concretely show that it is linearly reductive.
We will soon learn about a fifth way to characterize linearly reductive groups, which will motivate the main theme of my work:  Cayley's $\Omega$-process.

\begin{definition}[Reynolds Operator Of A Group]
  Let $G$ be a linear algebraic group.
  The multiplication $m \colon G\times G \longrightarrow G$ makes $G$ a $G$-variety.
  Assume that there exists a Reynolds operator $R_G \colon K[G] \twoheadrightarrow K[G]^G = K$ for this action.
  We call $R_G$ the \textit{Reynolds operator of $G$}.
\end{definition}

\begin{theorem}\label{ro}
  Let $G$ be linearly reductive, and let $G$ act regularly on an affine variety $X$, which induces a rational $G$-action on $K[X]$ as described in definition \ref{funrep}.
  Then, the following the map
  \begin{align}
    R \colon K[X] \longrightarrow K[X]^G && f \mapsto R_G \cdot f
  \end{align}
  defines a Reynolds operator.
\end{theorem}

\begin{proof}
  As per our construction from definition \ref{da}, the linearity of this map should be clear.
  Let $f \in K[X]$, $\sigma \in G$ and $x \in X$.
  Write $\bar{\mu}^\prime (f) = \Sigma_i p_i \otimes g_i \in K[G] \otimes K[X] $.
  Now we compute:
  \begin{equation}
    \begin{aligned}
      &\sigma. \left( R_G \cdot f \right) (x)
      &=& \left( R_G \cdot f \right) (\sigma^{-1}.x)\\
      &&=& \Sigma_i R_G \left( p_i \right)  \sigma.g_i \left(  x \right) \\
      &&=& \Sigma_i R_G (\sigma.p_i)  \sigma.g_i (x)\\
      &&=& \left( R_G \otimes \operatorname{id} \right) \left( \Sigma_i \sigma.p_i \otimes \sigma.g_i \right) (x)\\
      &&=& (R_G \otimes \operatorname{id}) (\bar{\mu}^\prime (f)) (x)
      &=& (R_G \cdot f) (x)
    \end{aligned}
  \end{equation}
  We made use of the $G$-invariance of $R_G$ and proposition \ref{rara}.
  This means that we have $R(K[X]) \subseteq K[X]^G$.
  If $f \in K[V]^G$, we have $\bar{\mu}^\prime (f) = 1 \otimes f$, therefore $R(f) = R_G \cdot f = R_G (1)f = f$.
  This gives us $\left. R \right|_{K[X]^G} = \operatorname{id}_{K[X]^G}$, showing that $R$ is a projection of $K[X]$ onto $K[X]^G$.\\
  Now let $\sigma \in G$ and let $ f \in K[X]$, and let $\bar{\mu}^\prime(f) = \Sigma_{i=1}^r p_i \otimes g_i \subseteq K[G] \otimes K[X]$.
  We then have
  \begin{equation}
    \begin{aligned}
      &R_G \cdot \sigma.f
      &=& (R_G \otimes \operatorname{id}) \left(\bar{\mu}^\prime(\sigma.f)\right)\\
      &&=& (R_G \otimes \operatorname{id}) \left(\sum_{i=1}^r \sigma \dot{\phantom{.}}p_i \otimes g_i \right)\\
      &&=& \sum_{i=1}^r R_G(\sigma\dot{\phantom{.}}p_i)g_i\\
      &&=& \sum_{i=1}^r R_G(p_i)g_i
      &=& R_G \cdot f
    \end{aligned}
  \end{equation}
We made use of proposition \ref{roro} and 
\end{proof}

\begin{corollary}
  $G$ is linearly reductive if and only if the Reynolds operator of $G$ exists.
  The Reynolds operator of $G$ is unique.
\end{corollary}
%%% Local Variables:
%%% mode: latex
%%% TeX-master: "bachelorarbeit"
%%% End:


\section{Cayley's $\Omega$-Process}

We want to express the Reynolds Operator in a concrete way.
For the Group $\operatorname{GL}_n$, we can explicitly formulate it with the help of Cayley's $\Omega$-Process.

First, how is $\operatorname{GL}_n$ an affine variety?
Consider $ K^{n \times n} \times K $, and its coordinate ring $ K\left\lbrack \lbrace Z_{i,j} \rbrace_{i,j \in \lceil n \rceil} , D \right\rbrack$.
Now define $ I := \left( \left( \sum_{\sigma \in S_n} \operatorname{sgn} \left( \sigma \right) \prod_{ i \in \lceil n \rceil } Z_{i, \sigma \left( i \right) } \right) \cdot D - 1 \right)  = \left( \operatorname{det} \left( Z \right) \cdot D - 1 \right) $ , where $ Z := \left\lbrack Z_{i,j} \right\rbrack_{i,j \in \lceil n \rceil} $.
Then $ V \left( I \right) = \lbrace \, \left( z , d \right) \mid z \in \operatorname{GL}_n , d = \operatorname{det} \left( z \right)^{-1} \, \rbrace $.
Equipped with the componentwise multiplication ($\operatorname{GL}_n$ and $K \setminus \lbrace 0 \rbrace$, respectively), this is a linear algebraic group isomorphic to $\operatorname{GL}_n$.
The coordinate ring $K\left\lbrack \operatorname{GL}_n\right\rbrack$ is isomorphic to $K \left\lbrack \lbrace Z_{i,j} \rbrace_{i,j \in \lceil n \rceil} , \operatorname{det} \left( Z \right)^{-1} \right\rbrack \subseteq K \left( \lbrace Z_{i,j} \rbrace_{i,j \in \lceil n \rceil} \right) $, and we will write it as such from now on.
\begin{definition}[Cayley's $\Omega$-Process]
  We call
  \begin{equation}
    \Omega := \sum_{\sigma \in S_n} \operatorname{sgn} \left( \sigma \right) \prod_{ i \in \lceil n \rceil } \frac{\partial}{\partial z_{i , \sigma \left( i \right)}}
  \end{equation}
\end{definition}
the \textbf{Cayley's $\Omega$-Process}.
It can also be thought of as $ \Omega = \operatorname{det} \left( \frac{\partial}{\partial Z} \right) $, where $\frac{\partial}{\partial Z} := \left\lbrack \frac{\partial}{\partial z_{i,j}} \right\rbrack_{i,j \in \lceil n \rceil} $.

\begin{lemma}
  \begin{equation}
    \left( \operatorname{det} \left( Z \right) ^{-1} \cdot \otimes \Omega \right) \circ m^\ast
    = m^\ast \circ \Omega
    = \left( \Omega \otimes \operatorname{det} \left( Z \right) ^{-1} \cdot \right) \circ m^\ast
  \end{equation}
  where I write ``$ p \cdot $'' for the operation \textit{multiply with $ p $} for a polynomial $ p \in K \left\lbrack \operatorname{GL}_n \right\rbrack $ (but don't worry, this is the only time I will make use of this notation).
\end{lemma}
a
\begin{proof}
  Here, we will follow the same convention as described in chapter 3: The ``left'' and ``right'' inputs of $ m $ will be represented by $ X = \left\lbrack X_{i,j} \right\rbrack_{i,j \in \lceil n \rceil} $ and $ Y = \left\lbrack Y_{i,j} \right\rbrack_{i,j \in \lceil n \rceil} $ in the occuring polynomials respectively, and the outputs $ m = \left\lbrack m_{i,j} \right\rbrack_{i,j \in \lceil n \rceil} $ are indexed the same as the inputs of the polynomials in $ Z_{1,1} , Z_{1,2} , \ldots Z_{n,n} $.\\
  Let $f \in K \left\lbrack \operatorname{GL}_n \right\rbrack $.
  Then $ f \circ m \in K \left\lbrack \left\{ X_{i,j} \right\}_{i,j \in \lceil n \rceil} , \operatorname{det} \left( X \right)^{-1} , \left\{ Y_{i,j} \right\}_{i,j \in \lceil n \rceil} , \operatorname{det} \left( Y \right)^{-1} \right\rbrack $.
  For fixed $i,j \in \lceil n \rceil $ we have
  \begin{equation}
    \begin{aligned}
    \left( \operatorname{id} \otimes \frac{\partial}{\partial Z_{i,j}} \right) \left( m^\ast \left( f \right) \right) 
    = t \left( \frac{\partial}{\partial Y_{i,j}} \left( f \circ m \right) \right) 
    = t \left( \sum_{k,l \in \lceil n \rceil} \left( \left( \frac{\partial}{\partial Z_{k,l}} f \right) \circ m \right) \cdot \frac{\partial}{\partial Y_{i,j}} m_{k,l} \right) \\
    = t \left( \sum_{k = 1}^n \left( \left( \frac{\partial}{\partial Z_{k,j}} f \right) \circ m \right) \cdot X_{k,i} \right) 
    = \sum_{k=1}^n \left( Z_{k,i} \cdot \otimes \operatorname{id} \right) \left( m^\ast \left( \frac{\partial}{\partial Z_{k,j}} f \right) \right)
    \end{aligned}
  \end{equation}
  Note the use of $ t $ as described in chapter 3 to aid in rephrasing terms.
  Successively applying this yields
  \begin{equation}
    \begin{aligned}
      \left( \operatorname{id} \otimes \Omega \right) \left( m^\ast \left( f \right) \right)
      = \sum_{\sigma \in S_n} \operatorname{sgn} \left( \sigma \right) \left( \operatorname{id} \otimes \prod_{i=1}^n \frac{\partial}{\partial Z_{i,\sigma \left( i \right)}} \right) \left( m^\ast \left( f \right) \right) \\
      = \sum_{\sigma \in S_n} \operatorname{sgn} \left( \sigma \right) \sum_{k \in \lceil n \rceil^n} \left( \prod_{i=1}^n Z_{k(i),i} \cdot \otimes \operatorname{id} \right) \left( m^\ast \left( \prod_{j=1}^n \frac{\partial}{\partial Z_{k(j),\sigma (j)}} f \right) \right) \\
      = \sum_{k \in \lceil n \rceil^n} \left( \prod_{i=1}^n Z_{k(i),i} \cdot \otimes \operatorname{id} \right) \left( m^\ast \left( \sum_{\sigma \in S_n} \operatorname{sgn} \left( \sigma \right) \prod_{j=1}^n \frac{\partial}{\partial Z_{k(j),\sigma (j)}} f \right) \right) \\
      = \sum_{k \in S_n} \left( \prod_{i=1}^n Z_{k(i),i} \cdot \otimes \operatorname{id} \right) \left( m^\ast \left( \sum_{\sigma \in S_n} \operatorname{sgn} \left( \sigma \right) \prod_{j=1}^n \frac{\partial}{\partial Z_{k(j),\sigma (j)}} f \right) \right) \\
      = \sum_{k \in S_n} \left( \prod_{i=1}^n Z_{k(i),i} \cdot \otimes \operatorname{id} \right) \left( m^\ast \left( \operatorname{sgn} (k) \Omega (f) \right) \right)
      = \left( \operatorname{det} (Z) \cdot \otimes \operatorname{id} \right) \left( m^\ast \left( \Omega (f) \right) \right)
    \end{aligned}
  \end{equation}
  This immediately shows the first equality, and the second equality is proven analogously.
\end{proof}

\begin{lemma}
  For $p \in \mathbb{N} $, $ c_{p,n} := \Omega^p \left( \operatorname{det} (Z)^p \right) = \operatorname{det} \left( \frac{\partial}{\partial Z}\right)^p \left( \operatorname{det} (Z)^p \right)$ is a nonnegative integer.
\end{lemma}

\begin{proof}
  Write $ \operatorname{det} (Z)^p = \Sigma_i a_i m_i \left( \left\{ Z_{k,l} \right\}_{k,l \in \lceil n \rceil} \right) $, where $a_i \in K$ and $m_i$ are (monic) monomials.
  Then
  \begin{equation}
    \Omega^p \left( \operatorname{det} (Z)^p \right)
    = \Sigma_i a_i m_i \left( \left\{ \frac{\partial}{\partial Z_{k,l}} \right\}_{k,l \in \lceil n \rceil} \right) \left( \Sigma_j a_j m_j \left( \left\{ Z_{k,l} \right\}_{k,l \in \lceil n \rceil} \right) \right)
\end{equation}
Notice that $ m_i \left( \left\{ \frac{\partial}{\partial Z_{k,l}} \right\}_{k,l \in \lceil n \rceil} \right) \left( m_j \left( \left\{ Z_{k,l} \right\}_{k,l \in \lceil n \rceil} \right) \right) $ is zero for $ i \neq j $ and a strictly positive integer for $ i = j $. Therefore
\begin{equation}
  c_{p,n}
  = \sum_i a_i^2 m_i \left( \left\{ \frac{\partial}{\partial Z_{k,l}} \right\}_{k,l \in \lceil n \rceil} \right) \left( m_i \left( \left\{ Z_{k,l} \right\}_{k,l \in \lceil n \rceil} \right) \right) \in \mathbb{N}_{>0}
\end{equation}
in particular $c_{p,n} \neq 0$.
\end{proof}
Now, finally, we have the tools to see the following way of expressing the Reynolds Operator.
\begin{theorem}
  For $ p \in \mathbb{N} $ and $ \tilde{f} \in K \left\lbrack \left\{ Z_{i,j} \right\}_{k,l \in \lceil n \rceil} \right\rbrack_{pn} $, define for $ f = \frac{\tilde{f}}{\operatorname{det}(Z)^p}$:
  \begin{equation}
    R \left( f \right) := \frac{\Omega^p \tilde{f}}{c_{p,n}}
  \end{equation}
  The linear extension of this (mapping anything else in $K \left\lbrack \operatorname{GL}_n \right\rbrack$ to zero), defines the Reynolds Operator $R_{\operatorname{GL}_n}$.
\end{theorem}

\begin{proof}
  First, check that this is well defined:
  For any such term, expanding the fraction by $ \operatorname{det} (Z)^q $ will yield the same result.
  Also, $\Omega^p$ is linear for any $p \in \mathbb{N}$.
  Now we show that $R$ is $\operatorname{GL}_n$-invariant.
  First, I will introduce a notation:
  For $f \in K \left\lbrack \operatorname{GL}_n \right\rbrack $ and $\alpha \in \operatorname{GL}_n$, define $ \alpha \dot{\phantom{.}} f := \left( x \mapsto f \left( x \alpha^{-1} \right) \right) $.
  This is \textit{not} an action, but a right action (normal actions should be called ``left actions'').
  Let $p \in \mathbb{N}$, $ \tilde{f} \in K \left\lbrack \operatorname{GL}_n \right\rbrack_{pn} $ and $ f := \frac{\tilde{f}}{\operatorname{det}(Z)^p} $.
  For $ \beta , \gamma \in \operatorname{GL}_n $, we notice
  \begin{equation}
    \begin{aligned}
      R \left( \beta . f \right) (\gamma)
      = R \left( \frac{ \operatorname{det} (\beta)^p \cdot \beta . \tilde{f}}{\operatorname{det} (Z)^p} \right) (\gamma)
      = \frac{ \operatorname{det} (\beta)^p \cdot \Omega^p \left( \beta . \tilde{f} \right) (\gamma) }{ c_{p,n} } \\
      = \frac{1}{c_{p,n}} \cdot \left( \epsilon_{\beta^{-1}} \otimes \epsilon_\gamma \right) \left( \left( \left( \operatorname{det}(Z)^{-p} \cdot \otimes \Omega^p \right) \circ m^\ast \right) \left( \tilde{f} \right) \right) \\
      = \frac{1}{c_{p,n}} \cdot \left( \epsilon_{\beta^{-1}} \otimes \epsilon_\gamma \right) \left( \left( \left( \Omega^p \otimes \operatorname{det}(Z)^{-p} \cdot \right) \circ m^\ast \right) \left( \tilde{f} \right) \right) \\
      = \frac{ \Omega^p \left( \gamma^{-1} \dot{\phantom{.}} \tilde{f} \right) (\beta^{-1}) \cdot \operatorname{det} \left (\gamma^{-1} \right)^p }{ c_{p,n} }
      = R \left( \frac{ \gamma^{-1} \dot{\phantom{.}} \tilde{f} \cdot \operatorname{det} \left( \gamma^{-1} \right)^p }{\operatorname{det} (Z)^p} \right) \left( \beta^{-1} \right) \\
      = R \left( \gamma^{-1} \dot{\phantom{.}} f \right) \left( \beta^{-1} \right)
    \end{aligned}
  \end{equation}
  Since each $ \frac {\partial}{ \partial Z_{i,j} } $ lowers the degree of a monomial by one or maps it to zero, $R$ maps to $K$, and therefore for $ \delta \in \operatorname{GL}_n $ and $ g \in K \left\lbrack \operatorname{GL}_n \right\rbrack $ we have $ R(g)(\delta) = R(g) \in K $.
  We then get for all $ \alpha \in \operatorname{GL}_n $
  \begin{equation}
    \begin{aligned}
      R \left( \alpha . f \right)
      = R \left( \alpha . f \right) \left( I_n \right)
      = R \left( I_n^{-1} \dot{\phantom{.}} f \right) \left( \alpha^{-1} \right) \\
      = R \left( I_n^{-1} \dot{\phantom{.}} f \right)
      = R \left( I_n \dot{\phantom{.}} f \right) 
      = R (f)
  \end{aligned}
  \end{equation}
  which shows the $\operatorname{GL}_n$-invariance.
  Finally, the definition immediately gives us that $R$ restricted to $K$ is the identity.  As mentioned in \ref{unique}, the uniqueness of the Reynolds Operator implies $ R = R_{\operatorname{GL}_n} $.
\end{proof}

Now we will look at the Reynolds Operator $R_{\operatorname{SL}_n}$.

\begin{corollary}
  With the convention of $ K \left\lbrack \operatorname{GL}_n \right\rbrack = K \left\lbrack \left\{ Z_{k,l} \right\}_{k,l \in \lceil n \rceil} , \operatorname{det} (Z) ^{-1} \right\rbrack $, view $ K \left\lbrack \operatorname{SL}_n \right\rbrack = K \left\lbrack \operatorname{GL}_n \right\rbrack / I $ where $ I = \left( \operatorname{det} (Z) -1 \right) $.
  Now, for $ p \in \mathbb{N} $ and $ \tilde{f} \in K \left\lbrack \left\{ Z_{i,j} \right\}_{k,l \in \lceil n \rceil} \right\rbrack_{pn} $ define for $ f = \frac{\tilde{f}}{\operatorname{det}(Z)^p} $:
  \begin{equation}
    R \left( f \right) := \frac{\Omega^p \tilde{f}}{c_{p,n}} + I
  \end{equation}
  The linear extension of this (mapping anything else in $K \left\lbrack \operatorname{SL}_n \right\rbrack$ to zero), defines the Reynolds Operator $R_{\operatorname{SL}_n}$.
\end{corollary}

\begin{proof}
  First, we will show $ K \left\lbrack \operatorname{GL}_n \right\rbrack ^{\operatorname{SL}_n} = K \left\lbrack \operatorname{det} (Z) , \operatorname{det} (Z) ^{-1} \right\rbrack $ (action by left multiplication).
  Let $g \in K \left\lbrack \operatorname{GL}_n \right\rbrack ^{ \operatorname{SL}_n }$, and let $ \alpha \in \operatorname{GL}_n $.
  Note that $ \frac{1}{n \operatorname{det} (\alpha)} \alpha \in \operatorname{SL}_n $.
  Define $ h := \beta \mapsto g \left( n \operatorname{det} (\beta) I_n \right) \in K \left\lbrack \operatorname{det} (Z) , \operatorname{det} (Z) ^{-1} \right\rbrack $.
  Now
  \begin{equation}
    \begin{aligned}
      g ( \alpha )
      = \left( \frac{1}{n \operatorname{det} (\alpha)} \alpha \right) . g \, (\alpha)
      = g \left( n \operatorname{det} (\alpha) \alpha ^{-1} \alpha \right) \\
      = g \left( n \operatorname{det} (\alpha) I_n \right)
      = h (\alpha)
    \end{aligned}
  \end{equation}
  Therefore $ g = h \in K \left\lbrack \operatorname{det} (Z) , \operatorname{det} (Z) ^{-1} \right\rbrack $.
  Conversely it is easy to see that $ K \left\lbrack \operatorname{det} (Z) , \operatorname{det} (Z) ^{-1} \right\rbrack \subseteq K \left\lbrack \operatorname{GL}_n \right\rbrack ^{\operatorname{SL}_n } $.
\end{proof}

%%% Local Variables:
%%% mode: latex
%%% TeX-master: "roughdraft"
%%% End:


\section{Further Discussion}

\subsection{A Complete Algorithm for Retrieving Generators of the Invariant Ring}
Our motivation for having a construction of the Reynolds operator was to not only see that $\operatorname{GL}_n$ is linearly reductive, but also to yield some invariants.
It would also be very helpful if we could somehow produce a generating set for the invariant ring.  \\
In example \ref{theex}, we saw that applying the Reynolds operator to any polynomial does not always result in retrieving a nonzero invariant.
It suggests that we somehow need to find the ``correct'' polynomials to apply the Reynolds operator to.
The following proposition (see \cite[prop. 4.1.1]{DK15}) gives us exactly that.
\begin{proposition}
  Let $V$ be a rational $G$-representation where $G$ is linearly reductive, and let $I_{>0}$ denote the ideal generated by all non-constant invariants.
  If $I_{>0} = \left(\{f_i\}_{i \in[r]}\right)$ for some homogeneous polynomials $\{f_i\}_{i\in[r]} \subseteq K[V]$, we have $I_{>0} = \left(\{R(f_i)\}_{i\in[r]}\right)$ and $K[V]^G = K\left[\{R(f_i)\}_{i\in[r]}\right]$.
\end{proposition}
In the proof of Hilbert's finiteness theorem (\ref{hilbert}), we made use of the existence of a finite set of invariants generating $I_{>0}$, which was non-constructively given.
The previous proposition looks helpful since we have a construction for the Reynolds operator for $G=\operatorname{GL}_n$ via Cayley's $\Omega$-process, but the problem still remains that we need to have a finite set of homogeneous polynomials generating $I_{>0}$, whose existence is here also non-constructively given.  \\
It is in fact possible to compute them with Groebner bases, which is extensively described in \cite[Algorithm 4.1.9]{DK15}.
This gives us a complete algorithm that takes as its input all of the information necessary to describe our rational representation, which can all be given in terms of polynomials, and outputs a list of generators of the invariant ring.

\subsection{Cross Ratio}
In examples \ref{domcr} and \ref{crinv} we discussed the cross ratio.
Our setting was affine and in $K^2$, which makes our results different from the projective setting, where there are not very many other polynomials other than the cross ratio.  \\
Using the same conventions and definitions as in the aforementioned examples, we can define the projective cross ratio:
\begin{equation}
  \begin{aligned}
    \operatorname{cr} \colon&&Y&\longrightarrow K \\
    &&([x_1],[x_2],[x_3],[x_4]) &\longmapsto \frac{\operatorname{det}(x_1,x_2)\operatorname{det}(x_3,x_4)}{\operatorname{det}(x_2,x_3)\operatorname{det}(x_4,x_1)}
  \end{aligned}
\end{equation}
where $Y \subseteq P(K^2)^4$ is the set of all pairwise distinct four-tuples of points in $P(K^2)$.
It should be clear that this is well-defined.
The action of $\operatorname{GL}_2$ on $X$ induces an action of $\operatorname{PGL}_n$ on $Y$.  \\
Let $f \colon Y \longrightarrow K$ be an invariant regular function.
If $X_1,X_2,X_3,Y_1,Y_2,Y_3 \in P(K^2)$ with $X_1$, $X_2$ and $X_3$ pairwise distinct and $Y_1$, $Y_2$ and $Y_3$ pairwise distinct, then an important theorem in projective geometry is that there exists a (unique) projective transformation $ \rho \in \operatorname{PGL}(K^2)$ such that $\rho (X_1) = Y_1$, $\rho(X_2) = Y_2$ and $\rho(X_3) = Y_3$ (see \cite[prop 5.6]{Aud03}).
  Let $A,B,C,D \in Y$, which implies that $B,C,D$ are pairwise distinct.
  For $x \in K $ we define $x_P := \left[\binom{x}{1}\right]$ and $\infty_P := \left[\binom{1}{0}\right]$.
  There then exists a $\rho \in \operatorname{PGL}(K^2)$ such that $\rho (B) = 0_P$, $\rho (C) = 1_P$ and $\rho (D) = \infty_P$.
  Since $A$ is distinct from $D$ we know $\rho (A) \neq \infty_P$, and therefore there exists some $a \in K$ such that $\rho(A) = [\binom{a}{1}]$.
  We then compute
  \begin{equation}
    \begin{aligned}
      &\rho (A)
      &&= \left[\binom{a}{1}\right]\\
      &&&= \operatorname{cr}\left(\left[\binom{a}{1}\right],\left[\binom{0}{1}\right],\left[\binom{1}{1}\right],\left[\binom{1}{0}\right]\right)_P\\
      &&&= \operatorname{cr}(\rho(A),\rho(B),\rho(C),\rho(D))_P &&= \operatorname{cr}(A,B,C,D)_P
    \end{aligned}
  \end{equation}
  We then have
  \begin{equation}
    \begin{aligned}
      &f(A,B,C,D)
      &&= \rho^{-1}.f (A,B,C,D)\\
      &&&= f(\rho(A),\rho(B),\rho(C),\rho(D))\\
      &&&= f(\operatorname{cr}(A,B,C,D)_P,0_P,1_P,\infty_P)
    \end{aligned}
  \end{equation}
  This shows that in the projective setting, there don't exist many more invariants than the cross ratio.  \\
  In our affine setting, it suggests that we can transfer the idea, which would mean that $K[\operatorname{cr},p(\binom{\operatorname{cr}}{1},\binom{0}{1},\binom{1}{1},\binom{1}{0})^{-1}] = K[\operatorname{cr},(\operatorname{cr}(\operatorname{cr}-1))^{-1}] = K[\operatorname{cr},\operatorname{cr}(X_1,X_3,X_4,X_2)]$ are all invariants.
  This is not true though, since for instance $\frac{\operatorname{det}(X_1,X_2)}{\operatorname{det}(X_3,X_4)} \in K[X]^{\operatorname{GL}_2}$ is an invariant not included in $K[\operatorname{cr},\operatorname{cr}(X_1,X_3,X_4,X_2)]$.
  
%%% Local Variables:
%%% mode: latex
%%% TeX-master: "roughdraft"
%%% End:

% \section{Test}
% 
% This part of my master file is for testing my environment.
%
% \begin{theorem}
  Hero
\end{theorem}

\begin{theorem}
  haroooo
\end{theorem}

\begin{remark}
  whaaaaa
\end{remark}

\begin{remark}
  okaydoakes!
\end{remark}

\begin{definition}
  shiawia
\end{definition}

\begin{theorem}
  yesss
\end{theorem}

\begin{corollary}
  okok whatevs mcshmeves
\end{corollary}

\begin{cremark}
  you got dis?
\end{cremark}

%%% Local Variables:
%%% mode: latex
%%% TeX-master: "roughdraft"
%%% End:


\bibliography{roughdraft}

\end{document}
%%% Local Variables:
%%% mode: latex
%%% TeX-master: t
%%% End:
