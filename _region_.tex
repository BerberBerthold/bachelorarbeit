\message{ !name(roughdraft.tex)}\documentclass{article}
\title{Cayley's $\Omega$-Process And The Reynolds Operator}
\author{Bert Lorke}

\usepackage{amssymb}
\usepackage{amsmath}
\usepackage{amsthm}
\usepackage{cite}
\usepackage{mathtools}

\newtheoremstyle{prrt}
  {\topsep}
  {\topsep}
  {}
  {0pt}
  {\bfseries}
  {\\}
  { }
  {\thmname{#1}\thmnumber{ #2}\thmnote{: #3}}

\theoremstyle{prrt}
\newtheorem{theorem}{Theorem}[section]
\newtheorem{definition}[theorem]{Definition}
\newtheorem{lemma}[theorem]{Lemma}
\newtheorem{proposition}[theorem]{Proposition}
\newtheorem{corollary}{Corollary}[theorem]
\newtheorem{remark}{Remark}[theorem]
\newtheorem{cremark}{Remark}[corollary]

\setlength{\parindent}{0pt}

\begin{document}

\message{ !name(prozess.tex) !offset(-29) }
 is isomorphic to $K \[ \lbrace X \rbrace_{i,j \in \lceil n \rceil} , det \left( \lbrace X \rbrace_{i,j \in \lceil n \rceil} \right)^{-1} \] \subseteq K \left( \lbrace X \rbrace_{i,j \in \lceil n \rceil} \right) $
\begin{definition}[Cayley's $\Omega$-Process]
  We call
  
\message{ !name(roughdraft.tex) !offset(43) }

\end{document}
%%% Local Variables:
%%% mode: latex
%%% TeX-master: t
%%% End:
