\message{ !name(roughdraft.tex)}\documentclass[a4paper]{article}

\usepackage{graphicx}
\usepackage{amssymb}
\usepackage{amsmath}
\usepackage{amsthm}
\usepackage{cite}
\usepackage{mathtools}
\usepackage{enumerate}
\usepackage{perpage}
\usepackage{blindtext}
% \usepackage{hyperref}

\title{Cayley's $\Omega$-Process And The Reynolds Operator}
\author{Berthold Lorke}

\newtheoremstyle{prrt}
  {\topsep}
  {\topsep}
  {}
  {0pt}
  {\bfseries}
  {}
  { }
  {\thmname{#1}\thmnumber{ #2}\thmnote{ (#3):}}

\theoremstyle{prrt}
\newtheorem{theorem}{Theorem}[section]
\newtheorem{definition}[theorem]{Definition}
\newtheorem{lemma}[theorem]{Lemma}
\newtheorem{proposition}[theorem]{Proposition}
\newtheorem{corollary}{Corollary}[theorem]
\newtheorem{remark}{Remark}[theorem]
\newtheorem{cremark}{Remark}[corollary]
\newtheorem{example}[theorem]{Example}
\newtheorem{dexample}{Example}[theorem]

% \setlength{\parindent}{0pt}

\setcounter{tocdepth}{5}

\MakePerPage{footnote}

\begin{document}

\message{ !name(prework.tex) !offset(-46) }
\subsection{Notation and general concepts}

$K$ will here always denote a field of characteristic $0$.

\vspace{0.1cm}
For us, zero is an element of the natural numbers.
Furthermore, for $n \in \mathbb{N}$ we write $[n] := \{\, m \in \mathbb{N} \mid 1 \leq m \leq n \,\}$.

\vspace{0.1cm}
For an affine variety $X$, we denote by $K[X]$ the coordinate ring of $X$.
If $\{f_i\}_{i\in[r]} \subseteq K[X]$ is a set of polynomials, we denote by $K[\{f_i\}_{i\in[r]}]$ the $K$-subalgebra of $K[X]$ generated by $\{f_i\}_{i\in[r]}$.
For a finite-dimensional vector-space $V$, we denote by $X_i$, or sometimes $Y_i$ or $Z_i$, the coordinate functions for a given (often a canonical) basis.  \\
Refer to \cite[p.~1-2]{DK15} for details.

\vspace{0.1cm}
For a set of functions in the coordinate ring $F \subseteq K[X]$ we denote by $Z(F)$ the zero set of $F$. % and for a set of points $P \subseteq X$ we denote by $V(P)$ the vanishing ideal of $P$.
For a subset of a ring $M$, $(M)$ denotes the ideal generated by $M$.

\vspace{0.1cm}
For a vector space $V$ we denote by $V^\ast$ the dual space of $V$, that is \linebreak$V^\ast = \operatorname{Hom}_K(V,K)$.

\vspace{0.1cm}
If $V$ and $W$ are vector spaces, we denote by $V\otimes W$ the tensor product of $V$ and $W$, which is equipped with the tensor product mapping $\otimes \colon V\times W \rightarrow V\otimes W$.
For a vector space $V$, we identify $K \otimes V$ with $V$ canonically via $\lambda \otimes v \leftrightarrow \lambda v$, analogously $V \otimes K$ is identified with $V$.

\vspace{0.1cm}
Let $V_1,V_2,W_1,W_2$ be vector spaces.
If $A \in \operatorname{Hom}_K(V_1,V_2)$ and $B \in \operatorname{Hom}_K(W_1,W_2)$, we define $(A\otimes B)(v\otimes w) := A(v)\otimes B(w) \in V_2 \otimes W_2$ for $v \in V_1$ and $w \in W_1$, from which by linear extension we obtain a map $A \otimes B \in \operatorname{Hom}_K(V_1\otimes W_1,V_2\otimes W_2)$.

\subsection{Concepts from algebraic geometry}


% We denote by $m$ the group multiplication of the group $G$.
% We want to view the pullback of $m$ as a map $m^\ast : K[G] \rightarrow K[G] \otimes K[G]$, which makes sense, because $m$ and $\otimes$ are associative.
% The strict pullback, which I will call $ \hat{m} $, should be a map of the type $ K[G] \rightarrow K[ G \times G] $, where $ f \mapsto f \circ m $.
% If we want to give the variables names, we can equivalently say it is a map $ \left. K[Z] \right|_G \rightarrow \left. K[X,Y] \right|_{G \times G} $, where $ Z = \lbrace Z_1 , \ldots , Z_k \rbrace $, $X$ and $Y$ analogously (here, $ m $ canonically takes its left input via $ X $ and its right input via $ Y $).
  If $g \colon X \rightarrow Y$ is a map of affine varieties, the \textbf{algebraic cohomomorphism} of $g$ is defined as $g^\ast \colon K[Y] \rightarrow K[X]$, $f\mapsto g^\ast(f) := f \circ g$.
  
  Now let $m \colon U_1 \times U_2 \rightarrow W$ be a morphism of affine varieties.
  The algebraic cohomomorphism $m^\ast$ is a morphism of the type $m^\ast \colon K[W] \rightarrow K[U_1 \times U_2]$.
  We have $ K[U_1 \times U_2] = K[\{X_k\}_{k\in[r]},\{Y_l\}_{l\in[s]}]$, where $\{X_k\}_{k\in[r]}$ and $\{Y_l\}_{l\in[s]}$ are generators of $K[U_1]$ and $K[U_2]$ respectively. 
  % If we want to give the variables names, we can equivalently say it is a map $ \left. K[Z] \right|_W \rightarrow \left. K[X,Y] \right|_{U_1 \times U_2} $, where $ Z = \lbrace Z_1 , \ldots , Z_k \rbrace $ $X$ and $Y$ analogously (here, $ m $ canonically takes its left input via $ X $ and its right input via $ Y $).  
  The $K$-algebra morphism
  % \begin{equation}
  %   \begin{aligned}
  %     t \colon K[\{X_k\}_{k\in[r]},\{Y_l\}_{l\in[s]}]
  %     & \rightarrow K[\{X_k\}_{k\in[r]}] \otimes K[\{Y_l\}_{l\in[s]}] \\
  %     \sum_i \lambda_i \prod_j X_{j}^{d_{i,j}} \prod_k Y_{k}^{e_{i,k}} &\longmapsto \sum_i \lambda_i \prod_j X_{j}^{d_{i,j}} \otimes \prod_k Y_{k}^{e_{i,k}}
  %   \end{aligned}
  % \end{equation}
  \begin{equation}
    \begin{aligned}
      K[U_1 \times U_2]
      & \longrightarrow K[U_1] \otimes K[U_2]\\
      \sum_i \lambda_i \prod_j X_{j}^{d_{i,j}} \prod_k Y_{k}^{e_{i,k}} &\longmapsto \sum_i \lambda_i \prod_j X_{j}^{d_{i,j}} \otimes \prod_k Y_{k}^{e_{i,k}}
    \end{aligned}
  \end{equation}
  is independent of the choice of generators and independent of the representatives and therefore well-defined.
  Conversely, we can look at the $K$-algebra morphism
  \begin{equation}
    \begin{aligned}
      K[U_1]\otimes K[U_2] &\longrightarrow K[U_1 \times U_2] \\
      \sum_i f_i \otimes g_i &\longmapsto \left( (u_1,u_2) \mapsto \sum_i f_i(u_1)g_i(u_2)\right) \; .
    \end{aligned}
  \end{equation}
  These morphisms are mutually inverse, thus we will identify with each other in this way.
  
This helps to formalize performing operations only on the ``left part'' or the ``right part'', as we will soon see.
This notation is found in \cite{DK15}, but other literature such as \cite{Stu08} doesn't take this approach.
To give a very simple example:
If $G$ is a linear algebraic group and $m$ is its multiplication, for $f \in  K[G] $ we would write $\operatorname{id} \otimes \frac{\partial}{\partial Z_i} (m^\ast (f))$ as in \cite{DK15}, whereas \cite{Stu08} would write $\frac{\partial}{\partial Y_i} (m^\ast (f))$, often also written as $\frac{\partial}{\partial Y_i} (f(XY))$.

\begin{definition}[Linear algebraic group]
  A group $G$ equipped with the structure of an affine variety whose group operations of the multiplication and inversion are morphisms of affine varieties is called a \textit{linear algebraic group}.
\end{definition}

Throughout this work, $G$ will always refer to a linear algebraic group, if not otherwise specified.

\begin{proposition}[Rabinowitsch trick]\label{rabbi}
  Let $V = K^n$ for some $n \in \mathbb{N} \setminus \{0\}$.
  For a polynomial $p \in K[V] = K[\{X_i\}_{i\in[n]}]$, the set $ X_p := \{\, v \in V \mid p(v) \neq 0 \,\}$ has the structure of an affine variety with the coordinate ring $K[X_p] = K[\{X_i\}_{i \in [n]}, p^{-1}]$.  \\
  (Compare to \cite{Rab30})
\end{proposition}

\begin{proof}
  The set $X_p$ is not an algebraic set itself.
  The trick (the ``Rabinowitsch-trick'') is ``adding an additional variable $X_0$''.
  We do this as follows:
  Consider the algebraic set $\tilde{X}_p := Z \left( X_0 \cdot p -1 \right) \subseteq K \times V$.
  We then notice that we have $\tilde{X}_p = \{\, (p(v)^{-1},v) \in K \times V \mid v \in X_p \,\}$.
  This means that $X_p$ corresponds to $\tilde{X}_p$ via the bijection $\Phi \colon X_p \rightarrow \tilde{X}_p$, $ v \mapsto (p(v)^{-1},v)$.
  The coordinate ring of $\tilde{X}_p$ can be written as $K[\bar{X_0}, \{\bar{X}_i\}_{i \in [n]}]$, where $\bar{X}_i = X_i \operatorname{mod} (X_0 \cdot p -1)$.
  If $x \in X_p$, we have $\bar{X}_0 (\Phi(v)) = p(x)^{-1}$ and for $ i \in [n] $ we have $\bar{X_i}(\Phi(x)) = v_i$.
  This shows our claim: $X_p$ has the structure of an affine variety with the coordinate ring $K[X] = K[\{X_i\}_{i \in [n]}, p^{-1}]$.
  % We see that $X_p$ corresponds to $\tilde{X}_p$ by noticing that $\tilde{X}_p = \{\, (1/p(v), v) \in K \times V \mid v \in X_p \,\}$, which therefore means we have the one-to-one correspondence $(1/p(v),v) \leftrightarrow v$.
  % By definition $\tilde{X}_p$ is an affine variety, and it is also easy to see that $K[X_p] \cong K[\tilde{X}_p] = K[K\times V] / (X_0 \cdot p -1)$ via $X_0 \leftrightarrow p^{-1}$, including evaluations with the above described correspondence of points, which is easy to check.
\end{proof}

\begin{dexample}[The general linear group $\operatorname{GL}_n$]
  One of the most important examples is the general linear group $\operatorname{GL}_n$.
  By the above proposition, this group is an affine variety with the coordinate ring $K[\{X_{i,j}\}_{i,j \in [n]}, \operatorname{det}^{-1}]$.
  This makes $\operatorname{GL}_n$ into a linear algebraic group.
\end{dexample}


\subsection{Concepts From Invariant Theory}

Our motivation is to look at transformations of spaces.
Concretely, we will look at transformations of vector spaces and also more generally actions on affine varieties.
These will be given by a group action of $G$, giving the variety an additional structure.
This gives rise to the notion of a $G$-variety.
For vector spaces $V$ we are interested in linear $G$-actions on $V$, from which we can make many first observations connecting to representation theory.

A given $G$-variety $X$ induces a linear $G$-action on the coordinate ring $K[X]$.
The main question of this work is how the ring of all invariants $K[X]^G$ looks like, that is asking which $f \in K[X]$ remain unchanged under the $G$-action.

\begin{definition}[Regular action, rational representation]
  Let $G$ be a linear algebraic group and $X$ an affine variety.
  We call an action $G \times X \rightarrow X$ a \textbf{regular action}, if and only if $\mu$ is a morphism of affine varieties.
  We say \textbf{$ G $ acts regularly on $ X $}, and we also call $X$ a \textbf{$G$-variety}.
  
  Let $V$ be a $G$-representation, that is, $V$ is a finite-dimensional vector space with an action $\mu \colon G \times V \rightarrow V$ that corresponds to group homomorphism $\rho_\mu \colon G \rightarrow \operatorname{GL}(V)$, meaning that we have $ \mu(\sigma,v) = \rho_{\mu}(g)(v)$ for all $\sigma \in G$ and $v \in V$.
  We call a representation $V$ (where the action $\mu$ is implied) a \textbf{rational representation of $G$}, or a \textbf{rational $G$-representation}, if and only if $\rho_\mu$ is a morphism of affine varieties.
  We also sometimes call $V$ a \textbf{rational $G$-module}.\\
  (See \cite[p.~31]{DK15})
  % We can view $V$ as an affine variety with respect to a basis.
  % We call $\mu$ a \textbf{rational representation} iff it is regular with respect to any basis.
  
  % For a finite dimensional vector space $V$ we call $D \colon G \rightarrow \operatorname{GL}(V)$ a \textbf{regular representation} iff $\mu_D := ((g,v) \mapsto D(g)(v))$ is a regular action.
  % If an action $\mu \colon G \times V \rightarrow V$ is given such that $D_\mu (g) := (v \mapsto g.v) \in \operatorname{GL}(V)$, we also call $\mu$ a regular representation (such $\mu$ and $D$ are in bijection).
\end{definition}

\begin{remark}
  The action $\mu \colon G \times V \rightarrow V$ of a rational $G$-representation $V$ is of the following form.
  
  Consider $\rho_{\mu} \colon G \rightarrow \operatorname{GL}(V)$.
  If then $ a_{i,j} : G \rightarrow K $ is the function of the \linebreak$\left( i,j \right) $-entry of $\rho_{\mu}$ (with respect to a given basis), then $ a_{i,j} \in K\lbrack G\rbrack $.
  
  Note that $\mu$ is also a morphism of affine varieties.
\end{remark}

\begin{dexample}
  If $G$ is a linear algebraic group, then the multiplication \linebreak$m \colon G \times G \rightarrow G$ defines a regular action, meaning that $G$ itself is a $G$-variety.
\end{dexample}

\begin{definition}
  If $V$ is a rational $G$-representation via $\mu \colon G \times V \rightarrow V$, we define a rational $G$-representation for the dual space $V^\ast$ via $\hat{\mu} \colon G \times V^\ast \rightarrow V^\ast$, $(\sigma,\varphi) \mapsto \sigma\cdot\varphi := (v \mapsto \varphi(\hat{\mu}(\sigma,v))= \varphi(\sigma^{-1}.v))$.
\end{definition}

% \begin{dexample}
%   A less trivial one
% \end{dexample}

% \begin{remark}
%   If $V$ is a finite-dimensional vector-space and $\mu \colon G \times V \rightarrow V$ is a representation of $G$, then $\mu$ is a rational representation if and only if it is regular with respect to a single basis.
%   Really, one can define 
% \end{remark}

\begin{definition}[Rational linear action]\label{rr}
  Let $V$ be a vector space (not necessarily finite dimensional), and $ \mu : G \times V \rightarrow V $ an action.
  We call $ \mu $ a \textbf{rational linear action} if and only if there exists a linear map $ \mu^\prime \colon V \rightarrow K[G] \otimes V $ such that $ \mu \left( \sigma , v \right) = \left( \left( \epsilon_\sigma \otimes \operatorname{id} \right) \circ \mu^\prime \right) \left(v\right) $, where $\epsilon_\sigma \colon K[G] \rightarrow K$, $p \mapsto p(\sigma)$ denotes the evaluation homomorphism for $\sigma \in G$.
  In other words, if for $v \in V$ we get $\mu^\prime(v) = \Sigma_{i=1}^rp_i \otimes v_i \in K[G]\otimes V$, we then have $\mu(\sigma,v) = \Sigma_{i=1}^r p_i(\sigma)v_i$ for all $\sigma \in G$.
  We also refer to $V$ as a \textbf{rational $G$-module}.  \\
  (Compare to \cite[A.1.7]{DK15})
\end{definition}

\begin{remark}
  From the definition, it should immediately be apparent that rational linear actions are linear and regular.
\end{remark}

We will shortly see that for finite dimensional vector spaces, the terms ``rational linear action'' and ``rational representation'' coincide, shich justifies calling them both $G$-modules.

\begin{definition}\label{funrep}
  Let $X$ be an affine $G$-variety with the regular action $\mu \colon G \times X \rightarrow X$.
  We now define an action $\bar{\mu} \colon G \times K[X] \rightarrow K[X]$ via $\bar{\mu}(\sigma,f)(x) := f(\mu(\sigma^{-1},x))$, and we write $\sigma\cdot f (x) := f( \sigma^{-1}.x )$ for  $\sigma \in G$, $f \in K[X]$ and $x \in X$.  (Compare to \cite[p.~31]{DK15})
  
  This action is obviously regular, but we easily see that it is in fact a rational linear action:
  If $\tilde{\mu} \colon G \times X \rightarrow X$ is the morphism of affine varieties defined by $ (\sigma,x)\mapsto\tilde{\mu} (\sigma,x) := \mu (\sigma^{-1},x)$, we can then define the linear map \linebreak$ \bar{\mu}^\prime := \tilde{\mu}^\ast \colon K[X] \rightarrow K[G] \otimes K[X]$ with the desired properties as described in definition \ref{rr}.
\end{definition}

The next proposition shows a practical property of $\bar{\mu}^\prime$ which will help us with some calculations later.
\begin{proposition}\label{rara}
  Let $X$ be an affine $G$-variety.
  If for $f \in K[X]$ we have $\bar{\mu}^\prime (f) = \Sigma_{i =1}^r p_i \otimes g_i $, then for every $\sigma \in G$ we have $\bar{\mu}^\prime (f) = \Sigma_{i=1}^r \sigma\cdot p_i \otimes \sigma\cdot g_i$.
\end{proposition}

\begin{proof}
  Let $\tau \in G$ and $x \in X$.
  Then
  \begin{equation}
    \begin{aligned}
      \sum_{i=1}^r \sigma\cdot p_i \otimes \sigma\cdot g_i (\tau,x)
      &= \sum_{i=1}^r p_i(\sigma^{-1}\tau) \otimes g_i (\sigma^{-1}.x)  \\
      &= \sigma^{-1}\tau\cdot f(\sigma^{-1}.x)  \\
      &= \tau\cdot f(x) \, = \, \bar{\mu}^\prime (f) (\tau,x) \quad .
    \end{aligned}
  \end{equation}
\end{proof}

\begin{definition}\label{back}
  Let $V$ be a rational representation via $\mu \colon G \times V \rightarrow V$.
  We then define an action $\hat{\mu} \colon G \times V^\ast \rightarrow V^\ast$, $ (\sigma,\varphi) \mapsto \sigma\cdot \varphi := (v \mapsto \varphi(\mu(\sigma^{-1},v)) = \varphi(\sigma^{-1}.v))$, which turns $V^\ast$ into a rational representation of $G$.
\end{definition}

\begin{definition}\label{rac}
  Let $G$ be a linear algebraic group with the multiplication $m \colon G \times G \rightarrow G$.
  For $\sigma \in G$ and for $p \in K[G]$ we define $\sigma\dot{\phantom{.}}p := (\tau \mapsto p(\tau\sigma)) \in K[G]$.
\end{definition}

Now we show another useful property of $\bar{\mu}^\prime$.

\begin{proposition}\label{roro}
  Let $X$ be an affine variety and $\mu \colon G \times X \rightarrow X$ a regular action.
  For $f \in K[X]$, if we have $\bar{\mu}^\prime (f) = \Sigma_{i = 1}^r p_i \otimes g_i$ for some $\{g_i\}_{i\in [r]}$, then for $\sigma \in G$ we get $\bar{\mu}^\prime (\sigma\cdot f) = \Sigma_{i = 1}^r \sigma \dot{\phantom{.}} p_i \otimes g_i$.
\end{proposition}

\begin{proof}
  For $f \in K[X]$ we have $ \bar{\mu}^\prime (f) = \Sigma_{i=1}^r p_i \otimes g_i$ for some $\{g_i\}_{i \in [r]}$.
  Now let $\sigma \in G$.
  Then for all $\tau \in G$ and for all $x \in X$ we have
  \begin{equation}
    \begin{aligned}
      \bar{\mu}^\prime (\sigma\cdot f) (\tau,x)
      &=((\epsilon_\tau \otimes \operatorname{id}) \circ \bar{\mu}^\prime) (\sigma\cdot f) (x)
      =(\tau\cdot (\sigma\cdot f)) (x)\\
      &=\sum_{i=1}^r p_i(\tau\sigma) g_i(x)
      =\sum_{i=1}^r \sigma\dot{\phantom{.}}p_i(\tau) g_i(x)\\
      &=(\sum_{i=1}^r \sigma \dot{\phantom{.}}p_i \otimes g_i) (\tau,x) \quad .
    \end{aligned}
  \end{equation}
\end{proof}

\begin{definition}[locally finite]
  For a vector space $V$, we call an action $\mu \colon G \times V \rightarrow V$ \textbf{locally finite}, if and only if for every $v \in V$ there exists a $G$-stable finite-dimensional vector space $U \subseteq V$ such that $v \in U$.
\end{definition}

\begin{definition}
  Let $V$ be a vector-space and $\mu \colon G \times V \rightarrow V$ an action.
  For $v \in V$ we define $V_v := \operatorname{span} G.v$.
\end{definition}

\begin{remark}
  $V_v$ is always a $G$-stable subspace of $V$.
  For any $G$-stable subspace $W \subseteq V$ we have $V_v \subseteq W$.
  Therefore, an action $\mu \colon G \times V \rightarrow V$ is locally finite if and only if $V_v$ is finite-dimensional for all $v \in V$.
\end{remark}

We will now show the connection between rational linear actions and rational representations.

\begin{proposition}\label{locfin}
  Let $V$ be a vector space.
  \begin{enumerate}[(a)]
  \item If $\mu \colon G \times V \rightarrow V$ is a rational linear action, then the action is locally finite, and every
    finite-dimensional $G$-stable subspace $W$ is a rational $G$-representation via $\left. \mu \right|_{G\times W}$.
  \item If $V$ is a rational representation via $\mu \colon G \times V \rightarrow V$, then $\mu$ is rational linear action.
  \end{enumerate}

  (Compare to \cite[A.1.8, 2.2.5]{DK15})
\end{proposition}

\begin{proof}
  (a) \; Assume that $\mu$ is a rational linear action.
  Let $v \in V$ and write $\mu^\prime (v) = \Sigma_{i=1}^l f_i \otimes v_i$. %with $f_i$ linearly independant over $K$.
  We then easily see that $V_v \subseteq \operatorname{span}\{v_i\}_{i=1}^l$, showing that the action is locally finite.
  Since $\mu^\prime$ is linear, $\mu$ is also linear, therefore we immediately get that $W$ is a rational representation via $\left. \mu \right|_{G\times W}$.

  (b) \; Let $V$ be a rational $G$-representation via $\mu \colon G \times V \rightarrow V$.
  This means that for all $\sigma \in G$ we have $\rho_\mu (\sigma) \in \operatorname{GL}(V)$.
  Let us now choose a basis $\{v_i\}_{i \in [r]}$ of $V$.
  For all $\sigma \in G$ there then exist unique $\{ \left( \rho_\mu \right)_{i,j}\}_{i,j \in [r]} \subseteq K$ such that for all $i \in [r]$ we have $\mu (\sigma,v_i) = \Sigma_{k=1}^r \left(\rho_\mu\right)_{i,k} v_k$.
  Since the action is regular, we must have $p_{i,j} := ( \mu \mapsto (\rho_\mu)_{i,k}) \in K[G]$.
  We now define $\mu^\prime \colon V \rightarrow K[G] \otimes V$ as the linear extension of $v_i \mapsto \Sigma_{k=1}^r p_{i,k} \otimes v_k$ where for $i \in [r]$.
  It should be clear that $\mu^\prime$ satisfies $ \mu \left( \sigma , v \right) = \left( \left( \epsilon_\sigma \otimes \operatorname{id} \right) \circ \mu^\prime \right) \left(v\right) $ for all $\sigma \in G$ and $v \in V$.
  This shows that $\mu$ is a rational linear action.
  % By assumption there exists a finite-dimensional $G$-stable vector space $W \subseteq V$ such that $v \in W$.
  % Choose a basis $\{w_1, \ldots, w_r\}$ of $W$, where $w_1 = v$ (this is done to make some terms easier, but choosing any basis also works).
  % Since $\left. \mu \right|_{G\times W}$ is a rational representation, we have $D_\mu (\sigma) \in \operatorname{GL}(W)$ for all $\sigma \in G$, and since our action is also regular, there exist $\{p_i\}_{i \in [r]} \subseteq K[G]$ such that for all $\sigma \in G$ we have $\mu(\sigma, v) = D_\mu(\sigma)(v) = \Sigma_{i=1}^r p_i(\sigma) \cdot w_i$.
  % Now define $\mu^\prime (v) := \Sigma_{i=1}^r p_i \otimes w_i$.
  % We shall now check that this is well-defined:
  \end{proof}
% \begin{definition}[Rational Representation]
%   Let $G$ be a linear algebraic group.
%   A representation $V$ of $G$ is called a \textbf{rational representation}, iff its corresponding action $ G \times V \rightarrow V $ is a regular action.
% \end{definition}

% \textbf{Claim:} If $V$ is finite dimensional, the notions of the definitions coincide.\\
% \textit{Proof:} First, let $V$ be a rational representation of $G$ with basis $\{ v_1 , \ldots , v_N \}$ be a basis of $V$.
% By our assumption, we have a rational representation, therefore there exist $p_{i,j} \in K \left\lbrack G \right\rbrack$ such that $\mu\left( \sigma, v_j \right) = \Sigma_{i=1}^{N} p_{i,j}\left(\sigma\right) \cdot v_i$.
% Define $\mu^\ast \left( v_j \right) := \Sigma_{i=1}^{N} p_{i,j} \otimes v_i$.
% Now we easily see:
% \begin{equation}
%   \begin{aligned}
%     \mu\left(\sigma,v\right)
%     &= \mu \left(\sigma, \Sigma_{j=1}^N \lambda_j v_j \right) \\
%     &= \sum_{j=1}^N \lambda_j  \sum_{i=1}^N p_{i,j}\left(\sigma\right) \cdot v_i \\
%     &= \sum_{j=1}^N \lambda_j \left(\left(\epsilon_\sigma \otimes \operatorname{id} \right) \circ \mu^\ast \right) \left(v_j \right)
%     &= \left(\left(\epsilon_\sigma \otimes \operatorname{id} \right) \circ \mu^\ast \right) \left(v \right)
%   \end{aligned}
% \end{equation}
% \textit{which was to show.}

\begin{remark}
  This shows that for a finite-dimensional vector space $V$, an action is a rational linear action if and only if it defines a rational representation.
  In other words, we have shown that rational representations are exactly defined by rational linear actions on finite-dimensional vector-spaces, which justifies the choice of the names of our definitions.
\end{remark}

\begin{definition}[Invariants]
  Let $X$ be an affine $G$-variety given by $\mu \colon G \times X \rightarrow X$, $(\sigma,x) \mapsto \sigma.x$.
  We define the set of all \textbf{invariants} of $X$ as
  \begin{equation}
    X^G := \left\{\, x \in X \mid \forall g \in G : g . x = x \,\right\}
  \end{equation}
  
  The given action $\mu$ induces an action $ \bar{\mu} \colon G \times K\lbrack X\rbrack \rightarrow K\lbrack X\rbrack $, $(\sigma,f)\mapsto\sigma\cdot f$ as described in definition \ref{funrep}.
  % \begin{equation}
  %   \left( g , f \right) \longmapsto g \cdot f :=
  %   \left( x \mapsto f \left( \sigma^{-1} . x \right) \right)
  % \end{equation}
  The \textbf{invariant ring} is defined as
  \begin{equation}
    K\lbrack X\rbrack^G := \left\{ \, f \in K\lbrack X \rbrack \mid \forall \sigma \in G : \sigma \cdot f = f \, \right\}
  \end{equation}
  Its elements are referred to as $G$-\textbf{invariant}.
  As the name implies, $ K\lbrack X\rbrack^G $ defines a subring, in fact also a $K$-subalgebra, of $ K\lbrack X\rbrack $.
\end{definition}
%%% Local Variables:
%%% mode: latex
%%% TeX-master: "roughdraft"
%%% End:
\message{ !name(roughdraft.tex) !offset(-251) }

\end{document}
%%% Local Variables:
%%% mode: latex
%%% TeX-master: t
%%% End:
