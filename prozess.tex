First, how is $\operatorname{GL}_n$ an affine variety?
Consider $ K^{n \times n} \times K $, and its coordinate ring $ K\left\lbrack \lbrace Z_{i,j} \rbrace_{i,j \in \lceil n \rceil} , D \right\rbrack$.
Now define $ I := \left( \left( \sum_{\sigma \in S_n} \operatorname{sgn} \left( \sigma \right) \prod_{ i \in \lceil n \rceil } Z_{i, \sigma \left( i \right) } \right) \cdot D - 1 \right)  = \left( \operatorname{det} \left( Z \right) \cdot D - 1 \right) $ , where $ Z := \left\lbrack Z_{i,j} \right\rbrack_{i,j \in \lceil n \rceil} $.
Then $ V \left( I \right) = \lbrace \, \left( z , d \right) \mid z \in \operatorname{GL}_n , d = \operatorname{det} \left( z \right)^{-1} \, \rbrace $.
Equipped with the componentwise multiplication ($\operatorname{GL}_n$ and $K \setminus \lbrace 0 \rbrace$, respectively), this is a linear algebraic group isomorphic to $\operatorname{GL}_n$.
The coordinate ring $K\left\lbrack \operatorname{GL}_n\right\rbrack$ is isomorphic to $K \left\lbrack \lbrace Z_{i,j} \rbrace_{i,j \in \lceil n \rceil} , \operatorname{det} \left( Z \right)^{-1} \right\rbrack \subseteq K \left( \lbrace Z_{i,j} \rbrace_{i,j \in \lceil n \rceil} \right) $, and we will write it as such from now on.
\begin{definition}[Cayley's $\Omega$-Process]
  We call
  \begin{equation}
    \Omega := \sum_{\sigma \in S_n} \operatorname{sgn} \left( \sigma \right) \prod_{ i \in \lceil n \rceil } \frac{\partial}{\partial z_{i , \sigma \left( i \right)}}
  \end{equation}
\end{definition}
the \textbf{Cayley's $\Omega$-Process}.
It can also be thought of as $ \Omega = \operatorname{det} \left( \frac{\partial}{\partial Z} \right) $, where $\frac{\partial}{\partial Z} := \left\lbrack \frac{\partial}{\partial z_{i,j}} \right\rbrack_{i,j \in \lceil n \rceil} $.

\begin{lemma}
  \begin{equation}
    \left( \operatorname{det} \left( Z \right) ^{-1} \cdot \otimes \Omega \right) \circ m^\ast
    = m^\ast \circ \Omega
    = \left( \Omega \otimes \operatorname{det} \left( Z \right) ^{-1} \cdot \right) \circ m^\ast
  \end{equation}
  where I write ``$ p \cdot $'' for the operation \textit{multiply with $ p $} for a polynomial $ p \in K \left\lbrack \operatorname{GL}_n \right\rbrack $ (but don't worry, this is the only time I will make use of this notation).
\end{lemma}

\begin{proof}
  Here, we will follow the same convention as described in chapter 3: The ``left'' and ``right'' inputs of $ m $ will be represented by $ X = \left\lbrack X_{i,j} \right\rbrack_{i,j \in \lceil n \rceil} $ and $ Y = \left\lbrack Y_{i,j} \right\rbrack_{i,j \in \lceil n \rceil} $ in the occuring polynomials respectively, and the outputs $ m = \left\lbrack m_{i,j} \right\rbrack_{i,j \in \lceil n \rceil} $ are indexed the same as the inputs of the polynomials in $ Z_{1,1} , Z_{1,2} , \ldots Z_{n,n} $.\\
  Let $f \in K \left\lbrack \operatorname{GL}_n \right\rbrack $.
  Then $ f \circ m \in K \left\lbrack \left\{ X_{i,j} \right\}_{i,j \in \lceil n \rceil} , \operatorname{det} \left( X \right)^{-1} , \left\{ Y_{i,j} \right\}_{i,j \in \lceil n \rceil} , \operatorname{det} \left( Y \right)^{-1} \right\rbrack $.
  For fixed $i,j \in \lceil n \rceil $ we have
  \begin{equation}
    \begin{aligned}
    \left( \operatorname{id} \otimes \frac{\partial}{\partial Z_{i,j}} \right) \left( m^\ast \left( f \right) \right) 
    = t \left( \frac{\partial}{\partial Y_{i,j}} \left( f \circ m \right) \right) 
    = t \left( \sum_{k,l \in \lceil n \rceil} \left( \left( \frac{\partial}{\partial Z_{k,l}} f \right) \circ m \right) \cdot \frac{\partial}{\partial Y_{i,j}} m_{k,l} \right) \\
    = t \left( \sum_{k = 1}^n \left( \left( \frac{\partial}{\partial Z_{k,j}} f \right) \circ m \right) \cdot X_{k,i} \right) 
    = \sum_{k=1}^n \left( Z_{k,i} \cdot \otimes \operatorname{id} \right) \left( m^\ast \left( \frac{\partial}{\partial Z_{k,j}} f \right) \right)
    \end{aligned}
  \end{equation}
  Note the use of $ t $ as described in chapter 3 to aid in rephrasing terms.
  Successively applying this yields
  \begin{equation}
    \begin{aligned}
      \left( \operatorname{id} \otimes \Omega \right) \left( m^\ast \left( f \right) \right)
      = \sum_{\sigma \in S_n} \operatorname{sgn} \left( \sigma \right) \left( \operatorname{id} \otimes \prod_{i=1}^n \frac{\partial}{\partial Z_{i,\sigma \left( i \right)}} \right) \left( m^\ast \left( f \right) \right) \\
      = \sum_{\sigma \in S_n} \operatorname{sgn} \left( \sigma \right) \sum_{k \in \lceil n \rceil^n} \left( \prod_{i=1}^n Z_{k(i),i} \cdot \otimes \operatorname{id} \right) \left( m^\ast \left( \prod_{j=1}^n \frac{\partial}{\partial Z_{k(j),\sigma (j)}} f \right) \right) \\
      = \sum_{k \in \lceil n \rceil^n} \left( \prod_{i=1}^n Z_{k(i),i} \cdot \otimes \operatorname{id} \right) \left( m^\ast \left( \sum_{\sigma \in S_n} \operatorname{sgn} \left( \sigma \right) \prod_{j=1}^n \frac{\partial}{\partial Z_{k(j),\sigma (j)}} f \right) \right) \\
      = \sum_{k \in S_n} \left( \prod_{i=1}^n Z_{k(i),i} \cdot \otimes \operatorname{id} \right) \left( m^\ast \left( \sum_{\sigma \in S_n} \operatorname{sgn} \left( \sigma \right) \prod_{j=1}^n \frac{\partial}{\partial Z_{k(j),\sigma (j)}} f \right) \right) \\
      = \sum_{k \in S_n} \left( \prod_{i=1}^n Z_{k(i),i} \cdot \otimes \operatorname{id} \right) \left( m^\ast \left( \operatorname{sgn} (k) \Omega (f) \right) \right)
      = \left( \operatorname{det} (Z) \cdot \otimes \operatorname{id} \right) \left( m^\ast \left( \Omega (f) \right) \right)
    \end{aligned}
  \end{equation}
  This immediately shows the first equality, and the second equality is proven analogously.
\end{proof}

\begin{lemma}
  For $p \in \mathbb{N} $, $ c_{p,n} := \Omega^p \left( \operatorname{det} (Z)^p \right) = \operatorname{det} \left( \frac{\partial}{\partial Z}\right)^p \left( \operatorname{det} (Z)^p \right)$ is a nonnegative integer.
\end{lemma}

\begin{proof}
  Write $ \operatorname{det} (Z)^p = \Sigma_i a_i m_i \left( \left\{ Z_{k,l} \right\}_{k,l \in \lceil n \rceil} \right) $, where $a_i \in K$ and $m_i$ are (monic) monomials.
  Then
  \begin{equation}
    \Omega^p \left( \operatorname{det} (Z)^p \right)
    = \Sigma_i a_i m_i \left( \left\{ \frac{\partial}{\partial Z_{k,l}} \right\}_{k,l \in \lceil n \rceil} \right) \left( \Sigma_j a_j m_j \left( \left\{ Z_{k,l} \right\}_{k,l \in \lceil n \rceil} \right) \right)
\end{equation}
Notice that $ m_i \left( \left\{ \frac{\partial}{\partial Z_{k,l}} \right\}_{k,l \in \lceil n \rceil} \right) \left( m_j \left( \left\{ Z_{k,l} \right\}_{k,l \in \lceil n \rceil} \right) \right) $ is zero for $ i \neq j $ and a strictly positive integer for $ i = j $. Therefore
\begin{equation}
  c_{p,n}
  = \sum_i a_i^2 m_i \left( \left\{ \frac{\partial}{\partial Z_{k,l}} \right\}_{k,l \in \lceil n \rceil} \right) \left( m_i \left( \left\{ Z_{k,l} \right\}_{k,l \in \lceil n \rceil} \right) \right) \in \mathbb{N}_{>0}
\end{equation}
in particular $c_{p,n} \neq 0$.
\end{proof}
Now, finally, we have the tools to see the following way of expressing the Reynolds Operator.
\begin{theorem}
  For $ p \in \mathbb{N} $ and $ \tilde{f} \in K \left\lbrack \left\{ Z_{i,j} \right\}_{k,l \in \lceil n \rceil} \right\rbrack_{pn} $, define for $ f = \frac{\tilde{f}}{\operatorname{det}(Z)^p}$:
  \begin{equation}
    R \left( f \right) := \frac{\Omega^p \tilde{f}}{c_{p,n}}
  \end{equation}
  The linear extension of this (mapping anything else in $K \left\lbrack \operatorname{GL}_n \right\rbrack$ to zero), defines the Reynolds Operator $R_{\operatorname{GL}_n}$, which makes $\operatorname{GL}_n$ \textit{linearly reductive}.
\end{theorem}

\begin{proof}
  First, check that this is well defined:
  For any such term, expanding the fraction by $ \operatorname{det} (Z)^q $ will yield the same result.
  Also, $\Omega^p$ is linear for any $p \in \mathbb{N}$.
  Now we show that $R$ is $\operatorname{GL}_n$-invariant.
  First, I will introduce a notation:
  For $f \in K \left\lbrack \operatorname{GL}_n \right\rbrack $ and $\alpha \in \operatorname{GL}_n$, define $ \alpha \dot{\phantom{.}} f := \left( x \mapsto f \left( x \alpha^{-1} \right) \right) $.
  This is \textit{not} an action, but a right action (normal actions should be called ``left actions'').
  Let $p \in \mathbb{N}$, $ \tilde{f} \in K \left\lbrack \operatorname{GL}_n \right\rbrack_{pn} $ and $ f := \frac{\tilde{f}}{\operatorname{det}(Z)^p} $.
  For $ \beta , \gamma \in \operatorname{GL}_n $, we notice
  \begin{equation}
    \begin{aligned}
      R \left( \beta . f \right) (\gamma)
      = R \left( \frac{ \operatorname{det} (\beta)^p \cdot \beta . \tilde{f}}{\operatorname{det} (Z)^p} \right) (\gamma)
      = \frac{ \operatorname{det} (\beta)^p \cdot \Omega^p \left( \beta . \tilde{f} \right) (\gamma) }{ c_{p,n} } \\
      = \frac{1}{c_{p,n}} \cdot \left( \epsilon_{\beta^{-1}} \otimes \epsilon_\gamma \right) \left( \left( \left( \operatorname{det}(Z)^{-p} \cdot \otimes \Omega^p \right) \circ m^\ast \right) \left( \tilde{f} \right) \right) \\
      = \frac{1}{c_{p,n}} \cdot \left( \epsilon_{\beta^{-1}} \otimes \epsilon_\gamma \right) \left( \left( \left( \Omega^p \otimes \operatorname{det}(Z)^{-p} \cdot \right) \circ m^\ast \right) \left( \tilde{f} \right) \right) \\
      = \frac{ \Omega^p \left( \gamma^{-1} \dot{\phantom{.}} \tilde{f} \right) (\beta^{-1}) \cdot \operatorname{det} \left (\gamma^{-1} \right)^p }{ c_{p,n} }
      = R \left( \frac{ \gamma^{-1} \dot{\phantom{.}} \tilde{f} \cdot \operatorname{det} \left( \gamma^{-1} \right)^p }{\operatorname{det} (Z)^p} \right) \left( \beta^{-1} \right) \\
      = R \left( \gamma^{-1} \dot{\phantom{.}} f \right) \left( \beta^{-1} \right)
    \end{aligned}
  \end{equation}
  Since each $ \frac {\partial}{ \partial Z_{i,j} } $ lowers the degree of a monomial by one or maps it to zero, $R$ maps to $K$, and therefore for $ \delta \in \operatorname{GL}_n $ and $ g \in K \left\lbrack \operatorname{GL}_n \right\rbrack $ we have $ R(g)(\delta) = R(g) \in K $.
  We then get for all $ \alpha \in \operatorname{GL}_n $
  \begin{equation}
    \begin{aligned}
      R \left( \alpha . f \right)
      = R \left( \alpha . f \right) \left( I_n \right)
      = R \left( I_n^{-1} \dot{\phantom{.}} f \right) \left( \alpha^{-1} \right) \\
      = R \left( I_n^{-1} \dot{\phantom{.}} f \right)
      = R \left( I_n \dot{\phantom{.}} f \right) 
      = R (f)
  \end{aligned}
  \end{equation}
  which shows the $\operatorname{GL}_n$-invariance.
  Finally, the definition immediately gives us that $R$ restricted to $K$ is the identity.  As mentioned in \ref{unique}, the uniqueness of the Reynolds Operator implies $ R = R_{\operatorname{GL}_n} $.
\end{proof}

Now we will look at the Reynolds Operator $R_{\operatorname{SL}_n}$.

\begin{corollary}
  With the convention of $ K \left\lbrack \operatorname{GL}_n \right\rbrack = K \left\lbrack \left\{ Z_{k,l} \right\}_{k,l \in \lceil n \rceil} , \operatorname{det} (Z) ^{-1} \right\rbrack $, view $ K \left\lbrack \operatorname{SL}_n \right\rbrack = K \left\lbrack \operatorname{GL}_n \right\rbrack / I $ where $ I = \left( \operatorname{det} (Z) -1 \right) $.
  Now, for $ p \in \mathbb{N} $ and $ \tilde{f} \in K \left\lbrack \left\{ Z_{i,j} \right\}_{k,l \in \lceil n \rceil} \right\rbrack_{pn} $ define for $ f = \frac{\tilde{f}}{\operatorname{det}(Z)^p} + I $:
  \begin{equation}
    R \left( f \right) := \frac{\Omega^p \tilde{f}}{c_{p,n}} + I
  \end{equation}
  The linear extension of this (mapping anything else in $K \left\lbrack \operatorname{SL}_n \right\rbrack$ to zero), defines the Reynolds Operator $R_{\operatorname{SL}_n}$, making $\operatorname{SL}_n$ \textit{linearly reductive}.
\end{corollary}

\begin{proof}
  First, we will show $ K \left\lbrack \operatorname{GL}_n \right\rbrack ^{\operatorname{SL}_n} = K \left\lbrack \operatorname{det} (Z) , \operatorname{det} (Z) ^{-1} \right\rbrack $ (action by left multiplication).
  Let $g \in K \left\lbrack \operatorname{GL}_n \right\rbrack ^{ \operatorname{SL}_n }$, and let $ \alpha \in \operatorname{GL}_n $.
  Note that $ \frac{1}{n \operatorname{det} (\alpha)} \alpha \in \operatorname{SL}_n $.
  Define $ h := \left( \beta \mapsto g \left( n \operatorname{det} (\beta) I_n \right) \right) \in K \left\lbrack \operatorname{det} (Z) , \operatorname{det} (Z) ^{-1} \right\rbrack $.
  Now
  \begin{equation}
    \begin{aligned}
      g ( \alpha )
      = \left( \frac{1}{n \operatorname{det} (\alpha)} \alpha \right) . g \, (\alpha)
      = g \left( n \operatorname{det} (\alpha) \alpha ^{-1} \alpha \right) \\
      = g \left( n \operatorname{det} (\alpha) I_n \right)
      = h (\alpha)
    \end{aligned}
  \end{equation}
  Therefore $ g = h \in K \left\lbrack \operatorname{det} (Z) , \operatorname{det} (Z) ^{-1} \right\rbrack $.
  Conversely it is easy to see that $ K \left\lbrack \operatorname{det} (Z) , \operatorname{det} (Z) ^{-1} \right\rbrack \subseteq K \left\lbrack \operatorname{GL}_n \right\rbrack ^{\operatorname{SL}_n } $.

  Now, define $\hat{R} \colon K \left\lbrack \operatorname{GL}_n \right\rbrack \longrightarrow \left\lbrack \operatorname{GL}_n \right\rbrack ^{\operatorname{SL}_n} $ as follows: \\
  For $ p,r \in \mathbb{N} $, $ \tilde{f} \in K \left\lbrack \left\lbrace Z_{k,l \in \lceil n \rceil} \right\rbrace \right\rbrack_{rn} $, and $ f = \frac{\tilde{f}}{\operatorname{det} (Z)^p } $, define
  \begin{equation}
    \hat{R} (f) := \operatorname{det} (Z)^{r-p} \cdot \frac{ \Omega^r \tilde{f} }{ c_{r,n} }
    = \operatorname{det} (Z)^{r-p} \cdot R_{\operatorname{GL}_n} \left( \frac{\tilde{f}}{\operatorname{det} (Z) ^r} \right)
  \end{equation}
  and as before we define the images of the other elements by linear extension.
  Well-definedness follows from the same observations as in the proof of the theorem (and $ r-p = (r+q)-(p+q) $ for $ q \in \mathbb{N} $).
  This map is the identity on $ K \left\lbrack \operatorname{GL}_n \right\rbrack ^{\operatorname{SL}_n }$:
  If $ f \in K \left\lbrack \left\lbrace Z_{k,l \in \lceil n \rceil} \right\rbrace \right\rbrack_{rn} $ and $ f \in K \left\lbrack \operatorname{GL}_n \right\rbrack ^{\operatorname{SL}_n} $, then $ \tilde{f} = \lambda \operatorname{det} (Z)^r $ with $ \lambda \in K $.
  Therefore we can say w.l.o.g. either $p=0$ or $r=0$.
  Then it is quite clear that then $f$ gets mapped to $f$.
  Finally, the $\operatorname{SL}_n$-invariance also follows from the last theorem: Let $\alpha \in \operatorname{SL}_n$.
  Then
  \begin{equation}
    \begin{aligned}
      \hat{R} (\alpha . f)
      = \hat{R} \left( \frac{ \operatorname{det} (\alpha)^p \cdot \alpha . \tilde{f} }{ \operatorname{det} (Z)^p } \right) 
      = \operatorname{det} (Z)^{r-p} \cdot \hat{R}_{\operatorname{GL}_n} \left( \frac{ \operatorname{det} (\alpha)^p \cdot \alpha . \tilde{f} }{ \operatorname{det}(Z)^r } \right) \\
      = \operatorname{det} (Z)^{r-p} \cdot \hat{R}_{\operatorname{GL}_n} \left( \frac{ \operatorname{det} (\alpha)^r \cdot \alpha . \tilde{f} }{ \operatorname{det}(Z)^r } \right) \\
      = \operatorname{det} (Z)^{r-p} \cdot \hat{R}_{\operatorname{GL}_n} \left( \alpha . \left( \frac{ \tilde{f} }{ \operatorname{det} (Z)^r } \right) \right) \\
      = \operatorname{det} (Z)^{r-p} \cdot \hat{R}_{\operatorname{GL}_n} \left( \frac{ \tilde{f} }{ \operatorname{det} (Z)^r } \right)
      = \hat{R} (f)
    \end{aligned}
  \end{equation}
  where we used $ \operatorname{det} (\alpha)^p = 1 = \operatorname{det} (\alpha)^r $ and the $\operatorname{GL}_n$-invariance of $R_{\operatorname{GL}_n}$.
  Thus we have shown that $\hat{R}$ is the Reynolds-Operator for the action of $\operatorname{SL}_n$ on $\operatorname{GL}_n$ by left-multiplication.

  Lastly, this shows our proposed statement $ R = R_{\operatorname{SL}_n} $, since $ \operatorname{det} (Z) \sim 1 $.
\end{proof}

%%% Local Variables:
%%% mode: latex
%%% TeX-master: "roughdraft"
%%% End:
