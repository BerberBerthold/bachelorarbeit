\begin{definition}[Reynolds Operator]
  Let $ V $ be a rational representation of a linear algebraic group $ G $.
  A $ G $-invariant linear projection $ K\lbrack V\rbrack \longrightarrow K\lbrack V\rbrack^G $ is called a \textbf{Reynolds Operator}.
\end{definition}

\begin{remark}\label{unique}
  If a Reynolds Operator exists, it is unique (\textbf{??}).
  See \cite[p.39f]{DK15}: In the proof of the equivalences, in the step ``(b)$\implies$(c)'', only the existence of the Reynolds operator is needed.
  Therefore, the existence of the Reynolds Operator already implies its uniqueness (\textbf{??}).
\end{remark}

\begin{definition}[linearly reductive]
  A group G is called \textbf{linearly reductive} iff there exists a Reynolds operator for the regular action $ G \times G \longrightarrow G $ by left multiplication $ R_G \colon K\lbrack G \rbrack \longrightarrow K\lbrack G \rbrack^G = K $.
\end{definition}

\begin{remark}
  We could have also defined linear reductive groups as such, for which every regular action has a Reynolds Operator.
  We will prove that this is in fact equivalent.
\end{remark}

Now we want to define an algebra structure on bla and an action bla.

Define the multiplication on $ K \left\lbrack G \right\rbrack^\ast $, denoted by $\ast$, as follows:  For $\alpha, \beta \in K \left\lbrack G \right\rbrack^\ast$:  
\begin{equation}
  \alpha \ast \beta := \left( \alpha \otimes \beta \right) \circ m^\ast
\end{equation}
More slowly: For $f \in K \left\lbrack G \right\rbrack$ we get $m^\ast \left( f \right) = \Sigma_i g_i \otimes h_i$ (with $g_i , h_i \in K \left\lbrack G \right\rbrack$), therefore the Kronecker-product gives us
\begin{equation}
  \left( \alpha \ast \beta \right) \left( f \right) = \Sigma_i \alpha \left( g_i \right) \otimes \beta \left( h_i \right) = \Sigma_i \alpha \left( g_i \right) \beta \left( h_i \right)
\end{equation}
As usual, we identify $K \otimes K$ with $K$ canonically.  
% % BEGIN COMPLETE UTTER BULLSHIT PLEASE DON'T READ
%
% Let us look at this more concretely.
% Let $f \in K \left\lbrack G \right\rbrack$, $\alpha,\beta \in K \left\lbrack G \right\rbrack^\ast$.
% Write $f= \Sigma_{E \in \mathbb{N}^{n \times n}} \quad \lambda_E \cdot X^E \cdot \operatorname{det}\left( X \right)^{-e}$.
% Then we can compute:
% \begin{equation}
%   m^\ast \left( f \right) = \sum_{E \in \mathbb{N}^{n \times n}} \lambda_E \cdot X^E \cdot \operatorname{det}\left( X \right)^{-e} \otimes X^E \cdot \operatorname{det}\left( X \right)^{-e}
% \end{equation}
% Note that $\operatorname{det}$ is multiplicative.
% We then conclude:
% \begin{equation}
%   \left( \alpha \ast \beta \right) \left( f \right) = \sum_{E \in \mathbb{N}^{n \times n} } \lambda_E \cdot \alpha \left( X^E \cdot \operatorname{det}\left( X \right)^{-e} \right) \cdot \beta \left( X^E \cdot \operatorname{det}\left( X \right)^{-e} \right)
% \end{equation}
% We here see that $\alpha \ast \beta \in K \left\lbrack G \right\rbrack^\ast$.
%
% % END COMPLETE UTTER BULLSHIT ARIGATOU GOZAIMASU

\textbf{Example:} TODO

\smallskip
\textbf{Claim:} The multiplication $\ast$ makes $K \left\lbrack G \right\rbrack^\ast$ into an associative algebra with the neutral element $ \epsilon := \epsilon_e$ (Note: $\epsilon_\sigma \left( f \right) = f \left( \sigma \right)$).

\textit{Proof:} First, a small observation:
\begin{equation}
  \left(m^\ast \otimes \operatorname{id} \right) \circ m^\ast = \left( \operatorname{id} \otimes m^\ast \right) \circ m^\ast
\end{equation}
This is true because $m$ (and $ \otimes $) is associative.
Then, for $\delta, \gamma, \varphi \in K \left\lbrack G \right\rbrack^\ast$:
\begin{equation}
  \begin{aligned}
  \left( \delta \ast \gamma \right) \ast \varphi
  = \left( \left( \left( \delta \otimes \gamma \right) \circ m^\ast \right) \otimes \varphi \right) \circ m^\ast
  = \left( \delta \otimes \gamma \otimes \varphi \right) \circ \left( m^\ast \otimes \operatorname{id} \right) \circ m^\ast \\
  = \left( \delta \otimes \gamma \otimes \varphi \right) \circ \left( \operatorname{id} \otimes m^\ast \right) \circ m^\ast
  = \left( \delta \otimes \left( \left( \gamma \otimes \varphi \right) \circ m^\ast \right) \right) \circ m^\ast
  = \delta \ast \left( \gamma \ast \varphi \right)
  \end{aligned}
\end{equation}
showing the associativity.
The second equation is easily checked (rewrite as described in the beginning of chapter 3).
Also, it should be clear that $\epsilon$ is the neutral element.
This concludes that $K \left\lbrack G \right\rbrack^\ast$ is an associative algebra, \textit{which was to show}.

Now, with this $\mu^\ast$, we can extend the operation from $\{\, \epsilon_\sigma \mid \sigma \in G \, \} $ to $K \left\lbrack G \right\rbrack^\ast$, defining an action $K \left\lbrack G \right\rbrack^\ast \times V \longrightarrow V$:
\begin{equation}
  \delta \cdot v := \left(\left( \delta \otimes \operatorname{id} \right) \circ \mu^\ast \right) \left(v\right)
\end{equation}

Now here blibla show that reynolds operator exists always blabla

then show lemmas and hilberts finiteness theorem...


%%% Local Variables:
%%% mode: latex
%%% TeX-master: "roughdraft"
%%% End:
