\begin{definition}[Reynolds Operator]
  Let $ V $ be a rational representation of a linear algebraic group $ G $.
  A $ G $-invariant linear projection $ K\lbrack V\rbrack \longrightarrow K\lbrack V\rbrack^G $ is called a \textbf{Reynolds Operator}.
\end{definition}

\begin{remark}\label{unique}
  If a Reynolds Operator exists, it is unique (?).
  See \cite[p.39f]{DK15}: In the proof of the equivalences, in the step ``(b)$\implies$(c)'', only the existence of the Reynolds operator is needed.
  Therefore, the existence of the Reynolds Operator already implies its uniqueness (?).
\end{remark}

\begin{definition}[linearly reductive]
  A group G is called \textbf{linearly reductive} iff there exists a Reynolds operator for the regular action $ G \times G \longrightarrow G $ by left multiplication $ R_G \colon K\lbrack G \rbrack \longrightarrow K\lbrack G \rbrack^G = K $.
\end{definition}

\begin{remark}
  We could have also defined linear reductive groups as such, for which every regular action has a Reynolds Operator.
  We will prove that this is in fact equivalent.
\end{remark}

Now we want to define an algebra structure on bla $K[G]$.

\begin{definition}
  Define the multiplication on $ K \left\lbrack G \right\rbrack^\ast $, denoted by $\ast$, as follows:  For $\alpha, \beta \in K \left\lbrack G \right\rbrack^\ast$:  
  \begin{equation}
    \alpha \ast \beta := \left( \alpha \otimes \beta \right) \circ m^\ast
  \end{equation}
  More slowly: For $f \in K \left\lbrack G \right\rbrack$ we get $m^\ast \left( f \right) = \Sigma_i g_i \otimes h_i$ (with $g_i , h_i \in K \left\lbrack G \right\rbrack$), therefore the Kronecker-product gives us
  \begin{equation}
    \left( \alpha \ast \beta \right) \left( f \right) = \Sigma_i \alpha \left( g_i \right) \otimes \beta \left( h_i \right) = \Sigma_i \alpha \left( g_i \right) \beta \left( h_i \right)
  \end{equation}
  As usual, we identify $K \otimes K$ with $K$ canonically.  
\end{definition}
% % BEGIN COMPLETE UTTER BULLSHIT PLEASE DON'T READ
%
% Let us look at this more concretely.
% Let $f \in K \left\lbrack G \right\rbrack$, $\alpha,\beta \in K \left\lbrack G \right\rbrack^\ast$.
% Write $f= \Sigma_{E \in \mathbb{N}^{n \times n}} \quad \lambda_E \cdot X^E \cdot \operatorname{det}\left( X \right)^{-e}$.
% Then we can compute:
% \begin{equation}
%   m^\ast \left( f \right) = \sum_{E \in \mathbb{N}^{n \times n}} \lambda_E \cdot X^E \cdot \operatorname{det}\left( X \right)^{-e} \otimes X^E \cdot \operatorname{det}\left( X \right)^{-e}
% \end{equation}
% Note that $\operatorname{det}$ is multiplicative.
% We then conclude:
% \begin{equation}
%   \left( \alpha \ast \beta \right) \left( f \right) = \sum_{E \in \mathbb{N}^{n \times n} } \lambda_E \cdot \alpha \left( X^E \cdot \operatorname{det}\left( X \right)^{-e} \right) \cdot \beta \left( X^E \cdot \operatorname{det}\left( X \right)^{-e} \right)
% \end{equation}
% We here see that $\alpha \ast \beta \in K \left\lbrack G \right\rbrack^\ast$.
%
% % END COMPLETE UTTER BULLSHIT ARIGATOU GOZAIMASU

\textbf{Example:} TODO

\begin{proposition}
The multiplication $\ast$ makes $K \left\lbrack G \right\rbrack^\ast$ into an associative algebra with the neutral element $ \epsilon := \epsilon_e$ (Note: $\epsilon_\sigma \left( f \right) = f \left( \sigma \right)$).
\end{proposition}

\begin{proof}
  First, a small observation:
  \begin{equation}
    \left(m^\ast \otimes \operatorname{id} \right) \circ m^\ast = \left( \operatorname{id} \otimes m^\ast \right) \circ m^\ast
  \end{equation}
  This is true because $m$ (and $ \otimes $) is associative.
  Then, for $\delta, \gamma, \varphi \in K \left\lbrack G \right\rbrack^\ast$:
  \begin{equation}
    \begin{aligned}
      \left( \delta \ast \gamma \right) \ast \varphi
      = \left( \left( \left( \delta \otimes \gamma \right) \circ m^\ast \right) \otimes \varphi \right) \circ m^\ast
      = \left( \delta \otimes \gamma \otimes \varphi \right) \circ \left( m^\ast \otimes \operatorname{id} \right) \circ m^\ast \\
      = \left( \delta \otimes \gamma \otimes \varphi \right) \circ \left( \operatorname{id} \otimes m^\ast \right) \circ m^\ast
      = \left( \delta \otimes \left( \left( \gamma \otimes \varphi \right) \circ m^\ast \right) \right) \circ m^\ast
      = \delta \ast \left( \gamma \ast \varphi \right)
    \end{aligned}
  \end{equation}
  showing the associativity.
  The second equation is easily checked %(rewrite as described in the beginning of chapter \ref{pw})
  .
  Also, it should be clear that $\epsilon$ is the neutral element.
  This concludes that $K \left\lbrack G \right\rbrack^\ast$ is an associative algebra.
\end{proof}

Now we can formally define $K [G]^\ast$-actions.

\begin{definition}\label{da}
  Let $G$ act regularly on a vectorspace $V$ via $\mu$, from which we retrieve $\mu^\ast$ as described in definition \ref{rr}.  $K[G]^\ast$ then acts on $V$ as follows:
  \begin{equation}
    \delta \cdot v := \left(\left( \delta \otimes \operatorname{id} \right) \circ \mu^\prime \right) \left(v\right)
  \end{equation}
\end{definition}

\begin{remark}
  If we look at definition \ref{rr}, we can see that this newly defined $K[G]^\ast$-action is an extension of the given $G$-action in the following way:
  The subgroup $\left\{\, \epsilon_\sigma \mid \sigma \in G \,\right\}$ of $K[G]^\ast$ is isomorphic to $G$, and its induced action coincides with the given action:
  For $\sigma \in G$ and for $v \in V$ we have:
  \begin{equation}
    \sigma . v = \epsilon_\sigma \cdot v
  \end{equation}
\end{remark}

\begin{remark}
  The subalgebra
  \begin{equation}
    \left\{\, \delta \in K[G]^\ast \mid \forall f,g \in K[G] : \delta (fg) = \delta (f) g(e) + f(e)\delta (g) \,\right\}
  \end{equation}
  is called the \textbf{Lie algebra}.
\end{remark}

\begin{proposition}
  Let $G$ be linearly reductive, and let $G$ act regularly on an affine variety $X$, which induces a rational $G$-action on $K[X]$ (maybe prove this? easy...).
  Then, the following the map
  \begin{align}
    R \colon K[X] \longrightarrow K[X]^G && f \mapsto R_G \cdot f
  \end{align}
  defines a Reynolds operator.
\end{proposition}

\begin{proof}
  As per our construction from definition \ref{da}, the linearity of this map should be clear.
  Let us now check that $R$ does map polynomials to invariant polynomials.
  For this, let $f \in K[X]$, $\sigma \in G$ and $x \in X$.
  Write $\mu^\prime (f) = \Sigma_i p_i \otimes g_i \in K[G] \otimes K[X] $.
  Now we compute:
  \begin{equation}
    \begin{aligned}
      &\sigma . \left( R_G \cdot f \right) (x)
      &=& \left( R_G \otimes id \right) \left( \mu^\prime(f) \right) \left( \sigma^{-1}.x \right) \\
      &&=& \Sigma_i R_G \left( p_i \right)  g_i \left( \sigma^{-1} . x \right) \\
      &&=& \Sigma_i R_G (p_i) \cdot (\sigma . g_i) (x)\\
      &&=& \Sigma_i R_G (\sigma \dot{\phantom{.}} p_i) \cdot (\sigma.g_i) (x)\\
      &&=&(R_G \otimes \operatorname{id}) \left( \Sigma_i \sigma \dot{\phantom{.}} p_i \otimes \sigma.g_i \right) (x) \\ 
      &&=& (R_G \otimes \operatorname{id}) (\mu^\prime (f)) (x)
      &=& (R_G \cdot f) (x)
    \end{aligned}
\end{equation}

  not done yet pls finish, also 2nd to last equation elaborate...
\end{proof}

\begin{lemma}
  If $G$ is linearly reductive and $X$ is a $G$-variety, we have:
  \begin{enumerate}
  \item For all $G$-stable subspaces $W \subseteq K[X]$ we have $R(W) = W^G$
  \item For all $f \in K[X]^G$ and for all $g \in K[X]$ we have $R(fg) = fR(g)$
  \end{enumerate}
\end{lemma}

\begin{corollary}[2.2.8]
  
\end{corollary}

\begin{corollary}[2.2.9]

\end{corollary}

\begin{theorem}[Hilbert's Finiteness Theorem]
  If $G$ is linearly reductive and $V$ is a rational $G$-representation, the invariant ring $K[V]^G$ is finitely generated.
\end{theorem}

\begin{proof}
  
\end{proof}

\begin{corollary}
  If $G$ is a linearly reductive group and $X$ is an affine $G$-variety, $K[X]^G$ is finietely generated.
\end{corollary}

%%% Local Variables:
%%% mode: latex
%%% TeX-master: "roughdraft"
%%% End:
