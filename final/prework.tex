\subsection{Notation}

In the following, $K$ is a field of characteristic $0$ and $G$ a linear algebraic group, that is a group whose set is an affine variety, and whose multiplcation and inversion are morphisms of affine varieties.
For us, zero is an element of the natural numbers.
Furthermore, for $n \in \mathbb{N}$ we write $[n] := \{\, m \in \mathbb{N} \mid 1 \leq m \leq n \,\}$
For an affine variety $X$, we denote by $K[X]$ the coordinate ring of $X$.
For a finite-dimensional vector-space $V$, we denote by $X_i$ the coordinate functions for a given (often a canonical) basis.
For a set of functions in the coordinate ring $F \subseteq K[X]$ we denote by $Z(F)$ the zero set of $F$. % and for a set of points $P \subseteq X$ we denote by $V(P)$ the vanishing ideal of $P$.
For a subset of a ring $M$, $(M)$ denotes the ideal generated by $M$.

\subsection{Concepts From Algebraic Geometry}

\begin{proposition}[Rabinovich Trick]\label{rabbi}
  Let $V = K^n$ for some $n \in \mathbb{N}$.
  For a polynomial $p \in K[V] = K[\{X_i\}_{i\in[n]}]$, the set $ X_p := \{\, v \in V \mid p(v) \neq 0 \,\}$ has the structure of an affine variety with the coordinate ring $K[X_p] = K[\{X_i\}_{i \in [n]}, p^{-1}]$.
\end{proposition}

\begin{proof}
  The set $X_p$ is not an algebraic set itself.
  The trick (the ``Rabinovich-trick'') is ``adding an additional variable $X_0$'', that means to consider $X_p$ as a subset of $K \times V$.
  We do this as follows:
  Consider the algebraic set $\tilde{X}_p := Z \left( X_0 \cdot p -1 \right) \subseteq K \times V$.
  We notice that $\tilde{X}_p = \{\, (p(v)^{-1},v) \in K \times V \mid v \in X_p \,\}$.
  This means that $X_p$ corresponds to $\tilde{X}_p$ via the bijection $\Phi \colon X_p \longrightarrow \tilde{X}_p$, $ v \leftrightarrow (1/p(v),v)$.
  The coordinate ring of $\tilde{X}_p$ can be written as $K[\bar{X_0}, \{\bar{X}_i\}_{i \in [n]}]$, where $\bar{X}_i = X_i \operatorname{mod} X_0 \cdot p -1$.
  Let $x \in X_p$.
  We have $\bar{X}_0 (\Phi(v)) = p(x)^{-1}$ and for $ i \in [n] $ we have $\bar{X_i}(\Phi(x)) = v_i$.
  This shows our claim: $X_p$ has the structure of an affine variety with the coordinate ring $K[X] = K[\{X_i\}_{i \in [n]}, p^{-1}]$.
  % We see that $X_p$ corresponds to $\tilde{X}_p$ by noticing that $\tilde{X}_p = \{\, (1/p(v), v) \in K \times V \mid v \in X_p \,\}$, which therefore means we have the one-to-one correspondence $(1/p(v),v) \leftrightarrow v$.
  % By definition $\tilde{X}_p$ is an affine variety, and it is also easy to see that $K[X_p] \cong K[\tilde{X}_p] = K[K\times V] / (X_0 \cdot p -1)$ via $X_0 \leftrightarrow p^{-1}$, including evaluations with the above described correspondence of points, which is easy to check.
\end{proof}

\begin{example}[The General Linear Group $\operatorname{GL}_n$]
  One of the most important examples is the general linear group $\operatorname{GL}_n$, which will be an essential theme in my work.
  By the above proposition this group is an affine variety via $p = \operatorname{det}$ with the coordinate ring $K[\{X_{i,j}\}_{i,j \in [n]}, \operatorname{det}^{-1}]$.
  This makes $\operatorname{GL}_n$ into a \textit{linear algebraic group}, that is a group which is an affine variety whose group operations of the multiplication and inversion are morphisms of affine varieties:
  The multiplication is just a polynomial function in each entry
  For the inversion each entry is a fraction of polynomials with $\operatorname{det}$ as the quotient, which means that each entry is in $K[\operatorname{GL}_n]$.
\end{example}

\begin{definition}[Algebraic Cohomomorphism For Product Spaces]\label{coh}
% We denote by $m$ the group multiplication of the group $G$.
% We want to view the pullback of $m$ as a map $m^\ast : K[G] \longrightarrow K[G] \otimes K[G]$, which makes sense, because $m$ and $\otimes$ are associative.
% The strict pullback, which I will call $ \hat{m} $, should be a map of the type $ K[G] \longrightarrow K[ G \times G] $, where $ f \mapsto f \circ m $.
% If we want to give the variables names, we can equivalently say it is a map $ \left. K[Z] \right|_G \longrightarrow \left. K[X,Y] \right|_{G \times G} $, where $ Z = \lbrace Z_1 , \ldots , Z_k \rbrace $, $X$ and $Y$ analogously (here, $ m $ canonically takes its left input via $ X $ and its right input via $ Y $).

  Let $m \colon U_1 \times U_2 \longrightarrow W$ be a morphism of affine varieties.
  The strict algebraic cohomomorphism, which we shall call $\hat{m}$, is a map of the type $K[W] \longrightarrow K[U_1 \times U_2]$.
  We have $ K[U_1 \times U_2] = K[\{X_k\}_{k\in[r]},\{Y_l\}_{l\in[s]}]$, where $\{X_k\}_{k\in[r]}$ and $\{Y_l\}_{l\in[s]}$ are generators of $K[U_1]$ and $K[U_2]$ respectively.
  
  % If we want to give the variables names, we can equivalently say it is a map $ \left. K[Z] \right|_W \longrightarrow \left. K[X,Y] \right|_{U_1 \times U_2} $, where $ Z = \lbrace Z_1 , \ldots , Z_k \rbrace $ $X$ and $Y$ analogously (here, $ m $ canonically takes its left input via $ X $ and its right input via $ Y $).
  
  We define the following map:
  % \begin{equation}
  %   \begin{aligned}
  %     t \colon K[\{X_k\}_{k\in[r]},\{Y_l\}_{l\in[s]}]
  %     & \longrightarrow K[\{X_k\}_{k\in[r]}] \otimes K[\{Y_l\}_{l\in[s]}] \\
  %     \sum_i \lambda_i \prod_j X_{j}^{d_{i,j}} \prod_k Y_{k}^{e_{i,k}} &\longmapsto \sum_i \lambda_i \prod_j X_{j}^{d_{i,j}} \otimes \prod_k Y_{k}^{e_{i,k}}
  %   \end{aligned}
  % \end{equation}
    \begin{equation}
    \begin{aligned}
      t \colon K[U_1 \times U_2]
      & \longrightarrow K[U_1] \otimes K[U_2]\\
      \sum_i \lambda_i \prod_j X_{j}^{d_{i,j}} \prod_k Y_{k}^{e_{i,k}} &\longmapsto \sum_i \lambda_i \prod_j X_{j}^{d_{i,j}} \otimes \prod_k Y_{k}^{e_{i,k}}
    \end{aligned}
  \end{equation}
  This is independent of the choice of generators and independent of the representatives and therefore well-defined.
  It is even an isomorphism.
  Now, finally, we define $m^\ast := t \circ \hat{m} : K[W] \longrightarrow K[U_1] \otimes K[U_2]$.
  Most literature still also calls $m^\ast$ the algebraic cohomomorphism of $m$, the $K$-algebra of all evaluation maps induced by $K[U_1] \otimes K[U_2]$ is equal to $K[U_1 \times U_2]$.
\end{definition}

\begin{remark}
One might ask why we want to look at these objects $ m^\ast \left( f \right) $ instead of $ \hat{m} \left( f \right) $.
Really, these objects are hardly different (the spaces are isomorphic), but it helps to formalize performing operations only on the ``left part'' or the ``right part'', as we will soon see.
This is an approach that \cite{DK15} follows, but other literature such as \cite{Stu08} (and probably also Cayley) rather consider $ \hat{m} \left( f \right) $ written as $ \hat{m} \left( f \right) = f (XY) $.
To give a very simple example:
If $f \in \left. K[Z] \right|_G$, we will write $\operatorname{id} \otimes \frac{\partial}{\partial Z_i} (m^\ast f)$ as in \cite{DK15}, whereas \cite{Stu08} would write $\frac{\partial}{\partial Y_i} f(XY)$.
\end{remark}

\subsection{Concepts From Invariant Theory}

\begin{definition}[Regular Action, Rational Representation]
  Let $G$ be a linear algebraic group and $X$ an affine variety.
  We call an action $G \times X \longrightarrow X$ a \textbf{regular action}, if and only if $\mu$ is a morphism of affine varieties.
  We say \textbf{$ G $ acts regularly on $ X $}, and we also call $X$ a \textbf{$G$-variety}.

  For a finite-dimensional vector space $V$, let $\mu \colon G \times V \longrightarrow V$ be a representation in the classical sense, that is for all $g \in G$ we have $D_\mu (g) := (v \mapsto \mu(g,v)) \in \operatorname{GL}(V)$.
  We call $\mu$ a \textbf{rational representation} if and only if it is regular.
  % We can view $V$ as an affine variety with respect to a basis.
  % We call $\mu$ a \textbf{finite rational representation} iff it is regular with respect to any basis.
  
  % For a finite dimensional vector space $V$ we call $D \colon G \longrightarrow \operatorname{GL}(V)$ a \textbf{regular representation} iff $\mu_D := ((g,v) \mapsto D(g)(v))$ is a regular action.
  % If an action $\mu \colon G \times V \longrightarrow V$ is given such that $D_\mu (g) := (v \mapsto g.v) \in \operatorname{GL}(V)$, we also call $\mu$ a regular representation (such $\mu$ and $D$ are in bijection).
\end{definition}

\begin{dexample}
  If $G$ is a linear algebraic group, then the multiplication $m \colon G \times G \longrightarrow G$ defines a regular action, meaning that $G$ itself is a $G$-variety.
\end{dexample}

\begin{dexample}
  A less trivial one
\end{dexample}

% \begin{remark}
%   If $V$ is a finite-dimensional vector-space and $\mu \colon G \times V \longrightarrow V$ is a representation of $G$, then $\mu$ is a finite rational representation if and only if it is regular with respect to a single basis.
%   Really, one can define 
% \end{remark}

\begin{remark}
  A rational representation $\mu \colon G \times V \longmapsto V$ is of the following form:\\
  Consider $D_{\mu} \colon G \longmapsto \operatorname{GL}(V)$.
  If then $ a_{i,j} : G \longrightarrow K $ is the function of the $\left( i,j \right) $-entry of $D_{\mu}$, then $ a_{i,j} \in K\lbrack G\rbrack $.\\
  In fact, it is equivalent to define a representation $\mu \colon G \times V \longrightarrow V$ ($V$ finite dimensional) as rational, iff $D_{\mu} \colon G \longrightarrow \operatorname{GL}(V)$ is a map of affine varieties.
\end{remark}

\begin{definition}
  If $\mu \colon G \times V \longrightarrow V$ is a finite rational representation, we define an action on $\hat{\mu} \colon G \times V \longrightarrow V$ by $(\sigma,\varphi) \mapsto \sigma.\varphi := v \mapsto \varphi(\sigma^{-1}.v)$.
  $\hat{\mu}$ is also a finite rational representation of $G$.
\end{definition}

\begin{definition}[Invariants]
  Let $ G $ act on $ X $ regularly.
  \begin{equation}
    X^G := \left\{\, x \in X \mid \forall g \in G : g . x = x \,\right\}
  \end{equation}
  This defines a linear subspace.
  The given action induces an action $ \bar{\mu} \colon G \times K\lbrack X\rbrack \longrightarrow K\lbrack X\rbrack $ as per definition \ref{funrep}.
  % \begin{equation}
  %   \left( g , f \right) \longmapsto g \cdot f :=
  %   \left( x \mapsto f \left( \sigma^{-1} . x \right) \right)
  % \end{equation}
  The \textbf{invariant ring} of the representation is defined as
  \begin{equation}
    K\lbrack X\rbrack^G := \left\{ \, f \in K\lbrack X \rbrack \mid \forall g \in G : g . f = f \, \right\}
  \end{equation}
  As the name implies, $ K\lbrack X\rbrack^G $ defines a subalgebra of $ K\lbrack X\rbrack^G $.
\end{definition}

The general theme of my work revolves around the question of whether the invariant ring $ K\lbrack X\rbrack^G $ is finitely generated.

\textit{Hilbert's finiteness theorem} states that if the group $G$ is linearly reductive, $ K\lbrack V\rbrack^G $ is finitely generated.
The strict definition of ``linearly reductive'' is quite tricky, but we will give an alternate equivalent definition shortly.

%%% Local Variables:
%%% mode: latex
%%% TeX-master: "bachelorarbeit"
%%% End: