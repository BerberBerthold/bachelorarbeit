Let us apply the Reynolds operator with respect to an action on concrete polynomials.
Before we look at finite generators of Hilbert's nullcone (which we will talk about later), we will just look at generators of the polynomial ring.

\begin{example}
  Consider the group $G = \operatorname{SL}_2$ and the vector space $ V = \left\{ \, A \in \mathbb{R}^{2 \times 2} \mid A^T = A \, \right\} $.
  Now we will look at the following action:
  \begin{equation}
    \begin{aligned}
      \mu \colon && \operatorname{SL}_2 & \times & V && \longrightarrow & V \\
      & ( & S & , &  A & ) & \longmapsto & SAS^T
    \end{aligned}
  \end{equation}
  Now consider the following for $S \in \operatorname{SL}_2$ and $A \in V$:
  \begin{equation}
    \begin{aligned}
      S &=
      \begin{bmatrix}
        s_{1,1} & s_{1,2} \\
        s_{2,1} & s_{2,2}
      \end{bmatrix}
      & A &=
      \begin{bmatrix}
        a_{1,1} & a_{1,2} \\
        a_{2,1} & a_{2,2}
      \end{bmatrix}
      \\
      S^{-1} &=
      \begin{bmatrix}
        s_{2,2} & -s_{1,2} \\
        -s_{2,1} & s_{1,1}
      \end{bmatrix}
    \end{aligned}
  \end{equation}
  We then have
  \begin{equation}
    \begin{aligned}
      &S^{-1}.A = S^{-1}A\left(S^{-1}\right)^T  \\ &=
      \begin{bmatrix}
        a_{1,1}s_{2,2}^2 - 2a_{1,2}s_{1,2}s_{2,2} + a_{2,2}s_{1,2}^2 & -a_{1,1}s_{2,1}s_{2,2} + a_{1,2}s_{1,1}s_{2,2} + a_{1,2}s_{1,2}s_{2,1} - a_{2,2}s_{1,1}s_{1,2} \\
        - a_{1,1}s_{2,1}s_{2,2} + a_{1,2}s_{1,1}s_{2,2} + a_{1,2}s_{1,2}s_{2,1} - a_{2,2}s_{1,1}s_{1,2} & a_{1,1}s_{2,1}^2 - 2a_{1,2}s_{1,1}s_{2,1} + a_{2,2}s_{1,1}^2
      \end{bmatrix}
    \end{aligned}
  \end{equation}
  Notice that we also have
  \begin{equation}
    \begin{aligned}
      &\operatorname{det}\left( \frac{\partial}{\partial S} \right)^n
      &&= \left( \frac{\partial}{\partial S_{1,1}} \frac{\partial}{\partial S_{2,2}} - \frac{\partial}{\partial S_{1,2}} \frac{\partial}{\partial S_{2,1}} \right)^n \\
      &&&= \sum_{k=0}^n (-1)^k \binom{n}{k} \frac{\partial}{\partial S_{1,1}}^{n-k} \frac{\partial}{\partial S_{1,2}}^k \frac{\partial}{\partial S_{2,1}}^k \frac{\partial}{\partial S_{2,2}}^{n-k}
    \end{aligned}
  \end{equation}
  Now consider a monomial $ f = A_{1,1}^{r_{1,1}} A_{1,2}^{r_{1,2}} A_{2,2}^{r_{2,2}} \in K[V] $.
  We then have
  \begin{equation}
    \begin{aligned}
      & \mu^\ast (f) \\
      =& \sum_{|t_{1,1}| = r_{1,1}} \, \sum_{|t_{1,2}| = r_{1,2}} \, \sum_{|t_{2,2}| = r_{2,2}} \binom{r_{1,1}}{t_{1,1}} \binom{r_{1,2}}{t_{1,2}} \binom{r_{2,2}}{t_{2,2}}\\
      =& \sum s_{1,1}^{ t_{1,2}^{(2)} + t_{1,2}^{(4)} + t_{2,2}^{(2)} + 2t_{2,2}^{(3)}}
    \end{aligned}
  \end{equation}
\end{example}

\begin{example}
  Consider the cross ratio.
  We can view its domain as the affine variety
  \begin{equation}
    \begin{aligned}
      &X&:=&\{\, (a,b,c,d) \in (\mathbb{K}^2)^4 \mid (b_1c_2 - b_2c_1)(d_1a_2 - d_2a_1) \neq 0 \,\}\\
      &&=& \{\, (a,b,c,d) \in (\mathbb{K}^2)^4 \mid \operatorname{det}(b,c)\operatorname{det}(d,a) \neq 0 \,\}
    \end{aligned}
  \end{equation}
  as described in [\textbf{yaboi}].
  Now consider the action of $\operatorname{GL}_2$ on $X$ via pointwise application.
  The \textit{cross ratio} $\operatorname{cr} \in K[X]$ defined as follows
  \begin{equation}
    \begin{aligned}
      \operatorname{cr} \colon&&X&\longrightarrow K \\
      &&(a,b,c,d) &\longmapsto \frac{\operatorname{det}(a,b)\operatorname{det}(c,d)}{\operatorname{det}(b,c)\operatorname{det}(d,a)}
    \end{aligned}
  \end{equation}
  is an invariant under this action, as is easily seen.
  Exactly this $\operatorname{GL}_2$-invariance guarantees a coordinate-independent definition of the cross ratio in the projective space.
  The invariant ring $K[X]^{\operatorname{GL}_n}$ is finitely generated, though we can't apply the Reynolds operator directly, since $X$ is not a vector space.
\end{example}

\begin{example}
  \textbf{($K$ needs to be algebraically closed for Zariski-denseness of diagonizable matrices...)}
  
  Consider $\operatorname{GL}_n$ viewed as the group of all change-of-coordinates transformations for endomorphisms on $K^n$, that is the rational representation
  \begin{equation}
    \begin{aligned}
      \mu \colon && \operatorname{GL}_n \times K^{n.n} & \longrightarrow K^{n,n} \\
      && (\sigma,A) &\longmapsto \sigma A \sigma^{-1}
    \end{aligned}
  \end{equation}
  What are its invariants?
  The invariants are exaclty those polynomials that are independent of the choice of the basis.
  The most well-known invariant is the determinant.
  From this obvservation we can find even more:
  We can follow that the characteristic polynomial of a matrix $A$, that is $\operatorname{det} (tI_n - A)$, does not change under a change of coordinates.
  If we write
  \begin{equation}
    \begin{aligned}
      \operatorname{det} (tI_n - A) = \sum_{i=0}^n p_{n,i} (A) t^i
    \end{aligned}
  \end{equation}
  this means that every $p_{n,i}$ is an invariant polynomial in $K[K^{n.n}]$!
  This is how one usually proves that the trace is an invariant polynomial after observing that $p_{n,n-1} = \operatorname{tr}_n$.
  Are there other invariants than these $p_{n,i}$?
  No!
  To see this, we will use a little trick:
  Consider $D := \{\, \delta \in K^{n,n} \mid \delta \operatorname{diagonalizable} \,\} \subseteq K^{n,n}$.
  Since $V$ is Zariski-dense in $K^{n,n}$, we have $K[K^{n,n}] \simeq \left. K[K^{n,n}] \right|_{D}$ via $p \mapsto \left. p \right|_{D}$.
  For this reason, we will look at the evaluation of an invariant polynomial $p \in K[K^{n.n}] $ only on elements in $D$, and can deduce what polynomial it is.

  Let $p \in K[K^{n,n}]^{\operatorname{GL_n}}$.
  We define a projection onto the diagonal:
  $\pi \colon K^{n,n} \longrightarrow K^n , [A_{i,j}]_{i,j \in [n]} \longmapsto (X_{i,i})_{i \in [n]} $.
  % Now, for $A \in D$, there exists a $\sigma_A \in \operatorname{GL}_n$ such that $\sigma_A . A $ is diagonal.
  % If $\tilde{p} = p \left( [ \delta_{i,j} Z_{i,j}]_{i,j \in [n]} \right) \in K[K^n]$, then, since $p$ is an invariant, for all $A \in D$ we have
  % \begin{equation}
  %   \begin{aligned}
  %     &p(A)&=& \sigma_A^{-1} . p (A)\\
  %     &&=& p(\sigma_A .A)\\
  %     &&=&  \tilde{p} ( \pi  (\sigma_A . A))
  %   \end{aligned}
  % \end{equation}
  % As long as $\sigma .A$ is diagonal, this is true regardless of the choice of $\sigma_A$.
  Consider $\tilde{p} = p \left( [ \delta_{i,j} Z_{i,j}]_{i,j \in [n]} \right) \in K[K^n]$.
  $\tilde{p}$ is $S_n$-invariant:
  If $M_\tau \in \operatorname{GL}_n$ is the matrix corresponding to $\tau \in S_n$, then for all $\tau \in S_n$ and for all $X \in K^n$ we have
  \begin{equation}
    \begin{aligned}
      &\tau.\tilde{p} (X) &=& \tilde{p} (\tau^{-1}.X)\\
      &&=& \tilde{p} (\pi (\operatorname{diag} (\tau^{-1} . X) ))\\
      &&=& p ( \operatorname{diag}(\tau^{-1}.X)\\
      &&=& M_\tau . p (\operatorname{diag}(X))\\
      &&=& p (\operatorname{diag}(X))&=& \tilde{p} (X)
    \end{aligned}
  \end{equation}
  From the fundamental theorem of symmetric polynomials we can follow that $\tilde{p} \in \operatorname{span}\{\,e_n^i \mid i \in [n]\,\}$, say $\tilde{p} = \Sigma_{i=1}^n \lambda_i e_{n,i}$, where $e_{n,i}$ are the elementary symmetric polynomials of dimension $n$.
  Now, for a choice (!) of $\sigma_A \in \operatorname{GL}_n$ such that $\sigma_A . A$ is diagonal, we easily see that for $s(A) := \sigma_A . A$ get $p = \tilde{p} \circ \pi \circ s$, therefore $p = \Sigma_{i=1}^n \lambda_i e_{n,i} \circ \pi \circ s$.
  Now we want to show that $e_{n,i} \circ \pi \circ s = p_{n,i}$, which would conclude our claim.
  For all $A\in D$ we have
  \begin{equation}
    \begin{aligned}
      &\sum_{i=1}^n (e_{n,i} \circ \pi \circ s)(A)t^i&=&\operatorname{det}(t-\sigma_A.A)\\
      &&=&\operatorname{det}(t-A)
      &=&\sum_{i=1}^n p_{n,i}(A)t^i\\
    \end{aligned}
  \end{equation}
  Note that this is independent of the choice of $s$, which means that we don't need the axiom of choice (rigorously, as usual, rewrite $s$ as a relation for all possible $s$ instead of a choice of a function\ldots).
\end{example}

%%% Local Variables:
%%% mode: latex
%%% TeX-master: "roughdraft"
%%% End:
